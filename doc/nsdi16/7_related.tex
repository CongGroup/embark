%!TEX root = mb.tex

\section{Related Work}
\label{sec:related}

\noindent{\bf Middlebox Outsourcing:}
APLOMB~\cite{aplomb} is a practical service for outsourcing enterprise's middleboxes to the cloud, which we discussed in detail in \S\ref{sec:overview}.

\noindent{\bf Data Confidentiality:}
More generally, confidentiality of data in the cloud has been widely recognized as an important problem and researchers proposed solutions for software~\cite{Baumann:Haven}, web applications \cite{giffin:hails, Mylar},  filesystems~\cite{blaze:cfs, kallahalla:plutus, goh:sirius},  databases~\cite{popa:cryptdb},  and virtual machines~\cite{Zhang:CloudVisor}. 

mcTLS~\cite{mctls} proposed a protocol in which client and server can jointly authorize a middlebox to process certain portions of the encrypted traffic. Middleboxes become no longer transparent, and both middleboxes and end users need to adopt the protocol. CryptDB~\cite{popa:cryptdb} introduced the vision for building practical systems that compute on encrypted data, but none of its encryption schemes or systems techniques apply to our setting. The most closely related work with \sys are BlindBox~\cite{blindbox}. We compared them in \S\ref{sec:overview}.

\noindent{\bf Trace Anonymization and Inference:}
A few systems \cite{Vern:Anonymize06, Vern:Anonymize03} anonymize the packet stream for offline auditing in the hope of hiding the identity of the hosts. Unlike \sys, they break middlebox functionality and are not suitable for online processing. 

Conversely, there are some work for inferring information from encrypted traffic. Yamada et al.\cite{Yamada_IDS} show how one can perform some very limited processing on an SSL-encrypted packet. by using only the size of the data and the timing of packets.

\noindent{\bf Encryption Schemes:}
Our RangeMatch scheme is similar to order preserving encryption schemes, but no existing scheme provided both the performance and security properties we required.
The order-preserving encryption~\cite{boldyreva:ope, popa:mope} used in CryptDB is 
 $>3000$ times slower than RangeMatch (\S\ref{sec:eval}) and cannot sustain realistic network throughputs; moreover, it leaks the order of the IP addresses encrypted, while RangeMatch protects this information. 
The encryption scheme of Boneh et al.~\cite{BonehRange} enables detecting if an encrypted value matches a range and provides a similar security guarantee to RangeMatch, however, it is orders of magnitude slower than the OPE schemes which are already 3000$\times$ slower than RangeMatch. 

%  mOPE unfortunately requires that the gateway and the service provider interact for a number of roundtrips (e.g., xxx in our experiments) which is too slow and requires additional setup for this interaction, and violates requirement~\ref{req:sec} or~\ref{req:injective}, and BCLO has weak security (leaking always the top half bits of the values encrypted and the order of IP addresses across different packets, thus violating requirement~\ref{req:sec}), is too slow, and not format-preserving. 

% HERE ARE A FEW USEFUL NOTES ABOUT HOW OUR DESIGN IS DIFFERENT FROM mOPE -- THERE IS SOME SIMILARITY DUE TO TREE AND ADJUST
% we do not readjust for encryption 
% - this is expensive, we do not leak data between two encryptions 
% The tree is stored at the gateway. The tree contains as nodes the ends of the intervals as opposed to all values encoded-- thus, the tree is much smaller. firewalls have on the order of thousands such rules, so the tree is not large. also store only ranges and not everything encoded, making it smaller and fit into the gateway, etc., they need adjustments when they encrypt too, etc. -- better point to related work for this
% Difference:
% we encode different values in the tree, have a different encryption algorithm, and create a much smaller tree that can be stored at the gateway. no roundtrips any more; they don't have the deterministic property
%store the tree at the gateway.
% This tree is stored at the gateway. The tree stores edges of the interval 
% one important point is that there are ciphertext updates only for rule changes and not for regular encryption







