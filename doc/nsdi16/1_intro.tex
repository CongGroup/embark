\section{Introduction}


Shift toward SW-based MBs has enabled a new arch for MB deployment and management in which a 3rd party offers MB processing as a service. 
  Service may be hosted in a public cloud [APLOMB] or in datacenters embedded in carrier infrastructure [NFV]. 
  Both offer the same benefits (elaborate)
But both introduce a new challenge (confidentiality, etc., etc., old text)
  A recent effort, BlindBox, was the first to address this new challenge for the specific context of IDS
  BB in one sentence
  Although BB makes significant progress towards a viable solution, it still suffers from two limitations that make it impractical for deployment today 
  performance overhead: connection setup times are impractically high
  generality: BB limited to IDS and, even there, only exact-match ops.
In this paper, we present MBArk, a system that uses a combination of system and algorithmic techniques to resolve both of the above limitations for the context of MB deployments in which users reside in a trusted/managed network environment such as an enterprise or campus  (vs., say, user-in-a-cafe). This context represents the vast majority of MB usage today (true? find citation)


  1.1 Limitations of existing approaches
  Generality
  BB only addresses IDS but MB-as-service must offer more
  e.g., NAT, firewalls, WAN optimizers, proxies
  Even within IDS, BB only supports exact-match operations on encrypted data; for more complex reg-ex rules BB requires that operators identify a `pre-filter’ based on exact-match operations; if a packet matches the pre-filter, it is decrypted for further processing (`probable cause’ model)
  BB thus resorts to decrypting for X-Y% of IDS rules
  Performance
  cite connection setup times in BB


  1.2 MBArk approach and contributions  
  Start with two insights: 
  Enterprise as intermediary: In many contexts, the user resides in an administrative domain such as an enterprise or campus and this enterprise can act as an intermediary (in both a physical and administrative sense) between the user and SP.
Why is this useful? Because the enterprise has stronger trust relationships with both the user and the SP than they have with each other. This allows us to carefully relax select assumptions in BB’s trust model and this relaxation enables new designs that avoid the above limitations BB incurred (as described below)
  What contexts we aren’t addressing: home, cafe, etc.
  Range-match as primitive operation: Just as BB identified EM as the fundamental primitive needed to support IDS processing, we identify RM as a fundamental primitive for a range of other MBs.


  How MBArk uses these insights
Enterprise as intermediary (TODO: resolve)
  → Enterprise acts as administrative intermediary: SP has a 
  long-term contractual relationship with the enterprise  → OK to let  
  enterprise (not user) `see’ SP rules. We exploit this to eliminate 
  BB’s expensive handshake and hence BB’s perf. limitations
  → Because performance is no longer a limitation, we can expand 
  BB’s generality w.r.t. IDS processing. Specifically, we can expand 
  ability to `natively’ handle reg-ex w/o resorting to decryption. The 
  XXX approach we propose reduces the use of probable decrypt 
  by X%
  → In addition, MBArk exploits the fact that the enterprise GW acts as 
  a physical intermediary on the data path between the user and the 
  SP to improve deployability. MBark (re)factors functionality to shift 
certain operations into the GW (away from the end-user or SP) 
  and this results in a design that: (i) requires no change at end 
  users (unlike BB) and (ii) no change to MB (fast path) algorithms 
  that have been honed over decades to meet network line rates. 
  RM as primitive operation
  we design a new crypto protocol for RM


  Tradeoffs
relative to BB: lose nothing, except certain user-to-SP context (for IDS) 
  relative to APLOMB: more heavyweight GW but we’ve carefully designed MBArk so that the GW:
  retains high performance: only X% slower than APLOMB
  adds complexity mostly on the GW’s control plane
retains generality (techniques per class not instance of MBs)

  Hence, contributions
  new RM technique:
  list RM’s nice properties
  note it’s a standalone contribution
  new system design and implementation 
  system is general: 
  more MBs: e.g., [list set]
  better coverage of IDS: e.g., [give numbers]
  system has good performance: e.g., [give numbers] 
  system is easier to deploy: e.g., … [give concrete examples]


