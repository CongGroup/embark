%!TEX root = mb.tex 

\section{Introduction}\label{sec:intro}

Middleboxes such as firewalls, NATs, and proxies, have grown to be a vital part of modern networks, but are 
also widely recognized as bringing significant problems including high cost, inflexibility, and complex management.  
These problems have led both research and industry to explore an alternate approach: moving middlebox functionality out of dedicated boxes and into 
software applications that run multiplexed on commodity server hardware~\cite{mb-manifesto,comb,aplomb,opennf,clickos,flowtags,etsi-nfv,domain20,opnfv}.
This approach -- termed Network Function Virtualization (NFV) in industry -- promises many advantages including the cost benefits of commodity infrastructure, 
the efficiencies of statistical multiplexing, and the flexibility of software solutions. 
In a short time, NFV has gained a significant momentum with over 270 industry participants~\cite{etsi-nfv} and a number of emerging product offerings~\cite{brocade,dell,juniper}.

Leveraging the above trend, several efforts are exploring a new model for middlebox deployment in which a third-party offers middlebox processing as a  
\emph{service}.
Such a service may be hosted in a public cloud~\cite{aplomb,zscaler,aryaka} or in private clouds embedded within an ISP 
infrastructure~\cite{domain20, telefonica}.  
This service model allows customers such as enterprises to ``outsource'' middleboxes from their networks entirely, and hence promises many of the known benefits of cloud computing  such as decreased costs and ease of management.%: decreased costs, ease of management, \etc{}.

However, outsourcing middleboxes brings a new challenge: the confidentiality of the traffic. 
Today, in order to process an organization's traffic, the cloud sees the traffic {\em unencrypted}.  This means that the cloud 
now has access to potentially sensitive packet payloads,  IP addresses, and ports. This is 
worrisome considering the number of documented data breaches by cloud employees or hackers~\cite{PrivacyRecords,databreach}.
Hence, an important question is: can we enable a third party to process traffic for an enterprise, {\em without seeing the enterprise's traffic}?

To address this, we designed and implemented \sys, the first system to allow an enterprise to outsource  a comprehensive set of enterprise middleboxes  to a cloud provider, while keeping its traffic confidential. 
Middleboxes in \sys operate directly over {\it encrypted} traffic without decrypting the traffic. 


In recent work, we designed a system called BlindBox to operate on encrypted traffic for a {\em specific} class of middleboxes: Deep Packet Inspection~\cite{blindbox} -- middleboxes that examine only the payload of packets. 
However, BlindBox is far from sufficient for this setting because
 (1) it has a restricted functionality that does not support all middleboxes required for outsourcing, and (2) it has prohibitive performance overheads.
 
 Regarding functionality, BlindBox enables equality-based operations on  encrypted payloads of packets, which supports certain DPI devices. However, this excludes middleboxes such as firewalls, proxies, load balancers, NAT,  and those DPI devices that also examine packet headers, because these need the ability to examine packet headers and/or perform range queries. 
 % I added this packet header thing due to NAT
 % even though some of these MB only do exact, the architecture that puts them all together is new
 Achieving such generality provided new challenges to \sys. 
As we discuss below, such middleboxes require a new encryption scheme and, moreover, a system design that supports all middleboxes {\it simultaneously}: for performance, adoption and extensibility, it does not suffice to craft {\it n} different designs to support {\it n} classes of middleboxes. 

 
Regarding performance, BlindBox has a prohibitive connection setup time of 97s per-connection. 
\sys's system setting is different from BlindBox's setting and this difference brings \sys new opportunities: it
not only allows us to remove the setup overhead, but also to achieve better deployability and higher security. 

   \sys supports a comprehensive set of middleboxes with practical performance. These cover all the applications that are typically outsourced as surveyed in~\cite{aplomb}. Table~\ref{tbl:mbreqs} lists these applications. 
   \sys achieves these contributions through a combination of systems and cryptographic innovations, as follows.
   
   From a cryptographic perspective, \sys provides a new and fast encryption scheme called RangeMatch  to enable the provider to perform prefix matching (\eg{}, if an IP address is in the subdomain 56.24.67.0/16) or port range detection (\eg{}, if a port is in the range 1000-2000). The design of RangeMatch is highly inspired from the network setting.
 Prior to RangeMatch, there was no mechanism that provided the functionality, performance, and  security needed in our setting. The closest practical encryption schemes are Order-Preserving Encryption (OPE)~\cite{boldyreva:ope,popa:mope,popa:cryptdb}. 
However, we show that these schemes are four orders of magnitude slower than {\it RangeMatch} making them unusable for our network setting. At the same time, RangeMatch provides stronger security guarantees than these schemes, which leak unnecessary information to the provider: for every encrypted value, {\em RangeMatch} reveals only the needed functionality of whether the value lies within a range or not, whereas OPE reveals the total ordering among all values in addition to whether the value lies in the range.

  From a systems design perspective, one of the key insights behind \sys is to keep packet formats unchanged: an encrypted IP packet is structured just as a normal IP packet, with each field (e.g., source address) containing an encrypted value of that field.
  This strategy ensures that encrypted packets never appear invalid, e.g., to existing network interfaces, forwarding algorithms, and error checking. 
  We choose which encryption scheme to use for each field based on the processing operations applied by typical middleboxes; this results in an encrypted packet that simultaneously supports {\it all} of the wide range of middleboxes we outsource.
  
  Moreover, combining standard packet formats with RangeMatch allows \sys to achieve good performance for middleboxes operating over header fields.
  RangeMatch ensures that comparisons (such as $\leq$) function normally, hence, a middlebox receiving an encrypted packet can simply perform the normal set of `fast-path' packet processing operations unmodified.
  Importantly, middleboxes can continue to take advantage of highly-efficient packet classification algorithms~\cite{packet_classif} as already implemented in software.
  Furthermore, even software-based NFV deployments make use of some hardware forwarding components, \eg{} NIC multiqueue flow hashing~\cite{nicdocument}, off-the-shelf switches~\cite{whitebox}, and error detection in NICs and switches~\cite{nicdocument, ciscov6}; \sys is compatible with all of these.
  
 
We implemented and evaluated \sys on EC2. \sys supports all applications typically deployed by outsourcing as surveyed in~\cite{aplomb}.
Further, \sys imposes  negligible throughput overheads at the service provider: for example, a single-core firewall operating over encrypted data achieves 9.8Gbps, equal to the same firewall over unencrypted data.
Our enterprise gateway can tunnel traffic at 1.5 Gbps on a single core;  a single server can transmit \sys encrypted data at up to the full 10Gbps line rate of its network interface.

\eat{Relative to BlindBox, \sys is more general. But even when comparing the two directly, \sys (1) has practical performance for deployment, where BlindBox does not: connection session establishment uses standard handshakes and does not have the 97s overhead that BlindBox does; (2) is more secure: \sys can support 79.8\%-88\% of IDS rules without resorting to decryption, where BlindBox can only support 40-67\%; (3) requires no changes to endhosts, where BlindBox requires a new protocol to be supported by each client.} 
