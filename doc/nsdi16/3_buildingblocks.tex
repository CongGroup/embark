%!TEX root = mb.tex



\section{Cryptographic Building Blocks}

In this section we present the building blocks \sys relies on.
AES is a well known encryption technique and we do not discuss it here. Instead, we briefly discuss keyword match (introduced by~\cite{blindbox}, to which we refer the reader for details) and more extensively discuss range match, a new cryptographic scheme we designed for this setting.
When describing these schemes we refer to the encryptor as the gateway whose secret key is $k$ and to the entity computing on the encrypted data as the service provider (SP).


\subsection{KeywordMatch}\label{s:kwmatch}


KeywordMatch allows detecting if an encrypted keyword matches an encrypted data item by equality.
For example, given an encryption of the keyword ``malicious'', and a list of encrypted strings  [$\Enc$(``alice''), $\Enc$(``malicious''), $\Enc$(``alice'')], SP can  detect that the keyword matches the second string. 
For  this, we use a searchable encryption scheme~\cite{song:search, blindbox}.
Using this scheme, the gateway can encrypt a value $v$ into $\enc(v)$ and a rule $r$ into $\encr$ and SP can detect if there is a match between $v$ an $r$. 
%In such a scheme, 
%to encrypt a string $x$, the gateway computes 
%$\enc(v) = (\salt, H(\salt, \aes_k(v))),$ where $\salt$ is a random value, $H$ is a hash function and $\aes$ is the AES algorithm.
%To encrypt a keyword $r$ to search for, the gateway computes $\encr = \aes_k(r)$. 
%Given an encrypted string $\enc(v) = (\salt, A)$ and an encrypted keyword $\encr$, SP knows that there is a match if  $H(\salt, \encr) = A$.  
 The security of searchable encryption is well studied~\cite{song:search, blindbox}: at a high level,  given a list of encrypted strings, and an encrypted keyword, SP does not learn anything about the encrypted strings, other than which strings match the keyword. % In \sys, we will not be concerned with hiding the keyword, although $\encr$ does hide it. 
% The encryption of the strings is {\em randomized}, so it does not leak whether two encrypted strings are equal to each other, unless, of course, they both match the encrypted keyword. 
%As an optimization, in some applications, we can use a deterministic encryption scheme instead of the scheme above {\em without sacrificing security}: this is the case when we are guaranteed that the strings encrypted are distinct. %In this case, the salt is not needed, and we can simply encrypt a value $v$ with $\Enc(v) = \aes_k(v)$. This enables fast equality matching and building indexes for fast search (which will be useful for the HTTP proxy middlebox in \S\ref{s:proxy} ).

%!TEX root = mb.tex


%\section{Building block: Range match}
%An important operation over an encrypted packet is to determine if an encrypted field in the packet falls in an encrypted range.
%We will use the firewall as an example. 
%Consider the following firewall rule:
%
%Constructing an encryption scheme that allows checking if an encrypted value is in an encrypted range, has been a challenge in the applied cryptography community. The reason is that ..
%
%\begin{itemize}
%\item preserve the order between Encryptd values
%\item candidate: OPE
%\item candidate: mOPE
%\item So none of the existing schemes are satisfactory. A new scheme \RM.
%\end{itemize}
%
%\RM applies to cases when we know an upper bound on the values encrypted with OPE and this is a small number of values (say, less than 10,000).
%
%The small number of values permits us to improve in two ways over mOPE [1]
%No more interaction. We store the tree in mOPE on the gateway (client) side, which means that the gateway can compute a new encryption by itself without help from service provider. The storage at the gateway will remain small.
%Rare updates to ciphertexts. We can space out the ciphertexts of the values encrypted sufficiently. 
%
%This also enjoys a stronger security than OPE! It leaks less than order.
%The reason is that, the server does not learn the order between the values in packets, and only whether they map between two values in the rules. 
%
%this one is new
%
%discuss 
%
%would be good to explain the challenge from the 
%
%\todo{a more interesting name to the scheme}
%
%prefix gets mapped into interval, at most a certain number
%
%talk about building certain data structures that all works the same
%
%firewall need not change 

\subsection{Range match } \label{sec:range}




Many middleboxes perform detection over a {\it range} of port numbers or IP addresses. For example, a network administrator might wish to block access to all servers hosted by MIT, in which case the administrator would block access to the network prefix 18.0.0.0/8, \ie{}, 18.0.0.0-18.255.255.255. RangeMatch enables a middlebox to tell whether or not an IP address $v$ lines in between such a range $[s_1, e_1]$, where $s_1$ = 18.0.0.0 and $e_1$ = 18.255.255.255; however, the middlebox does not learn the values of $v$, $s_1$, or $e_1$.
Unlike KeywordMatch (or AES), RangeMatch is a new scheme we designed specifically for our setting.

One might ask whether RangeMatch is strictly necessary. To detect whether or not a port number lies in the range 80-83 is the same as asking whether the port number is equal to anything in the set \{80,81,82,83\} -- hence, one could use KeywordMatch to implement all range detection operations by converting each range to a set of exact-match values.
However, in practice, the overhead of doing this is prohibitive.
For our own department network, doing so would convert our IPv6 and IPv4 firewall ruleset of only 97 range-based rules to $2^{238}$ exact-match only rules; looking only at IPv4 rules would still lead to 38M exact-match rules.
Hence, to perform typical middlebox behaviors {\it in practice}, we require a RangeMatch scheme which is more efficient.

\subsubsection{Requirements}
The functionality of our RangeMatch scheme is to encrypt a set of ranges $[s_1, e_1]$, $\dots$, $[s_n, e_n]$ into  $[\Enc(s_1)$, $\Enc(e_1)]$, $\dots$, $[\Enc(s_n)$,  $\Enc(e_n)]$, and a value $v$ into $\Enc(v)$, such that anyone with access to these encryptions can determine in which range $v$ lies, while not learning the values of $s_1$, $e_1$, $\dots$, $s_n$, $e_n$, and $v$. 
For concreteness, we explain our scheme by considering $v$, $e_i$ and $s_i$ as IP addresses (although it can be used for encrypting ports too).

Our goal in designing RangeMatch was for it to be both efficient/fast {\em and} provides strong security.

In terms of performance, both encryption (performed at the gateway) and detection (performed at the middlebox) should be practical for typical middlebox line rates.
Our RangeMatch encrypts in $< 3\mu$s per value (we evaluate in \S\ref{sec:eval}).
Our design performs comparison between encrypted values and an encrypted rule (performed at the middlebox) using only on normal $\leq$/$\geq$ operators; hence it is compatible with existing classification algorithms such as tries, area-based quadtrees, FIS-trees, or hardware-based algorithms~\cite{packet_classif}.

For security, we require that the encryption scheme  not leak $v$, $e_1$, $s_1$, $\dots$, $e_n$, $s_n$ to SP.
Ideally, SP does not learn anything about $v$ other than what interval it matches to. In  particular, even if $v_1$ and $v_2$ match the same range, SP should not learn their order. SP is allowed to learn the order relation of the intervals (in fact, in many setups, SP provides the intervals). 
RangeMatch is the only scheme we know of to provide this security guarantee.

Although RangeMatch is similar to order-preserving encryption such as BCLO~\cite{boldyreva:ope} or mOPE~\cite{popa:mope}, neither meets our security requirement (because they leak the ordering between encrypted values, not just their ordering relative to rules), and neither provides the performance necessary for packet processing (which we investigate further in \S\ref{sec:eval}).

\eat{
\begin{CompactEnumerate}[leftmargin=*]

\item  {\em be fast}: the throughput of encryption should be not much lower than network throughput. In particular, the scheme should preserve the ability to use {\em existing fast packet matching algorithms}, such as  various kinds of tries, area-based quadtrees, FIS-trees, or hardware-based algorithms~\cite{packet_classif}.  All of these rely on the ability of SP  to compute ``>'' between $v$ and the endpoints of an interval,
hence the encryption scheme should preserve this property. 

\item {\em provide strong security}: The encryption scheme should not leak $v$, $e_1$, $s_1$, $\dots$, $e_n$, $s_n$ to SP.
Ideally, SP does not learn anything about $v$ other than what interval it matches to. In  particular, even if $v_1$ and $v_2$ match the same range, SP should not learn their order. SP is allowed to learn the order relation of the intervals (in fact, in many setups, SP provides the intervals). \label{req:sec}

\item {\em be deterministic}: To integrate with NAT and to enable middleboxes to piece together packets from the same flow, each value should get  consistently  the same encryption. Any changes in the encryption assigned should happen rarely.  \label{req:injective}

\item {\em be format-preserving}: The encryption should have the same format as the data. Concretely, an encrypted IP address should look like an IP address.  This property is important to avoid changing the packet header structure, which would be a hurdle to adoption. 
 \end{CompactEnumerate}
}


\subsubsection{Our RangeMatch scheme} 



%We explain the scheme based on encryption IP addresses for a firewall and source IP addresses in packets, although the scheme is used in encryption other fields too, as explained in Sec.~\ref{xx}.

To encrypt the endpoints of the intervals, we sort them, and choose as their encryptions values equally distributed in the domain space in a way that preserves the order of the endpoints. For example, the encryption of the intervals 127.0.0.0/8 and 172.16.0.0/16, is [51.0.0.0, 102.0.0.0] and [153.0.0.0, 204.0.0.0]. This preserves the order of the intervals but does not leak anything else.

For now, consider that the gateway  maintains a mapping of each interval endpoint  to its encryption, called the {\em interval map}.  The interval map also contains the points $- \infty$ and $+ \infty$, encrypted with 0.0.0.0 and 255.255.255.255. 


When the gateway needs to encrypt an IP address $v$, the gateway first determines what  is the interval  $v$ falls in (we discuss below what happens when more than one interval matches). It uses the interval map to determine the encryptions of the endpoints of this interval. Then, to encrypt $v$, it chooses a random value in this interval.
For example, if $v$ is 127.0.0.1, a possible encryption is 48.124.24.85. The encryption does not retain any information about $v$ other than the range it is in. In particular, for two values $v_1$ and $v_2$ that match the same interval, SP does not learn their order. Thus, this satisfies the security requirement above.

\justine{Todo: Figure. Three cases: one interval, multiple intervals, and no intervals. Highlight area that range should be mapped to.}
To generalize this insight, we need to specify how to encrypt $v$ when $v$ fits in multiple intervals or in no interval at all. Consider an example in which $v$ fits in no interval at all. Let $v$ be $127.0.0.0$ and the intervals be 18.0.0.0/8 and 172.16.0.0/16 with encryptions [51.0.0.0, 102.0.0.0] and [153.0.0.0, 204.0.0.0]. 
Recall that we want SP  to learn only whether $v$ matches an interval or not, but nothing else. 
Hence, $v$ should be mapped to a random value anywhere in the intervals [0, 51.0.0.0), (102.0.0.0, 153.0.0.0) and (204.0.0.0, 255.255.255.255]. 

We now consider the general case, where $v$ may map to one interval, no intervals, or many intervals. To achieve the desired security, the idea is to find the interval $I$ (which may consist of many sub-intervals, as in previous example) inside which $v$ should be mapped at random. The equation for $I$ is as follows. Consider the intervals {\em inside} which $v$ falls, and let $I_0, I_1, \dots, I_{n_I}$ be their encryptions. We always include the total interval $I_0 = [0.0.0.0, 255.255.255.255]$. Now consider the intervals {\em outside} of which $v$ falls and let $O_1, \dots, O_{n_O}$ be their encryptions. Then, to provide our desired security goal, $v$ should be chosen at random from the interval  
\begin{equation}
 I = I_0 \cap I_1 \cap ... \cap I_{n_I} - (O_1 \cup \dots \cup O_2). \label{eq:randominterval}
 \end{equation}

 


To support middleboxes like NATs -- which require that every time a value it is encrypted, it return the same value -- we need to generate the random encryptions using a deterministic function. For this, we use a pseudorandom function~\cite{GoldreichVol1}, $\prf$, seeded in $k$.   Let $|I|$ be the size of the interval $I$. 
Then, the encryption of $v$ is the $\mathsf{index}$-th element in $I$, where $\mathsf{index}$ is 
\[ \mathsf{index}(v) = \prf_k(v)\ \mod\ |I|. \] 
  Note that, in the systems setup with two gateways, the gateways generate the same encryption because they share $k$. 

When encrypting IP addresses, we do not want two different IP addresses to map to the same encryption (which would break the NAT). Fortunately, the probability that two IP addresses get assigned to the same encryption is negligibly low for IPv6. This probability is very low for IPv4, but not low enough that a collision will not happen in a large interval of time. The reason is that each interval of encryptions is large because we distributed the endpoint encryptions evenly and because there is a small number of such endpoints in a realistic setting (e.g., a firewall has less than 100,000 rules). Suppose we have $n$ distinct rules, $m$ flows and a $w$-bit space, the probability of getting collision is approximately $1-e^\frac{-m^2 (n+1)}{2^{w+1}}$. Therefore, if $w=128$ (which is the case when we use IPv6), the probability is negligible in a normal setting. 
\begin{figure}
  \includegraphics[width=3.45in]{fig/tree}
  \caption{\label{fig:tree} Range match tree. The values of nodes in the tree are the unencrypted IP addresses, and the blue values on the horizontal axis are their encryptions. }
\end{figure}



\textbf{Security guarantees.}
The scheme achieves our desired security goal: the only information leaked about a value $v$ encrypted is which ranges it matches. 
In particular, the scheme is not order preserving because it does not leak the order of two encrypted values that match the same range. It is easy to check that the scheme is secure: since the encryption of $v$ is random in $I$ (Eq.~\ref{eq:randominterval}), the scheme only leaks the fact that $v$ is in $I$. $I$ is chosen in such a way that the only information about $v$ it encodes is which intervals $v$ matches and which it does not match.

\textbf{Implementing RangeMatch}
The last set of issues to address are the implementation and management of a RangeMatch encrypter. 
To encrypt using RangeMatch, the gateway must store a tree -- as shown in Figure~\ref{fig:tree} -- mapping the unencrypted intervals to the encrypted address space.
We discuss how this tree is constructed, how rules can be added to or deleted from this structure with low overhead, and the performance implications of RangeMatch in \S\ref{sec:rangetree}.

\eat{
The last issue to address is addition and removal of intervals. For example, this can happen when a new rule is added to a firewall. 
Since we tried to spread out the encryptions of the intervals evenly in the IP space, there is no room for the new interval. We can 
readjust all the encryptions of these intervals to make space for the new interval. However, this would require the firewall hardware to reconfigure fully which is slow. Ideally, we would only reconfigure the firewall hardware incrementally, for the new interval. For this, we build on the idea from mOPE~\cite{popa:mope} and store the intervals at the gateway in a balanced scapegoat tree as in Fig.~\ref{fig:tree}. This tree is a search tree that has the property that when inserting or deleting a node, the number of other nodes that change encryption is small, namely, $O(\log n)$ amortized worst case where $n$ is the number of nodes. Each node in the tree is now encrypted similarly to before: the root gets the middle of the IP range,  the left node gets the middle of the IP space to the left of the middle, and so on, as in Fig.~\ref{fig:tree}.  Note that, since the tree is balanced, it maintains our desire of having the encryptions of endpoints roughly uniformly distributed.
We discuss the implementation of this tree in \S\ref{sec:gateway}.
}


%We now explain concretely the API and the implementation of the scheme. 

