\section{Introduction}\label{sec:intro}

Middleboxes such as firewalls, NATs, and proxies, have grown to be a vital part of modern networks but are 
also widely recognized as bringing  with them significant problems including high cost, inflexibility, and complex management.  
These problems have led both research and industry to explore an alternate approach: moving middlebox functionality out of dedicated boxes and into 
software applications that run multiplexed on commodity server hardware~\cite{mb-manifesto,comb,aplomb,opennf,clickos,flowtags,nfv,domain20,opnfv}.
This approach -- termed Network Function Virtualization (NFV) in industry -- promises many advantages including: the cost benefits of commodity infrastructure, 
the efficiencies of statistical multiplexing, and the flexibility of software solutions. 
In a short time, NFV has significant momentum with over 270 industry participants~\cite{etsi-nfv} and a number of emerging product offerings~\cite{something}.

Leveraging the above trend, several efforts are exploring a new model for middlebox deployment in which a third-party offers middlebox processing as a  
\emph{service}.
Such a middlebox service may be hosted in a public cloud~\cite{aplomb,zscalar,aryaka} or in private clouds embedded within ISP 
infrastructure~\cite{domain20, telefonica, find-more}.  
This service model allows customers such as enterprises to ``outsource'' middleboxes from their networks entirely, and hence promises many of the same 
benefits that outsourcing compute and storage have attained through cloud computing.%: decreased costs, ease of management, \etc{}.

However, outsourcing middleboxes to a third party provider brings a new challenge: the confidentiality of the traffic. 
Today, in order to process an organization's traffic, the cloud receives the traffic {\em unencrypted}.  This means that the cloud 
now has access to potentially sensitive packet payloads,  IP addresses, and ports. This is 
worrisome considering the number of documented data breaches by cloud employees or hackers~\cite{PrivacyRecords}.
Hence, an important question is: can we enable a third party to perform traffic processing for an enterprise, {\em without seeing the enterprise's traffic}?
To address this, we design and implement \sys, the first system to allow an enterprise to outsource {\it all} traffic processing to a cloud provider while keeping traffic private. 
\sys achieves this goal by allowing the cloud provider to perform packet processing operations directly over {\it encrypted} data without ever observing the plaintext.


In recent work, we addressed a similar challenge for one class of middlebox: Deep Packet Inspection~\cite{blindbox}.  
We developed BlindBox, a system that uses new encryption schemes to enable IDS processing directly on an \emph{encrypted} bytestream. 
The context for which we designed BlindBox differs from the work in this paper in two fundamental ways.

First, BlindBox addresses the scenario in which an arbitrary client connects to a network with an unknown middlebox, whereas \sys targets the enterprise setting where an entire companies traffic is outsourced to a known service provider.
\sys's context is an important one as it represents the majority of middlebox usage today~\cite{need-citation}. 
Nevertheless, this difference in setting allows \sys to improve upon BlindBox in performance, privacy guarantees, and in deployability. 
Importantly, this difference makes \sys practical for usage today, where BlindBox isn't: BlindBox imposes a 97 second setup time on {\it every} new connection, where \sys does not.

The second difference is in generality: where BlindBox only addresses DPI devices, \sys is general, supporting the full range of middleboxes used in enterprises today, from firewalls, to proxies, to DPI devices like BlindBox supports.
Achieving generality provided new challenges to \sys, not seen in BlindBox. 
To address new use cases, we had to design a new cryptographic algorithm in addition to the scheme BlindBox already provided. 
Further, \sys also required careful systems and protocol design to make sure that \sys's encryption would support all middleboxes {\it simultaneously}: it would not have been sufficient to craft {\it n} different designs to support {\it n} classes of middlebox.


    \sys achieves both generality with practical performance through a combination of cryptographic and systems innovations.
    \sys relies on two forms of computational encryption: {\it keyword match}, a form of searchable encryption developed by BlindBox  and {\it RangeMatch}.
    RangeMatch is a new encryption scheme we designed for \sys in order to enable the provider to perform prefix matching (\eg{}, if an IP address is in the subdomain 56.24.67.0/16) or port range detection (\eg., if a port is in the range 1000-2000), which are common actions performed by middleboxes such as firewalls. 

    We designed {\it RangeMatch} because no existing system provided the computational expressiveness, performance properties, and the privacy properties we desired for \sys.
    Of existing cryptographic schemes, OPE~\cite{cryptdb} is the closest relative of {\it RangeMatch}; however, it is four orders of magnitude slower than {\it RangeMatch} and leaks unnecessary information to the cloud provider.
    Where {\em RangeMatch} only reveals whether or not a value lies within a pre-determined range, OPE reveals the total ordering among all encrypted values, whether they lie within a range or not.

  From a systems design perspective, one of the key insights behind \sys is to keep packet formats unchanged: an encrypted IP packet is structured just as a normal IP packet, with each field (e.g., source address) containing an encrypted value of that field.
  We choose which encryption scheme to use for each field based on the processing operations applied by typical middleboxes.
  Structuring packets in this way, combined with {\it KeywordMatch} and {\it RangeMatch}, allows most middleboxes to remain completely unmodified on the dataplane, importantly, taking advantage of highly-efficient packet classification algorithms~\cite{somethingclassification--chang?} as already implemented.
  Consequently, forwarding rates at the service provider are unchanged for these middleboxes (and even those middleboxes we do modify have only modest overheads).
  This also means that all packets appear to middleboxes as formatted in a standard way -- the encryption of a given field for one category of middlebox never interferes with the behavior of another middlebox which does not operate over that field and does not make the packet appear invalid. 

We implemented and evaluated \sys. We support all applications typically deployed by outsourcing as surveyed in~\cite{aplomb} -- any appliance which it outsourced today can also be outsourced using \sys.
Further, \sys imposes very negligible throughput overheads at the service provider: for example, a single-core firewall operating over encrypted data achieves 9.8Gbps, equal to the same firewall over unencrypted data.
Our enterprise gateway can tunnel traffic at 1.5 Gbps on a single core;  our 8 core server can transmit \sys encrypted data at up to the full 10Gbps line rate of its network interface.

Relative to BlindBox, \sys is more general. But even when comparing the two directly, \sys (1) has practical performance for deployment, where BlindBox does not: connection session establishment uses standard handshakes and does not have the 97s overhead that BlindBox does; (2) is more secure: \sys can support 79.8\%-88\% of IDS rules without resorting to decryption, where BlindBox can only support 40-67\%; (3) requires no changes to endhosts, where BlindBox requires a new protocol to be supported by each client. 
We revisit BlindBox and disuss how these benefits come about throughout this paper and summarize in \S\ref{sec:relatedwork}.

The rest of this paper is organized as follows...
