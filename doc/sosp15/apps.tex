%!TEX root = mb.tex

\section{Firewall}\label{sec:firewall}

Firewalls from different vendors may vary significantly in terms of rule syntaxes and organizations. However,
in general both hardware and software firewalls have a few interfaces. Both ingress and egress of an interface 
can be associated with an access control list (ACL). Each ACL has a number of rules, possibly in the form 
<action, protocol, src ip, src port, dst ip, dst port>. Without loss of generality, we take \texttt{pf}, the 
default firewall under BSD, as an example to illustrate how \sys works with firewalls. Figure \ref{fig:fwrule1} 
shows an example of \texttt{pf} rules. 

\begin{figure}[h]\label{fig:fwrule1}
\begin{lstlisting}[frame=single]
ext_if = "kue0"

block out log quick on $ext_if \
from 157.161.48.0/24 to any

block in quick on $ext_if \
from any to 255.255.255.255

pass out on $ext_if proto tcp \
from any to any port 80
\end{lstlisting}
\caption{\texttt{pf} configuration example}
\end{figure}

To work with \sys, we need to rewrite the rules above. Using the Range Match scheme, we encode all IP addresses, 
prefixes, and port numbers. Note that \texttt{any} is the alias of \texttt{0.0.0.0/0}. Recall that in Section \ref{sec:range}
we want to keep the chance of collision as low as possible, therefore we map the IPv4 addresses into the IPv6 space.
For example, to encode the IP prefix \texttt{157.161.48.0/24}, the gateway first extends it to \texttt{::ffff:157.161.48.0/120}.
Now the prefix represents the range from \texttt{::ffff:157.161.48.0} to \texttt{::ffff:157.161.48.255}. The gateway then insert
both endpoints to the tree, and gets their encoded values. 

Suppose those two values are \texttt{fd6a:1faf:577b:bee4:0:0:0:0} and
\texttt{fd6a:1faf:577b:bee4:0:1:0:1}. Note that we can't represent this range using a single prefix, instead it has to be 
represented by at least two prefixes (\texttt{fd6a:1faf:577b:bee4:0:0:0:0/96} and \texttt{fd6a:1faf:577b:bee4:0:1:0:0/127}).
In this case, we have to split the original rule into 2 new rules:

\begin{lstlisting}[frame=single]
block out log quick on $ext_if \
from fd6a:1faf:577b:bee4:0:0:0:0/96 to any

block out log quick on $ext_if \
from fd6a:1faf:577b:bee4:0:1:0:0/127 to any
\end{lstlisting}

\todo{evaluate how many additional new rules are needed.}

After the gateway has rewritten the rules, it sends the rules back to the firewall on the cloud so that the
firewall can filter packets using the new rules.

Every time the gateway forwards a packet from the user, it replace the IPv4 header with the IPv6 header 
using the Stateless IP/ICMP Translation (SIIT) \cite{rfc2765}. The purpose of this operation is to leave 
enough address space for avoiding address collision. The gateway then encrypts the source IP, source port, 
destination IP, destination port fields using the Range Match scheme in Section \ref{sec:range}. In addition,
the gateway appends an additional header containing the ciphertexts.

\todo{a diagram that shows the packet format before and after encryption}

To remark, a key contribution of our scheme is that firewall implementation don't need to be changed. The central 
piece of packet filtering is packet classification, which is essentially a problem of finding a hyperrectangle 
that contains a point from a given set of hyperrectangles. Our Range Match scheme keeps the order among the coordinates 
of hyperrectangles, and the order between the coordinates of hyperrectangles and points. In other words, 
their topological relations are preserved. Therefore firewalls still work with \sys without modification. This 
property is important, since many high-speed firewalls are implemented in hardware, which is difficult and 
expensive to redesign.

- describe how we rewrite rules 

- give an example of a rewritten rule

- describe how the packet looks like now, what fields are encrypted (ports, source IP, destination IP)

- discuss how the firewall works: importantly, existing data structures still work, hardware is unchanged

- anything else you think the section should contain

\section{NAT}\label{sec:nat}

give example of what a rule becomes

\section{Load balancer}\label{sec:loadb}

L4 load balancer
L7 load balancer

\section{Proxy/cache}\label{sec:proxy}

L7 Proxy / Cache

\section{IDS and Exfiltration}\label{sec:IDS}

IDS/ exfiltration through watermarking/ parental filtering

even though for this we use an existing scheme, we incorporate it into the whole
system and make it work with the other components

-- gateway checks that the rules from the service provider have a minimal length 


\section{Discussion of other middleboxes}\label{sec:vpn} \label{sec:other} \label{sec:not_supp}

VPN naturally remains the same. 