%!TEX root = mb.tex
\section{Applications}

\todo{tie all apps in with the two basic operations}

In this section we present how to apply the building blocks to real-world applications.

\section{Application: Firewall}\label{sec:firewall}
We use the term ``firewall'' for stateful and stateless packet filters that filter the traffic based on network layers and transport layers. Stateless firewalls commonly examine the combination of the packet's source and destination address, its protocol, and for TCP/UDP traffic, its source and destination port number. Stateful firewalls additionally keep track of protocol states for each flow and retain packets until they has enough information to make decision. 

In sum, the \RM scheme supports both types of firewalls. The gateway encrypts the rules from the cloud beforehand, and send them back to the cloud. The cloud installs those rules to the firewall. After the initialization is finished, the gateway can encrypt the traffic and forward to the cloud, and the firewall in the cloud can process the traffic normally without modification. When the traffic gets back to the gateway, it gets decrypted.

\subsection{Firewall rules}
Firewalls from different vendors may vary significantly in terms of rule syntaxes and organizations. However,
in general both hardware and software firewalls have a few interfaces. Both ingress and egress of an interface 
can be associated with an access control list (ACL). Each ACL has a number of rules, possibly in the form 
<action, protocol, src ip, src port, dst ip, dst port>. Without loss of generality, we take \texttt{pf}, the 
default firewall under BSD, as an example to illustrate how \sys works with firewalls. Figure \ref{fig:fwrule1} 
shows an example of \texttt{pf} rules. 

\begin{figure}[h]\label{fig:fwrule1}
\begin{lstlisting}[frame=single]
ext_if = "kue0"

block out log quick on $ext_if \
from 157.161.48.0/24 to any

block in quick on $ext_if \
from any to 255.255.255.255

pass out on $ext_if proto tcp \
from any to any port 80
\end{lstlisting}
\caption{\texttt{pf} configuration example}
\end{figure}

To work with \sys, we need to rewrite the rules above. Using the \RM scheme, we encode all IP addresses, prefixes, and port numbers inside the rules. Note that \texttt{any} is the alias of \texttt{0.0.0.0/0}. Recall that in Section \ref{sec:range} we want to keep the chance of collision as low as possible, therefore we map the IPv4 addresses into the IPv6 space before encoding those values. For example, to encode the IP prefix \texttt{157.161.48.0/24}, the gateway first extends it to \texttt{::ffff:157.161.48.0/120}.
Now the prefix represents the range from \texttt{::ffff:157.161.48.0} to \texttt{::ffff:157.161.48.255}. The gateway then insert
both endpoints to the tree, and gets their encoded values. 

Suppose those two values are \texttt{fd6a:1faf:577b:bee4:0:0:0:0} and
\texttt{fd6a:1faf:577b:bee4:0:1:0:1}. Note that we can't represent this range using a single prefix, instead it has to be 
represented by at least two prefixes (\texttt{fd6a:1faf:577b:bee4:0:0:0:0/96} and \texttt{fd6a:1faf:577b:bee4:0:1:0:0/127}).
In this case, we have to split the original rule into 2 new rules:

\begin{lstlisting}[frame=single]
block out log quick on $ext_if \
from fd6a:1faf:577b:bee4:0:0:0:0/96 to any

block out log quick on $ext_if \
from fd6a:1faf:577b:bee4:0:1:0:0/127 to any
\end{lstlisting}

\todo{evaluate how many additional new rules are needed.}

After the gateway has rewritten the rules, it sends the rules back to the firewall on the cloud so that the
firewall can filter packets using the new rules.

\subsection{Encryption and decryption}
Every time the gateway forwards a packet from the user, it replace the IPv4 header with the IPv6 header 
using the Stateless IP/ICMP Translation (SIIT) algorithm \cite{rfc2765}. The purpose of this operation is to leave 
enough address space for avoiding address collision. The gateway then encrypts the source IP, source port, 
destination IP, destination port fields using the Range Match scheme in Section \ref{sec:range}.

\todo{a diagram that shows the packet format before and after encryption}

\subsection{Remark}
A key feature of our scheme is that we don't require to change the firewall implementations. As we discussed in Section \ref{sec:range}, the central 
piece of firewalls is packet classification, which is essentially a problem of finding a hyper-rectangle that contains a point from a given set of hyper-rectangles. Our \RM scheme keeps the order among the coordinates of hyper-rectangles, and the order between the coordinates of hyper-rectangles and points. In other words, their topological relations are preserved. Therefore existing firewalls can still work with \sys without modification. This property is important, since many high-speed firewalls are implemented in hardware, which is difficult and expensive to redesign.

\section{Application: NAT}\label{sec:nat}

give example of what a rule becomes



\section{Application: IDS and Exfiltration}\label{sec:IDS}

\todo{tie into the story of keyword match}

A lot of middleboxes run intrusion detection systems (IDS) and exfiltration services on the payload of network packets.
For example, middleboxes run network intrusion detection or IDS (e.g., using  Snort~\cite{Snort} 
or Bro~\cite{Bro}) to detect if packets from a compromised sender contain an attack and they 
perform data loss/ exfiltration prevention such as searching for watermarks in 
documents~\cite{CMU_exfiltration_report} to detect if an insider outsources confidential information.

In most of these systems, the middlebox has a set of {\em rules}, which are descriptions of attacks, and which 
the middlebox tries to match against the traffic. The rules can contain {\em rule strings} which should be matched
exactly against the traffic, offset information describing offsets in the packet where to search the rule string, and
sometimes even regular expressions.

To protect traffic privacy from the cloud, the payload of packets in \sys must be encrypted. 
However, how can the middlebox now perform the detection on the encrypted traffic? 
We know that nontrivial computation on encrypted data is very slow. ~\cite{CITE}

A recent system, BlindBox~\cite{blindbox},  enables detection at the middlebox on the encrypted traffic
in a practical way. 
Unfortunately, BlindBox was designed for a different network setting than our setting 
and does not fit our setting  as follows.


BlindBox was designed for the common network setting when the endpoints are users who generate
traffic that is encrypted and then sent through a middlebox. This means that the encryption algorithm
has access to the contiguous byte sequence of the traffic. In contrast, in \sys, the encryption 
happens at the gateway 
which only receives packets from the user endpoints. 
These packets can correspond to multiple flows and can be out of order, 
making it challenging to encrypt each packet in such a way that the middlebox can  {\em detect 
a rule string that spans multiple packets}, that is, a rule string where a part is at the end of a packet and the other part is at the beginning of the next packet.  Reconstructing the flow by putting the packets from each flow together and in order  is a very expensive 
operation. Performing such reconstruction at the gateway is 
very expensive, and would obviate a big part of the benefits of the Aplomb model. 


\sys addresses this issue through a new technique called {\em cross-packet detection}. Using this technique,
\sys can encrypt the packets in a special way that enables the middlebox to detect a rule string that spans multiple
packets. The gateway does not have to do flow reconstruction and can perform the encryption 
independently on a per packet basis. Only the middlebox
needs to perform flow reconstruction, which is the norm in IDS. 


Another difference of BlindBox's setting is the threat model. In BlindBox, user
endpoints perform the encryption and since a potential attacker is also a user, 
the endpoints are not trusted. As a result, in BlindBox the middlebox cannot
give the rules to the endpoints either because this will enable an attacker to eschew the
rules and because these rules are proprietary. In order to enable detection of the rules
over the encrypted data, they need to be encrypted with the same key as the 
key of the connection between the two endpoints. This situation requires BlindBox to perform
a sophisticated computation at the start of each connection, which makes the connection
startup time be slow. 


In our model, the gateway is performing the encryption, which is trusted. Even if the
user endpoint is malicious, the middlebox can share the rules with the gateway, and the
endpoint has no way of obtaining those rules. In \sys, we exploited this property to create a
much faster setup phase than in BlindBox. \todo{give some comparative numbers}




We now describe the common parts our system has to BlindBox and the new techniques we introduce. 





\subsection{Basic protocol}\label{sec:BB}

We now describe the part of our protocol that is common with BlindBox. 
Since this part is from BlindBox, we present it briefly, and refer the reader for details  to BlindBox~\cite{blindbox}.

The middlebox (MB) has a set of rules. The rules contain rule strings or regular expressions. 
For example, consider the  simplified rule number 2003296 from the Snort Emerging Threats ruleset:
%
\begingroup     \fontsize{9pt}{10pt}\selectfont
\begin{tabbing}
\green{alert} tcp \$EXTERNAL\_NET \$HTTP\_PORTS  [...] ( \\ 
%
% \= \kill
%\> -> \$HOME\_NET 1025:5000 ( \\
%
te \= content \= \kill
\> \green{content}:\ ``Server|3a| nginx/0.'';  \\
\> \green{content}:\ ``Content-Type|3a| text/html''; \\
\> \green{content}:\ ``|3a|80|3b|255.255.255.255''; )
\end{tabbing}
\endgroup

Each content must be matched exactly against the traffic. ``|'' denotes binary data. 

\paragraph{Connection setup.} 
Two endpoints setup a connection with SSL key $k$. 
In order for MB to match a rule string (denoted content above) $r$ on the encrypted traffic, it needs to have them encrypted 
under the same key $k$ with AES: $\aes_k(r)$. 
As explained, in BlindBox's setup, MB cannot give $r$ to the users and the endpoints cannot give $k$ to MB 
(or else, they don't have any privacy). As a result, BlindBox employs a sophisticated algorithm for encrypting
these rules based on Yao garbled circuits~\cite{}, which is slow. We eliminate this slowdown in \S\ref{sec:IDS}.


\paragraph{Encrypt traffic at endpoint.}
The stream of data at each endpoint gets encrypted with SSL as before 
and separately with BlindBox. 
For the BlindBox encryption, the stream gets tokenized and then encrypted. 
The tokenization produces a set of keywords from the traffic stream, e.g.,
at every offset in the stream, a keyword of size $m=8$ is produced.
For simplicity of presentation, assume that each rule string is also $m$ bytes long, although
~\cite{blindbox} explains how to detect rules that are longer.
 Each token $t$ then gets encrypted using
a new and strong encryption scheme $\salt, H(\salt, \aes_k(t))$, where 
$\salt$ is a random value and $H$ is a hash. 

\paragraph{Match at middlebox.} 
Given an encrypted rule $\encr = \aes_k(r)$ and an encrypted token $t$ from the traffic
$\salt, c = H(\salt, \aes_k(t))$, the middlebox detects a match if 
\[H(\salt, \encr) \stackrel{?}{=} c.\] 
BlindBox has a fast detection algorithm that employs this matching on many rules and
packets and we use it too.


\paragraph{Security and functionality.}
BlindBox provides two types of security guarantees, SEC I and SEC II,  and we inherit these too. 
SEC I is stronger than SEC II, but SEC II can support more general functionality while 
still providing meaningful privacy guarantees.
Table~\ref{tab:BB_sec_func} in \S\ref{sec:eval} summarizes this tradeoff. 

SEC I: The first security guarantee is that the middlebox does not learn anything about the data
content except for which offsets match a malicious string, namely, a content in a rule. 
If a part of the packet does not match a rule, the middlebox does not learn the content 
of that part of the packet. The middlebox also learns the number of tokens, but not what
the tokens are. In some sense, learning that a malicious string matched the packet is 
a minimal amount of information that MB needs to learn to take certain actions efficiently.
 For this security guarantee, BlindBox can support fully exfiltration 
detection and about half of the rules of common IDS. 

SEC II: The second security guarantee is a bit weaker, but still meaningful. This allows BlindBox
to support regular expressions in a practical way, and thus support 100\% of IDS rules. 
The privacy model is called prabable cause privacy which says that the middlebox
learns an entire packet if the packet is suspicious (namely it contains a  malicious string from a rule). 
If a packet does not contain any such suspicious string, the middlebox does not learn the content of the packet as 
in SEC I. 



\subsection{\sys's IDS}\label{sec:IDS}

We now describe \sys's IDS. This consists of the protocol above together with two new techniques 
for enabling detection of rule strings that span multiple packets and speeding up the connection
setup time. 


\subsubsection{Cross-packet encrypted detection}

Since the gateway receives packets as opposed to byte streams as in BlindBox, the gateway
must be able to encrypt the packets in such a way that enables detecting a rule string that
spans multiple packets. The gateway must not reconstruct the flow by piecing together packets in order
and only work with one packet at a time.

For example, if the stream has two packets with the payload
``this is inse'' and ``cure string''. MB might have a rule with the rule string ``insecure'': 
``drop packet \{content: insecure\}''.  

If the gateway produces tokens per packet, there will be a ``inse'' token in the first packet
and ``cure'' token in the second packet, none of which match "insecure". 

 Let $m$ be the token length.
 
 \paragraph{Strawman} The first idea is to have the gateway tokenize the end and beginning of each 
 packet in shorter tokens than $m$ for all possible ways that a string of $m$ bytes could have been split in.
 Also, MB will also have encrypted rules that correspond to all possible breakdowns
 of a rule of size $m$. 
 For example, for the packet ``this is inse'', the gateway produces additional tokens for the end of the packet
 "is inse", "s inse", $\dots$, "se", "e" and for the packet ``cure string'' the beginning tokens ``cure st", ``cure s", $\dots$,  ``cu", ``c".
 Besides having an encryption of ``insecure", MB also gets encryptions of the pairs (``i", ``nsecure"), (``in", ``secure"), 
 $\dots$ (``insecur", ``e").
 MB will detect the attack because it has the rule (``inse'', ``cure"), it reconstructs the flow, and will see that a packet contains at the end ``inse'' and the next packet
 contains ``cure" which match the rule. 
 
 While this strawman achieves correct functionality, it has a set of serious security issues:
 \begin{enumerate}
 \item some rules are now less than $m$ bytes long (e.g., some are one letter long) and the attacker learns further information by seeing where they match. A short rule string can match a packet even if the corresponding full $m$-bytes rule string it is part of does not match in the packet. For example, if a packet ends in ``in'' and the next packet starts with ``telligent", MB will know that the first packet ended in ``in" because it has a rule string for ``in". MB should not learn this information unless the byte stream contains the string ``insecure". 

 \item MB can combine {\em non-consecutive} packets together and see if the form ``insecure'' on their border.
 \end{enumerate}
 
 Ideally, we would like to preserve the security guarantee that MB only learns if an $m$-bytes rule string matches at an offset, but nothing else. To achieve this and address the challenges above, we provide a new way to encrypt the strings on borders of packets. 
 
 The goal is for the gateway to somehow tie together the last $i$ bytes from the end of a packet to the first $m-i$ bytes from the beginning of the next packet. This is difficult because the gateway cannot order packets. We have a few insights to make this goal possible. First, recall that each packet has a serial number at a fixed offset in the packet, which is easy to read for the gateway. The second packet has the serial number of the first packet plus one. Hence, we will tie together the two packets through the serial number. Second, the encrypted rule should not contain the two parts of the rule encrypted separately (or separably) so that an attacker cannot detect each string individually. To achieve this, we encrypt the rule into $g^{\aes_k(\mathsf{left}), \aes_k(\mathsf{right})}$, for a random $g$. Finally, we use a set of other cryptographic insights which we do not explain here because the focus is on the system. A reader uninterested in the crypto can directly skip to the protocol description \S\ref{sec:cross_prot}. 
 
 \subsection{Crypto scheme}
 
 \todo{sequence number}
 \todo{say somewhere that we do not have full formalism, see some paper}
 All operations are in a group $\bbG$ of prime order $p$, even if we don't say so.  
 
 The scheme is as follows:
 \begin{itemize}
 \item $\keygen$: generate a random key $k$ for AES and a random value $g$. The final key is $(k, g)$. 
 \item $\kwenc((k,g),  r_1, r_2)$: receives as input a split of a rule into a part $r_1$, at the end of a packet, and $r_2$ at the beginning of the next packet. Outputs the encryption of the rule \[ \encr = g^{\aes_k(r_1, \text{``left"}) \cdot \aes_k(r_2, \text{``right"})}.\] 
 \item $\encleft((k, g), s, \sno)$: receives as input the beginning $s$ of a packet with sequence number $\sno$. It chooses a random $\salt$ and outputs \[ \salt, \ct_L = H(\salt, g^{\aes_k(s) \cdot \aes_k(\sno)^{-1}}).\]
 \item $\encright((k, g), s, \sno):$ receives as input the end of a packet $s$ with sequence number $\sno$ and outputs $\ct_R = \aes_k(s) \cdot \aes_k(\sno+1)$. 
 \item $\match(\encr, (\salt, \ct_L), \ct_R)$: check whether \[H(\salt, \encr^{\ct_R}) \stackrel{?}{=} \ct_L.\]
 \end{itemize}
 
 It is easy to check that if a rule consists of strings $(r_1, r_2)$ and we have $r_1$ at the end (right) of a packet and $r_2$ at the beginning (left) of the next packet, a match will be flagged. 
 The reason is that the right encryption is done with a serial number of $\sno +1$ to match the serial number of the left encryption of the subsequent packet. 
 
The security of the scheme relies on the standard Diffie-Hellman assumption. We prove formally that this scheme achieves the same security as the regular matching scheme (\S\ref{simple_match}) in XXX, as if the rule was not split over two packets. The scheme is implemented over elliptic curves allowing for faster exponentiation and shorter ciphertext sizes. 

 
\subsection{Protocol using the scheme} \label{sec:cross_prot}
 
 -- in comparison to BlindBox, new encryption scheme
 
 
 -- call it N times
 
 
\paragraph{Optimization.} The protocol above, requires $N$ exponentiations per rule per packet boundary. 
Hence, there are $N R$ exponentiations per packet boundary which is a significant slowdown, considering that $R$ can be thousands. 
While $N$ exponentiations are a most overhead, $N R$ is a significant overhead, where $R$ is the number of rules. To reduce 



 \begin{framed}
\begin{algorithmic}[1]

\Procedure{MarginEncrypt}{$p$, a packet's payload}
	\State \Comment Encrypt the left margin: 
	\For{i = 1, m-1}
		\State \todo{finish}
	\EndFor
\EndProcedure
\end{algorithmic}
\end{framed}

 
 The middlebox reconstructs the flow and pieces together consecutive packets. For each packet, it performs detection 
as in the basic protocol above. Moreover, MB has to check rules across packet margins. For this, it invokes  the 
{\em MarginMatch} procedure below for every pair of consecutive packets. 

\begin{framed}
\begin{algorithmic}[1]
\Procedure{MarginMatch}{$p_L$ and $p_R$, the payloads of two consecutive packets}
	\State \todo{finish}
\EndProcedure
\end{algorithmic}
\end{framed}

no need for salt any more, the salt is in fact hte serial no

 
 
  to account 
 for all the possible ways that an $m$ byte word could have been split at the boundary between the two packets.
 Each of these new tokens are encrypted using our protocol so far.
 MB 
 
 and ``cure string''.
 
 security

 
 \subsubsection{Connection setup}
 
 It turns out that \sys's setting allows us to greatly simplify and speed up the connection setup time. In \sys, a trusted gateway performs the encryption as opposed to untrusted user machines. Since the gateway is trusted, it can see the rules $r$, unlike in BlindBox. Hence, 
to encrypt a rule string $r$ at the middlebox, MB can give $r$ to the gateway, which has $k$, and can compute $\aes_k(r)$. This is much faster than using Yao garbled circuits as in BlindBox, using which the gateway would have to produce a garbled circuit for AES, send it to the middlebox, MB runs a special protocol to obtain some encoding of the rules $r$, and finally MB evaluates the garbled circuit on the encoding of $r$ and obtains $\aes_k(r)$. oblivious transfer~\cite{} to obtain the rule, resulting in XXX speedup and XXX less bandwidth usage. 





\section{Application: Proxy/cache}\label{sec:proxy}


- emphasize networking aspect, more than crypto
- delegate html pipelining to a longer discussion, and treatment of corner cases, figures

\begin{itemize}
\item architecture overview
\item HTTP request -- absolute url and Host + relative url
\item tokenization special case \\
    \begin{itemize}
    \item in order to fetch the url from the header
    \item tokenize HTTP methods: GET, POST, ...
    \item tokenize attribute names: "Host:"
    \item tokenize versions: "HTTP/1.1", ...
    \end{itemize}
\item walk through the whole process
\item do not support pipelining yet
\end{itemize}

- also relies on the cross boundaries matching but we do not present here 

we are focusing on the transparent proxy
- discuss the kind of proxies we are focusing on

proxies have two benefits: latency savings which aplomb does not give you 
and bandwidth savings, which aplomb gives you
 
L7 Proxy / Cache

explain how the proxy finds if there is  a match --looks at header bytes
extra field attached to the packet -- gateway understands http and points out file id, 

discuss both cases: cache miss and hit

- two indep tcp connections 

how to populate the cache: content providers pushing the content to CDNs, or the gateway understands
http and tags what is content and what is ID

the web proxy needs to send http response 

proxy only looks at the top N bits corresponding to a large header and matches the file id with the entire path
and matches GET and a few others -- avoid parsing http this way


discuss how the proxy can reconstruct the response?
can proxy reconstruct the ip header and the http header  and the tcp header without being able to encrypt
--- check the http header details


\section{Other middleboxes}\label{sec:vpn} \label{sec:other_apps} \label{sec:not_supp}\label{sec:loadb}
\qu{should we call them applications or middleboxes in the sections above?}


L4 load balancer
L7 load balancer

-- VPN naturally remains the same. 

-- make the tables of applications consistent with each other and make sure we support all 

-- what is difference between L4 and L7 load balancer -- do we need both?

-- IP forwarding app, what is that?

-- how about the application firewall?

-- discuss wan optimizers and compression, that aplomb does not really support them in the basic model (requiring this special model that is not quite aplomb) -- also the conflicts it would show up for us
