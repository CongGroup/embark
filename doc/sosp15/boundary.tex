%!TEX root = mb.tex

\paragraph{Some attacks to be aware of.}
Before the boundary scheme can be secure, it has to withstand the following attacks:

\begin{enumerate}
\item given rule (a1, b1), (a1, b2), an attacker may try to compute rule (b1, b2)
\item given rule (a2, b2), (a1, b1), an attacker creates rule (a1, b2)
\item attacker detects if two boundaries are teh same when neither match a rule string fully
\item \label{step:rule} given rules (a2, b1), (a1, b1), (a2, b2), create a rule for (a1, b2) or detect if this shows up in package
\end{enumerate}

\paragraph{In search for a scheme.}
Intuitively, a secure scheme prevents the rule from being separated in components of $a$ and of $b$. Hence, it cannot be a simple multiplication or xor. 

Here are some directions where I looked for ideas until I found this protocol: polynomial interpolation, garbled circuit inspired, encryption with certain combined key like in garbled circuits, exponentiation, recognize $s | f(s)$, ElGamal type. 

There is an obvious and simple solution with bilinear maps but that would be too slow.  

Another protocol I had relied on composite order groups (hopefully on elliptic curves so exponentiation is cheaper): packets: $R$, $R^{a_1}$, $R^{b_1}$, the encrypted rule is $(a_1b_1)^{-1}) \in \phi_n$. Note that an attacker cannot compute the inverse of this encrypted rule because it does not know $\phi_n$ as in RSA. 


\paragraph{Proof sketch for our scheme.}
Ways to prove secure our boundary scheme. First create a hybrid that replaces the random oracle $\salt, H(\salt, a)$ with an entiry that gets the question ``do you have $a$''. The entity answers yes or no. 

The second hybrid tries to get rid of the random oracles completely for rules that do not match.
Basically, we need to show that one cannot compute $g^{a_1 r_1}$ without getting both of $g^{a_1 b_1}$ and $b_1^{-1} r_1$ for some $b_1$.
We show this in two cases:

Case 1:   one cannot compute $g^{a_1 r_1}$  when given only
$g^{a_1 b_1}$, $c_1^{-1} r_1$, $g^{a_1b_2}$, $b_2^{-1} r_2$, and so forth.

Case 2:   one cannot compute $g^{a_1 r_1}$  when given only
$g^{a_2 b_1}$, $b_1^{-1} r_1$, $g^{a_1b_2}$, $b_2^{-1} r_2$, and so forth.

Otherwise CDH is broken (note that if these were not exponents of $g$, the attacker could mount the attack~\ref{step:rule} above which should be prevented by raising to the power of $g$).

An useful fact to use is that each $r_i$ shows up in only two terms.

