%!TEX root = mb.tex


    Network processing appliances (''middleboxes'') such as firewalls, proxies, and WAN optimizers make up a substantial fraction of modern network infrastructure; studies show that as much as 1/3 of enterprise network hardware consists of such devices~\cite{aplomb}.
    However, recent trends suggest that this fraction may begin to decline as more and more networks begin to {\it outsource} their network processing, either to cloud providers~\cite{aplomb, aryaka, zscalar} or to service providers through Network Functions Virtualization (NFV)~\cite{nfv}.
    At the time of this writing, the NFV working group~\cite{nfvwg} has over 250 members ranging from large telecoms to hardware manufacturers, all of whom are investing in new technologies to enable outsourced traffic processing.
    For consumers, outsourcing traffic processing offers many of the same benefits that outsourcing compute and storage have attained through cloud computing: decreased costs, ease of management, the ability to scale and failover on demand, \etc{}.
    Nevertheless, outsourcing network processing brings new challenges, and among them, an issue which is critical to most enterprises: privacy.
    
    Under traditional middlebox deployments, traffic is processed and inspected by devices which are owned, hosted, and managed entirely within the enterprise itself.
    Encrypted traffic is often decrypted to enable deep packet inspection (DPI) such as intrusion detection and exfiltration detection.
    This configuration places critical trust in the hands of network administrators, who can potentially read or even modify {\it any} connection within the company.
    Outsourcing shifts this trust away from network administrators who are employed and monitored by the company whose data is being processed, and to a third party company with its own employees, hiring practices, and motivations.
    Hence, in this paper, we ask: is it possible to enable a third party to perform traffic processing for an enterprise, without transferring the ability to read and monitor the enterprise's traffic?
    
    
\section{Introduction}\label{sec:intro}
\begin{itemize}
\item why aplomb
\item why adding security
\item how to secure (which information need to protect)
\item how does mbark work: building blocks, applications secured (a nice table could go well here)
\item perhaps aplomb overview figure already in intro
\item contribution
\end{itemize}
