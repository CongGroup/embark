\documentclass[10pt,preprint]{sigplanconf}
\usepackage{times}
\usepackage{amsmath}
% SOME OF OUR DEFINITIONS
%\usepackage{savetrees-clean}
\usepackage{xspace}
\newcommand{\sys}{MBArk\xspace}
%\newcommand{\sys}{Embark\xspace}
%\newcommand{\sys}{EMBArk\xspace}
\newcommand{\RM}{RM\xspace}
% MBArk
% EMBArk

\usepackage{fancyvrb}
\usepackage{listings}

%more compact than computer modern
\usepackage{times}

\lstset{
  basicstyle=\footnotesize\ttfamily, 
  columns=fullflexible,
  keepspaces=true,
}

\usepackage{subfigure}
\usepackage{ifpdf}
\usepackage[usenames,dvipsnames]{color}
\usepackage[letterpaper, breaklinks, pdfborder={0 0 0}]{hyperref}
\hypersetup{
  backref=true
  bookmarksnumbered,
  colorlinks=true,
  pdfstartview={FitH},
  citecolor={blue},
  linkcolor={blue},
  urlcolor={blue},
  citecolor={blue},
  pdfpagemode={UseOutlines}
  }
  
  %\usepackage{algorithmic}
  
\usepackage[noend]{algpseudocode}

% edit the comments 
\usepackage{eqparbox}
\renewcommand{\algorithmiccomment}[1]{\eqparbox{COMMENT}{// {\emph #1}}}
  
%\usepackage{multicol}
%\usepackage{amsmath, amssymb}
%\usepackage{amsthm}
\usepackage{enumitem}
%\usepackage{rotating}
%\onehalfspacing
%\newcommand{\tbd}[1]{[{\bf{#1}}]}
\newcommand{\tbd}[1]{}
\newcommand{\ie}{{\it i.e.}}
\newcommand{\eg}{{\it e.g.}}
\newcommand{\etc}{{\it etc.}}
\newcommand{\eat}[1]{}


\newcommand{\mypara}[1]{\medskip\noindent{\bf {#1}.}~}
\newcommand{\bpara}[1]{\noindent{\bf {#1}.}~}
\newcommand{\submypara}[1]{\medskip\noindent{\it {#1}.}~}
\newcommand{\chk}{$\checkmark$}
\newcommand{\dsh}{{\bf --}}
\newcommand{\til}{{\bf\large \textasciitilde}}
\usepackage{rotating}

\usepackage{framed}
\FrameSep0.5pt
\setlength{\topsep}{0pt}

\newcommand{\CTE}{CTE\xspace}
\newcommand{\Name}{APLOMB\xspace}
\newcommand{\Nameplus}{{APLOMB+}\xspace}
\newcommand{\NameArch}{{\sc APLOMB}\xspace}

% RESULTS

\newcommand{\generalFHEslowdown}{9\xspace}
\newcommand{\squaregeneralFHEslowdown}{18}
\newcommand{\strawmanslowdown}{10^6}
\newcommand{\setuptime}{414 s}
\newcommand{\strawmansetuptimesslower}{1.8 \cdot 10^3}

\usepackage{enumitem}
\setlist[itemize]{itemsep=0.00cm}
\setlist[itemize]{parsep=0.01cm}
%\setlist[itemize]{parskip=0.01cm}


\newenvironment{myitemize}
{ \begin{itemize}[nolistsep]
    \setlength{\itemsep}{0.5pt}
    \setlength{\parskip}{0.9pt}
    \setlength{\parsep}{0.1pt}     }
{ \end{itemize}                  } 

\newcommand{\simplequote}[1]{\\\noindent{\it``{#1}''}\linebreak[3]}
\newcommand{\green}[1]{{\color{ForestGreen}{#1}}}


% our encryption scheme
\newcommand{\bbenc}{DPIEnc\xspace}
\newcommand{\bbdetect}{\sys Detect\xspace}

\newcommand{\RG}{RG\xspace}
\newcommand{\MB}{MB\xspace}

\newcommand{\sslk}{k_{\mathsf{SSL}}}
\newcommand{\bbG}{\mathbb{G}}

\newcommand{\garble}{\mathsf{Garble}}
\newcommand{\eval}{\mathsf{Eval}}
\newcommand{\keygen}{\mathsf{KeyGen}}
\newcommand{\Enc}{\mathsf{Enc}}
\newcommand{\kwenc}{\mathsf{KWEnc}}
\newcommand{\enc}{\Enc}
\newcommand{\en}{\mathsf{enc}}
\newcommand{\encleft}{\mathsf{EncLeft}}
\newcommand{\encright}{\mathsf{EncRight}}
\newcommand{\match}{\mathsf{Match}}

\newcommand{\sig}{\mathsf{sig}}
\newcommand{\ct}{\mathsf{ct}}
\newcommand{\sno}{\mathsf{serial\_no}}
\newcommand{\dirtyrange}{\mathsf{dirtyrange}}

\renewcommand{\mod}{\mathsf{mod}\xspace}

\newcommand{\low}{\mathsf{low}\xspace}
\newcommand{\high}{\mathsf{high}\xspace}
\newcommand{\prf}{\mathsf{prf}\xspace}
\newcommand{\IV}{\mathsf{IV}}

\newcommand{\twosnortcom}{67\%}
\newcommand{\twosnortemerge}{42\%}


\newcommand{\SSL}{\mathsf{SSL}}
\newcommand{\rand}{\mathsf{rand}}
\newcommand{\salt}{\mathsf{salt}}
% requires amsthm, enumitem                                                                        
%\theoremstyle{algorithms}
%\newtheorem{construction}{Algorithm}
%\newcommand{\ALGORITHM}[4]{%  name, llabel, intro, \items                                          
%\begin{construction}[#1]\label{#2} \mbox{}
%\normalfont\noindent
% #3
% \vspace{0.13\baselineskip}
% \begin{enumerate}[noitemsep,nolistsep]\itemsep=0.1\baselineskip
% #4
% \end{enumerate}
% \end{construction}
%}


\newcommand{\mf}[1]{ {\fontfamily{cmtt}\selectfont#1}}


\newcommand{\ALGORITHM}[4]{%  name, llabel, intro, \items    
\let\oldi\labelenumi
\let\oldii\labelenumii
\let\oldiii\labelenumiii
\renewcommand{\labelenumi}{\arabic{enumi}: }
\renewcommand{\labelenumii}{\arabic{enumi}.\arabic{enumii}: }
\renewcommand{\labelenumiii}{\arabic{enumi}.\arabic{enumii}.\arabic{enumiii}: }                                     
\noindent \textbf{#1:} \label{#2} 
#3
 \vspace{0.13\baselineskip}
 \begin{enumerate}[noitemsep,nolistsep]\itemsep=0.1\baselineskip
 #4
 \end{enumerate}
 \let\labelenumi\oldi
\let\labelenumii\oldii
\let\labelenumiii\oldii  
}

%\newcommand{\subparagraph}{\paragraph}
\usepackage[compact]{titlesec}

%\newcommand{\simplequote}[1]{\begin{quote}{\it{#1}}\end{quote}}

%\renewcommand\%bibname{}


%\usepackage[compact]{titlesec}  
%\titlespacing{\section}{0pt}{0pt}{0pt}
%\titlespacing*{?command?}{?left?}{?beforesep?}{?aftersep?}[?right?]






\newcommand{\tickYes}{\checkmark}
\newcommand{\todo}[1]{}
\newcommand{\justine}[1]{}
\newcommand{\raluca}[1]{}
\newcommand{\clan}[1]{}
\newcommand{\sr}[1]{{\color{CarnationPink} SR: {#1}}} % :)
\newcommand{\shaddi}[1]{{\color{SkyBlue} SH: {#1}}} % :)
\newcommand{\qu}[1]{{\color{Magenta}  {\bf Question:} {#1}}} 
\newcommand{\warning}[1]{{\color{Red}{\bf Warning: #1}}}

\newcommand{\aes}{\mathsf{AES}}
\newcommand{\encr}{\mathsf{enc\_r}}


% Compact itemize and enumerate.  Note that they use the same counters and                         
% symbols as the usual itemize and enumerate environments.                                         
\makeatletter
\def\compactify{\itemsep=3pt plus3pt \topsep=3pt plus3pt \partopsep=0pt
\parsep=0pt \leftmargin=1.3em}
\let\latexusecounter=\usecounter
\def\CompactItemize{%                                                                              
  \ifnum \@itemdepth >\thr@@\@toodeep\else
    \advance\@itemdepth\@ne
    \edef\@itemitem{labelitem\romannumeral\the\@itemdepth}%                                        
    \expandafter
    \list
      \csname\@itemitem\endcsname
      {\compactify\def\makelabel##1{\hss\llap{##1}}}%                                              
  \fi}
\let\endCompactItemize\endlist
\newenvironment{CompactEnumerate}
  {\def\usecounter{\compactify\latexusecounter}
   \begin{enumerate}}
  {\end{enumerate}\let\usecounter=\latexusecounter}
\makeatother

\authorinfo{Paper \textbf{\#59}}{14 pages}

\begin{document}



%
% --- Author Metadata here ---
%\conferenceinfo{WOODSTOCK}{'97 El Paso, Texas USA}
%\CopyrightYear{2007} % Allows default copyright year (20XX) to be over-ridden - IF NEED BE.
%\crdata{0-12345-67-8/90/01}  % Allows default copyright data (0-89791-88-6/97/05) to be over-ridden - IF NEED BE.
% --- End of Author Metadata ---

\title{\sys: Securely outsourcing middleboxes to the cloud}

\maketitle

%!TEX root = mb.tex

\begin{abstract}

Network middleboxes such as firewalls, NAT, proxies, and intrusion prevention systems are crucial components of modern networks. Recently, more and more organizations are outsourcing their middleboxes to the cloud. However, this poses a problem for the confidentiality of the traffic because now the cloud provider has access to the organization's traffic.


We design and build \sys (read ``embark''), the first system that enables running a wide range of middleboxes at a cloud provider while maintaining the confidentiality of traffic. \sys encrypts the traffic that reaches the cloud and enables the cloud to process the {\em encrypted} traffic without decrypting it.
\sys supports a wide-range of middleboxes such as firewall, NAT, web proxy, load balancing, IP forwarding, intrusion prevention systems, data exfiltration systems, and VPN. Our evaluation shows that \sys supports these applications with competitive performance.
\todo{some numbers?}

\end{abstract}

% A category with the (minimum) three required fields
%\category{H.4}{Information Systems Applications}{Miscellaneous}
%A category including the fourth, optional field follows...
%\category{D.2.8}{Software Engineering}{Metrics}[complexity measures, performance measures]


%\terms{Systems}

%\keywords{ACM proceedings, \LaTeX, text tagging}

%!TEX root = mb.tex

\section{Introduction}\label{sec:intro}

    Network processing appliances (''middleboxes'') such as firewalls, proxies, and WAN optimizers make up a substantial fraction of modern network infrastructure; studies show that as much as 1/3 of enterprise network hardware consists of such devices~\cite{aplomb}.
    However, recent trends suggest that this fraction may begin to decline as more and more networks begin to {\it outsource} their network processing, either to cloud providers~\cite{aplomb, aryaka, zscalar} or to service providers through Network Functions Virtualization (NFV)~\cite{nfv}.
    At the time of this writing, the NFV working group~\cite{nfvwg} has over 250 members ranging from large telecoms to hardware manufacturers, all of whom are investing in new technologies to enable outsourced traffic processing.
    For consumers, outsourcing traffic processing offers many of the same benefits that outsourcing compute and storage have attained through cloud computing: decreased costs, ease of management, the ability to scale and failover on demand, \etc{}.
    Nevertheless, outsourcing network processing brings new challenges, and among them, an issue which is critical to most enterprises: privacy.
    
    Under traditional middlebox deployments, traffic is processed and inspected by devices which are owned, hosted, and managed entirely within the enterprise itself.
    Encrypted traffic is often decrypted to enable deep packet inspection (DPI) such as intrusion detection and exfiltration detection.
    This configuration places critical trust in the hands of network administrators, who can potentially read or even modify {\it any} connection within the company.
    Outsourcing shifts this trust away from network administrators who are employed and monitored by the company whose data is being processed, and to a third party company with its own employees, hiring practices, and motivations.
    Hence, in this paper, we ask: is it possible to enable a third party to perform traffic processing for an enterprise, without transferring the ability to read and monitor the enterprise's traffic?
    
    
\begin{itemize}
\item why aplomb
\item why adding security
\item how to secure (which information need to protect)
\item how does mbark work
\item contribution
\end{itemize}


%!TEX root = mb.tex



     
\section{Overview}\label{sec:overview}


In this section, we present \sys's architecture, the threat model and applications supported.


% TO CUT REMOVE: just mentino the figures and no need to explain 
\subsection{Aplomb/NFV architecture}

We recall the system setup in Aplomb and NFV setup  in Fig.~\ref{fig:sys-overview}.
We do not delve into the details, motivation, and gains of this setup, and refer the reader to~\cite{aplomb} for details. 
There are three parties: enterprise(s), the cloud running the middlebox, and an external site providing
some service. 
The enterprise runs a gateway (GW) which sends traffic to a middlebox (MB) running in the cloud.

There are two typical setups as in Fig.~\ref{fig:sys-overview}.  The first setup, in Fig.~\ref{fig:model1},  occurs when the enteprise communicates with an external site. The second setup occurs when the traffic is sent between two enterprises as in Fig.~\ref{fig:model2}. This allows for a more optimized and faster layout~\cite{aplomb} because  each enterprise has its own gateway.




%The traffic from a client inside the enteprise passes through the gateway on the way out of the enterprise. The gateway redirects this traffic to the middlebox in the cloud.
%After performing various middlebox processing, MB returns the traffic to the gateway. The gateway finally sends the traffic to the external site. 



%Traffic from a client in enterprise 1 passes through the gateway of this enterprise, then it goes  to the middlebox in the cloud, which after processing the traffic, sends it to the gateway of the second enterprise, and the traffic finally arrives at the receiver. This setup allows for better latency, as discussed in~\cite{aplomb}.





\subsection{Threat model}

The goal of \sys is to protect the privacy of the traffic against an attacker at the cloud 
(cloud employee, or hacker gaining access to cloud machines). 
We consider a strong cloud attacker, one that has gained access to {\em all the data on the cloud}.
This includes any traffic and communication the cloud receives from the 
gateway, any old logged information, cloud state, and so on. Nevertheless, we assume that 
the cloud provides good service and runs middlebox functionality correctly.  We are only concerned with 
the traffic's confidentiality.

We assume that the gateways are trusted. They do not leak information.


Some middlebox functionalities (such as intrusion detection or exfiltration detection) have a threat model
of their own regarding the client and the server. For example, intrusion detection assumes that 
the client or the server could misbehave and try to mount an attack, but at most one of them misbehaves~\cite{Bro}  
(indeed, if both misbehave, they can send attack traffic to each other encrypted with a shared symmetric key and fundamentally
no one can detect such an attack).  We preserve all these threat models unchanged. These applications rely
on the good behavior of the middlebox to detect attacks in these threat models. Since we assume the middlebox executes
these functions correctly and \sys preserves the functionality of these middleboxes, 
these threat models are irrelevant to the protocols in \sys, and we will not discuss them again. 



\subsection{\sys overview}


To protect privacy of the traffic, \sys encrypts all traffic passing through the middlebox at the cloud. 
As in Fig.~\ref{fig:sys-overview}, the gateway has a secret key $k$; in the setup with two gateways, they share
the same secret key. The gateway encrypts all traffic going to the middlebox in the cloud using \sys's protocols.
The middlebox in the cloud processes {\em encrypted traffic} using \sys's protocols. 
After the processing, the middlebox
will produce encrypted traffic which it sends back to the gateway. The gateway decrypts the traffic using the key $k$.

Throughout this process, MB handles only the encrypted traffic and never gets the decryption key. This ensures
that an attacker that steals all data from the cloud, will only see encrypted traffic and hence protects the privacy of the 
traffic. 

\sys encrypts IP addresses, ports, and the payload of the packet, thus protecting the privacy of all these parameters. 


{\bf Miscellaneous TODOs:}

- some of the paragraphs above are redundant with intro perhaps

-maybe the overall packet structure here or should this bein systems implementation? 

- need to discuss what happens to SSL connections here; what if the client initiates one? the gateway can no longer
encrypt things

-- handling two gateways with non-determinism

-- goals on the gateway, gateway only needs to know which apps are running, note that in our design
it does not depend on the version, etc. 

-- have a way to show that what the gateway does is generic 
   
-- somewhere we need to discuss goals regarding the gateway

-- why we choose certain operations

which app for which operation


- need to discuss somewhere the above goals from the gateway
%A LOT OF SYSTEMS BUILDING

- security guarantees:

- why we choose certain operations

- Need to summarize somewhere "overall" security

- Somewhere we need to say that range match info leaks for header and keyword match for content. In some sense, these are minimal. 



%!TEX root = mb.tex



\section{Building block: Keyword match}

\warning{this section is a mess. We need to decide what we do with cross-boundary match before I can fix it}
%%!TEX root = mb.tex

\paragraph{Some attacks to be aware of.}
Before the boundary scheme can be secure, it has to withstand the following attacks:

\begin{enumerate}
\item given rule (a1, b1), (a1, b2), an attacker may try to compute rule (b1, b2)
\item given rule (a2, b2), (a1, b1), an attacker creates rule (a1, b2)
\item attacker detects if two boundaries are teh same when neither match a rule string fully
\item \label{step:rule} given rules (a2, b1), (a1, b1), (a2, b2), create a rule for (a1, b2) or detect if this shows up in package
\end{enumerate}

\paragraph{In search for a scheme.}
Intuitively, a secure scheme prevents the rule from being separated in components of $a$ and of $b$. Hence, it cannot be a simple multiplication or xor. 

Here are some directions where I looked for ideas until I found this protocol: polynomial interpolation, garbled circuit inspired, encryption with certain combined key like in garbled circuits, exponentiation, recognize $s | f(s)$, ElGamal type. 

There is an obvious and simple solution with bilinear maps but that would be too slow.  

Another protocol I had relied on composite order groups (hopefully on elliptic curves so exponentiation is cheaper): packets: $R$, $R^{a_1}$, $R^{b_1}$, the encrypted rule is $(a_1b_1)^{-1}) \in \phi_n$. Note that an attacker cannot compute the inverse of this encrypted rule because it does not know $\phi_n$ as in RSA. 


\paragraph{Proof sketch for our scheme.}
Ways to prove secure our boundary scheme. First create a hybrid that replaces the random oracle $\salt, H(\salt, a)$ with an entiry that gets the question ``do you have $a$''. The entity answers yes or no. 

The second hybrid tries to get rid of the random oracles completely for rules that do not match.
Basically, we need to show that one cannot compute $g^{a_1 r_1}$ without getting both of $g^{a_1 b_1}$ and $b_1^{-1} r_1$ for some $b_1$.
We show this in two cases:

Case 1:   one cannot compute $g^{a_1 r_1}$  when given only
$g^{a_1 b_1}$, $c_1^{-1} r_1$, $g^{a_1b_2}$, $b_2^{-1} r_2$, and so forth.

Case 2:   one cannot compute $g^{a_1 r_1}$  when given only
$g^{a_2 b_1}$, $b_1^{-1} r_1$, $g^{a_1b_2}$, $b_2^{-1} r_2$, and so forth.

Otherwise CDH is broken (note that if these were not exponents of $g$, the attacker could mount the attack~\ref{step:rule} above which should be prevented by raising to the power of $g$).

An useful fact to use is that each $r_i$ shows up in only two terms.

 \subsection{Crypto scheme}
 
 \todo{sequence number}
 \todo{say somewhere that we do not have full formalism, see some paper}
 All operations are in a group $\bbG$ of prime order $p$, even if we don't say so.  
 
 The scheme is as follows:
 \begin{itemize}
 \item $\keygen$: generate a random key $k$ for AES and a random value $g$. The final key is $(k, g)$. 
 \item $\kwenc((k,g),  r_1, r_2)$: receives as input a split of a rule into a part $r_1$, at the end of a packet, and $r_2$ at the beginning of the next packet. Outputs the encryption of the rule \[ \encr = g^{\aes_k(r_1, \text{``left"}) \cdot \aes_k(r_2, \text{``right"})}.\] 
 \item $\encleft((k, g), s, \sno)$: receives as input the beginning $s$ of a packet with sequence number $\sno$. It chooses a random $\salt$ and outputs \[ \salt, \ct_L = H(\salt, g^{\aes_k(s) \cdot \aes_k(\sno)^{-1}}).\]
 \item $\encright((k, g), s, \sno):$ receives as input the end of a packet $s$ with sequence number $\sno$ and outputs $\ct_R = \aes_k(s) \cdot \aes_k(\sno+1)$. 
 \item $\match(\encr, (\salt, \ct_L), \ct_R)$: check whether \[H(\salt, \encr^{\ct_R}) \stackrel{?}{=} \ct_L.\]
 \end{itemize}
 
 It is easy to check that if a rule consists of strings $(r_1, r_2)$ and we have $r_1$ at the end (right) of a packet and $r_2$ at the beginning (left) of the next packet, a match will be flagged. 
 The reason is that the right encryption is done with a serial number of $\sno +1$ to match the serial number of the left encryption of the subsequent packet. 
 
The security of the scheme relies on the standard Diffie-Hellman assumption. We prove formally that this scheme achieves the same security as the regular matching scheme (\S\ref{simple_match}) in XXX, as if the rule was not split over two packets. The scheme is implemented over elliptic curves allowing for faster exponentiation and shorter ciphertext sizes. 

Keyword match allows detecting if an encrypted keyword matches an encrypted data item by equality.
For example, given a certain type of encryption of the keyword ``malicious'', $\kwenc(``malicious'')$ someone can determine that it matches the second string in the list of encrypted data items [$\Enc$(``alice''), $\Enc$(``malicious''), $\Enc$(``alice'')] but it should learn nothing else about the other strings.

A few systems such as CryptDB~\cite{cryptdb, someother} used deterministic encryption for this purpose, which has the property that if $x = y$ then $\enc(x) = \enc(y)$. This encryption scheme is weaker than ours because it leaks equality relations even when no keyword is provided for matching. In contrast, we use strong encryption scheme, a randomized encryption scheme that leaks the minimum information needed: which string in the list matches the keyword and nothing else about the strings in the list. 

Since we are in a network setting where the matching is run over packets, sometimes a keyword to be matched spans two packets. \todo{give example here and explain why it is challenging}. Hence we had to come up with a new such scheme. \todo{it is a new contribution}

We employ two types of keyword matches: simple match (as above) and one for cross-boundary match. 


we will call the keywords rules because 

 and Enc(``malicious''), they should be able 

The API of this encryption scheme is in 


\begin{tabular}{c|c|c|c}
Scheme    & Function        & Input(s) & Output \\
\hline
 both                & $\keygen$     & security parameter	      & a secret key $k$ \\ 
                 & $\kwenc$	& $k$, a rule string $r$   &  encryption of $r$, $\encr$ \\
\hline
  simple               & $\enc$           &  $k$, a string $s$                        &  encryption of $s$, $\ct$          \\
                 & $\match$  & $\encr$, $\ct$ & yes/no if there is match \\
\hline
  cross-               & $\encleft$  & $k$, a string $s$ &encryption of $s$, $\ct_L$   \\
   boundary              & $\encright$ & $k$, a string $s$ & encryption of $s$, $\ct_R$  \\
                 & $\match$ & $\encr$, $\ct_L$, $\ct_R$ & yes/no if there is a match \\
\end{tabular}

-- make sure the API so far is clear and a reader can skip the crypto to the systems becasue they don't have
to understand it 
-- might be better without table because it is more clear
\todo{cross bondary has two rules in the encryption}

\subsection{Simple match}


searchable enc

more secure than DET

for full enc explain why it does not leak security, it was already proven and we dont need to 
also for DET point to that paper 

2 types:

- full and boundary
DET special case of full if we do not have repetitions

all randomized and strong

\begin{itemize}
\item deterministic enc
\item strawman: enc every byte -- not secure
\item solution: det enc + tokenization
\end{itemize}

\subsection{Cross-boundary match}

We present this encryption scheme when we talk about IDS in \S\ref{sec:IDS} because its API and design is
motivated by the network setting.
\todo{or should I extract it here?}




%!TEX root = mb.tex


%\section{Building block: Range match}
%An important operation over an encrypted packet is to determine if an encrypted field in the packet falls in an encrypted range.
%We will use the firewall as an example. 
%Consider the following firewall rule:
%
%Constructing an encryption scheme that allows checking if an encrypted value is in an encrypted range, has been a challenge in the applied cryptography community. The reason is that ..
%
%\begin{itemize}
%\item preserve the order between Encryptd values
%\item candidate: OPE
%\item candidate: mOPE
%\item So none of the existing schemes are satisfactory. A new scheme \RM.
%\end{itemize}
%
%\RM applies to cases when we know an upper bound on the values encrypted with OPE and this is a small number of values (say, less than 10,000).
%
%The small number of values permits us to improve in two ways over mOPE [1]
%No more interaction. We store the tree in mOPE on the gateway (client) side, which means that the gateway can compute a new encryption by itself without help from service provider. The storage at the gateway will remain small.
%Rare updates to ciphertexts. We can space out the ciphertexts of the values encrypted sufficiently. 
%
%This also enjoys a stronger security than OPE! It leaks less than order.
%The reason is that, the server does not learn the order between the values in packets, and only whether they map between two values in the rules. 
%
%this one is new
%
%discuss 
%
%would be good to explain the challenge from the 
%
%\todo{a more interesting name to the scheme}
%
%prefix gets mapped into interval, at most a certain number
%
%talk about building certain data structures that all works the same
%
%firewall need not change 

\subsection{Range match } \label{sec:range}






%
The functionality of the RangeMatch scheme is to encrypt a set of ranges $[s_1, e_1]$, $\dots$, $[s_n, e_n]$ into  $[\Enc(s_1)$, $\Enc(e_1)]$, $\dots$, $[\Enc(s_n)$,  $\Enc(e_n)]$, and a value $v$ into $\Enc(v)$, such that anyone with access to these encryptions can determine in which range $v$ lies, while not learning the values of $s_1$, $e_1$, $\dots$, $s_n$, $e_n$, and $v$. 
For concreteness, we explain our scheme by considering $v$, $e_i$ and $s_i$ as IP addresses (although the scheme can be used for encrypting ports too).

\mypara{Requirements}
%
In order for this encryption scheme to fit \sys efficiently and securely, it must:

\begin{CompactEnumerate}[leftmargin=*]

\item  {\em be fast}: the throughput of encryption should be not much lower than network throughput. In particular, the scheme should preserve the ability to use {\em existing fast packet matching algorithms}, such as  various kinds of tries, area-based quadtrees, FIS-trees, or hardware-based algorithms~\cite{packet_classif}.  All of these rely on the ability of SP  to compute ``>'' between $v$ and the endpoints of an interval,
hence the encryption scheme should preserve this property. 




\item {\em provide strong security}: The encryption scheme should not leak $v$, $e_1$, $s_1$, $\dots$, $e_n$, $s_n$ to SP.
Ideally, SP does not learn anything about $v$ other than what interval it matches to. In  particular, even if $v_1$ and $v_2$ match the same range, SP should not learn their order. SP is allowed to learn the order relation of the intervals (in fact, in many setups, SP provides the intervals). \label{req:sec}


\item {\em be deterministic}: To integrate with NAT and to enable middleboxes to piece together packets from the same flow, each value should get  consistently  the same encryption. Any changes in the encryption assigned should happen rarely.  \label{req:injective}

\item {\em be format-preserving}: The encryption should have the same format as the data. Concretely, an encrypted IP address should look like an IP address.  This property is important to avoid changing the packet header structure, which would be a hurdle to adoption. 
 \end{CompactEnumerate}


Unfortunately, there is no encryption scheme that supports all these requirements. The closest schemes to these requirements  is {\em order-preserving encryption}, BCLO~\cite{boldyreva:ope} and mOPE~\cite{popa:mope}. However, these schemes are both significantly less secure and less efficient than our scheme. In terms of security, they leak the order between any two IP addresses encrypted, and not just whether they match the same interval or not. Moreover, BCLO~\cite{boldyreva:ope} leaks the top half of the bits too. In terms of performance, they cannot keep up with packet-processing demands, as they take milliseconds per encryption (as demonstrated in \S\ref{sec:eval}), which would only allow the gateway to forward a few hundred packets per second. 


Instead, we designed a new encryption scheme that satisfies all the requirements above called {\em range match} scheme. 
The scheme takes advantage of the network setting; it does not rely on advanced cryptography, so it can be easily understood and validated by a non-cryptographically trained reader. 


\subsubsection{Our RangeMatch scheme} 



%We explain the scheme based on encryption IP addresses for a firewall and source IP addresses in packets, although the scheme is used in encryption other fields too, as explained in Sec.~\ref{xx}.

To encrypt the endpoints of the intervals, we sort them, and choose as their encryptions values equally distributed in the domain space in a way that preserves the order of the endpoints. For example, the encryption of the intervals 127.0.0.0/8 and 172.16.0.0/16, is [51.0.0.0, 102.0.0.0] and [153.0.0.0, 204.0.0.0]. This preserves the order of the intervals but does not leak anything else about the intervals.

For now, consider that the gateway  maintains a mapping of each interval endpoint  to its encryption, called the {\em interval map}.  The interval map also contains the points $- \infty$ and $+ \infty$, encrypted with 0.0.0.0 and 255.255.255.255. 


When the gateway needs to encrypt an IP address $v$, the gateway first determines what  is the interval  $v$ falls in (we discuss below what happens when more than one interval matches). It uses the interval map to determine the encryptions of the endpoints of this interval. Then, to encrypt $v$, it chooses a random value in this interval.
For example, if $v$ is 127.0.0.1, a possible encryption is 48.124.24.85. This is great for security because the encryption does not retain any information about $v$ other than the range it is in. In particular, for two values $v_1$ and $v_2$ that match the same interval, SP does not learn their order. Thus, this satisfies the security requirement above, and makes our scheme more secure than order-preserving encryption schemes.

We need to specify how to encrypt $v$ when $v$ fits in multiple intervals or in no interval at all. Consider an example in which $v$ fits in no interval at all. Let $v$ be $127.0.0.0$ and the intervals be 18.0.0.0/8 and 172.16.0.0/16 with encryptions [51.0.0.0, 102.0.0.0] and [153.0.0.0, 204.0.0.0]. The encryption of $v$ should not be chosen at random between  102.0.0.0 and 153.0.0.0 because SP learns that $v$ is between the two intervals. Recall that we want SP  to learn only whether $v$ matches an interval or not, but nothing else. Hence, $v$ should be mapped to a random value anywhere in the intervals [0, 51.0.0.0), (102.0.0.0, 153.0.0.0) and (204.0.0.0, 255.255.255.255]. 

To achieve the desired security, the idea is to find the interval $I$ inside which $v$ should be mapped at random. The equation for $I$ is as follows. Consider the intervals {\em inside} which $v$ falls, and let $I_0, I_1, \dots, I_{n_I}$ be their encryptions. We always include the total interval $I_0 = [0,0,0,0, 255.255.255.255]$. Now consider the intervals {\em outside} of which $v$ falls and let $O_1, \dots, O_{n_O}$ be their encryptions. Then, to provide our desired security goal, $v$ should be chosen at random from the interval  
\begin{equation}
 I = I_0 \cap I_1 \cap ... \cap I_{n_I} - (O_1 \cup \dots \cup O_2). \label{eq:randominterval}
 \end{equation}

 


To satisfy requirement~\ref{req:injective}, we need to generate the random encryptions using a deterministic function. For this, we use a pseudorandom function~\cite{GoldreichVol1}, $\prf$, seeded in $k$.   Let $|I|$ be the size of the interval $I$. 
Then, the encryption of $v$ is the $\mathsf{index}$-th element in $I$, where $\mathsf{index}$ is 
\[ \mathsf{index}(v) = \prf_k(v)\ \mod\ |I|. \] 
  Note that, in the system's setup with two gateways, the gateways generate the same encryption because they share $k$. 

When encrypting IP addresses, we do not want two different IP addresses to map to the same encryption (which would break the NAT). Fortunately, the probability that  two IP addresses get assigned to the same encryption is negligibly low for IPv6. This probability is very low for IPv4, but not low enough that a collision will not happen in a large interval of time.  The reason is that  each interval of encryptions is large because we distributed the endpoint encryptions evenly and because there is a small number of such endpoints in a realistic setting (e.g., a firewall has less than 100,000 rules).



The last issue to address is addition and removal of intervals. For example, this can happen when a new rule is added to a firewall. 
Since we tried to spread out the encryptions of the intervals evenly in the IP space, there is no room for the new interval. We can 
readjust all the encryptions of these intervals to make space for the new interval. However, this would require the firewall hardware to reconfigure fully which is slow. Ideally, we would only reconfigure the firewall hardware incrementally, for the new interval. For this, we build on the idea from mOPE~\cite{popa:mope} and store the intervals at the gateway in a balanced scapegoat tree as in Fig.~\ref{fig:tree}. This tree is a search tree that has the property that when inserting or deleting a node, the number of other nodes that change encryption is small, namely, $O(\log n)$ amortized worst case where $n$ is the number of nodes. Each node in the tree is now encrypted similarly to before: the root gets the middle of the IP range,  the left node gets the middle of the IP space to the left of the middle, and so on, as in Fig.~\ref{fig:tree}.  Note that, since the tree is balanced, it maintains our desire of having the encryptions of endpoints roughly uniformly distributed.

\begin{figure}
  \includegraphics[width=3.45in]{fig/tree}
  \caption{\label{fig:tree} Range match tree. The values of nodes in the tree are the unencrypted IP addresses, and the blue values on the horizontal axis are their encryptions. }
\end{figure}



We now explain concretely the API and the implementation of the scheme. 

\subsubsection{Gateway API and implementation}\label{s:rangealg}

\mypara{Gateway state} The gateway keeps a small amount of state (in our implementation, about 200 bytes/range) encryption the intervals, but maintains no state per IP address encrypted or per connection. The gateway stores the tree in Fig.~\ref{fig:tree}: for each node, it stores the unencrypted endpoint, whether it is the left or  right margin of an interval, and the other endpoint of the interval it is part of. It does not need to store the encryption of the endpoint because this is easy to derive from the position in the tree. 


The gateway can use the following functions. EncryptRanges encrypts the initial ranges. Note that some ranges could consist of
one point only, namely $s = e$. 

% TO CUT TO REDUCE: put all these algorithms into one box
% framed takes space around it, above it, below it 

\begin{framed}
\begin{algorithmic}[1]

\Procedure{EncryptRanges}{[$s_1$, $e_1$], $\dots$, [$s_n$, $e_n$]}
  \State Build scapegoat tree on the values 
              $\{s_1, \dots, s_n\}$ 
              $\cup$ $\{e_1, \dots, e_n\}$ 
              $\cup$ $\{-\infty, \infty\}$.
  \State Assign an encryption $\enc(x)$ to each node $x$ in the tree:
  \begin{itemize}
  \item the root gets the middle of the IP range, $e$, 
  \item the node to the left of the root gets the middle of the interval to the left of $e$: ($e/2$),
  \item the right node gets the middle of the range
  to the right of $e$: ($3e/2$), and so on.
  \end{itemize}

  \State \Return{[$\enc(s_1)$, $\enc(e_1)$], $\dots$, [$\enc(s_n)$, $\enc(e_n)$]}
\EndProcedure

\end{algorithmic}
\end{framed}



EncryptValue encrypts the values to be matched against ranges.

\begin{framed}
\begin{algorithmic}[1]

\Procedure{EncryptValue}{$v$}
  \State Search the tree for $v$ to compute efficiently I as in Eq.~\ref{eq:randominterval}.
  \State Compute $\mathsf{index}(v) = \prf_k(v)\ \mod\ |I|.$ 
  \State Let $\enc(v)$ to be the $\mathsf{index}$-th element of $I$. 
  \State \Return $\left(\enc(v), \IV, \aes_k(\IV, v)\right)$ for random $\IV$. 
\EndProcedure

\end{algorithmic}
\end{framed}

Here is how to compute $I$ efficiently. When searching for $v$ in the tree, the gateway
can identify the tightest enclosing interval [$p_1$, $p_2$] in logarithmic time. 
 If $[p_1, p_2]$ are endpoints of the
same interval, then I = [$p_1$, $p_2$]. Otherwise, move towards the left in the tree until you identify the first endpoint
$\ell_1$
that belongs to an interval $[\ell_1, \ell_2]$ enclosing $v$. Then, using the tree, scan $[\ell_1, \ell_2]$ and eliminate
any intervals not containing $v$. The gateway can precompute and store this interval for every two consecutive nodes in the tree.

EncryptValue returns an AES encryption of $v$ too, because $\enc(v)$ is not decryptable. 

We now describe the procedure for AddRange and DeleteRange which add or delete an interval. 
These will modify the state at the gateway. Besides the interval added or deleted, a small number
of other intervals may be moved. For these, the algorithm returns the old and new encryption of the interval. 


\begin{framed}
\begin{algorithmic}[1]

\Procedure{AddRange}{$[s, e]$}
  \State Insert $s$ and $e$ into the scapegoat tree. If $s=e$, insert the value only once.
  %	
  \State Initialize $L$ to be the empty list.
  \If{tree needs to be rebalanced}
  	\State Record which nodes change position in the tree during rebalancing, together with 
	their old and new encryptions. Namely, record	\[L = \{ \en_1 \leftarrow \en^*_1, \dots, \en_m \rightarrow \en^*_m\},\] where $m$ is the number of nodes who changed position in the tree, and $\en_i$ and $\en^*_i$ are the old and new encryption of the $i$-th node that changed position. 
  \EndIf
  \State Compute  $\enc(s)$ and $\enc(e)$, the encryptions of $s$ and $e$, as in EncryptRanges.
   \State \Return{$[\enc(s), \enc(e)], L$}
\EndProcedure

\end{algorithmic}
\end{framed}

%  \State Determing the smallest and the largest encryption in the values $[\enc(s), \enc(e)]$ and $L$, and call this $\dirtyrange$.

Since we are using a scapegoat tree, the number of nodes that change position during rebalancing is amortized worst case $O(\log n)$ where $n$ is the total number of nodes in the scapegoat tree. 

A natural question is whether there exists a scheme that does not need to adjust already encrypted intervals. For a related setting, Popa et al.~\cite{popa:mope} proved that whenever one supports ``>'' over  encrypted data, there must be some adjustment similar to this; the proof holds for our setting too.  

DeleteRange is similar, except that the result contains only $L$. 

\subsubsection{Cloud API}

The cloud can run ``$\ge$'' and "$\le$" between any encrypted value $\enc(v)$ and an encrypted endpoint $\enc(s)$ and $\enc(e)$, and will obtain a correct answer. Computing $\enc(v_1) < \enc(v_2)$  between two encrypted values in the same range is meaningless, and returns a random value.


\subsubsection{Security guarantees}

The scheme achieves our desired security goal: the only information leaked about a value $v$ encrypted is which ranges it matches. 
In particular, the scheme is not order preserving because it does not leak the order of two encrypted values that match the same range. It is easy to check that the scheme is secure: since the encryption of $v$ is random in $I$ (Eq.~\ref{eq:randominterval}), the scheme only leaks the fact that $v$ is in $I$. $I$ is chosen in such a way that the only information about $v$ it encodes is which intervals $v$ matches and which it does not match.




% PROTOCOL

% the gateway encrypts teh IP has to be clear

% Here somewhere we need to discuss that there is one encryption scheme adn then there is how we use it for ports and other ip addresses.

% in the firewall protocol describe how hardware does not change 

% can use the same processing 


%!TEX root = mb.tex

\todo{should have all header MBs into one section to save some space?}
\todo{would really make things better to have the borderbox send ipdates to all, you mention it here not much in overview, 
not useful for DPI anyways}
\todo{have a figure here}

\section{Header-only middleboxes}




\subsection{Middlebox: Firewall}\label{sec:firewall}

\todo{there was some text somewhere about a tool we use to switch to ipv6}

We use the term ``firewall'' for stateful and stateless packet filters that filter the traffic based on network layers and transport layers. Stateless firewalls commonly examine the combination of the packet's source and destination address, its protocol, and for TCP/ UDP traffic, its source and destination port number. Stateful firewalls additionally keep track of protocol states for each flow and retain packets until they have enough information to make decision. 
Our RangeMatch scheme supports both types of firewalls. We now explain the design of the firewall based on this scheme.

\mypara{Setup} Initially, the gateway (G) encrypts the rules to be used by the firewall, by encrypting all IP addresses and ports in the rules, as follows.

First, it prepares the IP and port intervals. Recall that RangeMatch requires the IP addresses to be IPv6 to guarantee the injectivity of encryption (see \S\ref{sec:range}); hence, we extend an IPv4 prefix  such as 157.161.48.0/24 to an IPv6 one  ::ffff:157.161.48.0/120. 
% Now the prefix represents the range from ::ffff:157.161.48.0 to ::ffff:157.161.48.255.
 This problem does not show up for ports because they only need to be distinct within the same IP address.
The gateway then expands every prefix into an interval, and every exact match $x$ into [$x$, $x$]. 

Next, the gateway encrypts these intervals using EncryptRanges (\S\ref{sec:range}) by running one instance of RangeMatch for IP addresses and one instance for ports.
It then convets each encrypted IP range into a set of prefixes and duplicates the rule for each prefix in this set. 

Consider an example rule from  \mf{pf}, the 
default firewall under BSD:
 \mf{ block out log quick on \$ext\_if from} \\ \mf{157.161.48.0/24 to any.}
Let the encryption of 157.161.48.0/24 be the interval [80.0.0.0, 160.0.0.0] (for brevity, we use IPv4). 
This interval is equivalent to the prefixes: 80.0.0.0/4, 96.0.0.0/3, 144.0.0.0/4, and 160.0.0.0/32. 
Hence, the gateway replaces the original rule with four rules, one for each encrypted prefix. 
The worst-case number of prefixes for IPv6 is $O$($\min$($\log$ number of rules, $128$)) = $O$($\log$ number of rules), 
which is small. 


% Firewalls from different vendors may vary significantly in terms of rule syntaxes and organizations. However,
% in general both hardware and software firewalls have a few interfaces. Both ingress and egress of an interface 
% can be associated with an access control list (ACL). Each ACL has a number of rules, possibly in the form 
% <action, protocol, src ip, src port, dst ip, dst port>. Without loss of generality, we take \mf{pf}, the 
% default firewall under BSD, as an example to illustrate how \sys works with firewalls. Figure \ref{fig:fwrule1} 
% shows an example of \mf{pf} rules. 





The gateway sends the new rules to the service provider (SP) which installs them into the firewall {\em the same way it would install 
them if they were not encrypted}. 

\mypara{Processing traffic}
When a packet arrives at the gateway, the gateway encrypts its source/destination IP addresses and ports using the EncryptValue algorithm
and fits these into the packet header because RangeMatch is format preserving.
For example, for an IP address $v$, EncryptValue produces $\enc(v)$ and $\IV, \aes_k(\IV, v)$.
 $\enc(v)$ preserves the format of an IP address and hence 
it will fit in the packet header in the place of the unencrypted IP address. $\IV, \aes_k(\IV, v)$ is placed in the packet's extension as in Fig.~\ref{fig:packet}.

The gateway sends the packet to the firewall. The firewall can execute on the encrypted header {\em
the same way as on the unencrypted header} because RangeMatch maintains the order relation between values in rules and in 
packet headers. 
In particular, it can use any of the existing fast matching algorithms unchanged. 
Moreover, it can still use  any specialized hardware without changes. This property is important, since many high-speed firewalls are implemented in hardware, which is difficult and expensive to redesign.


\mypara{Updating rules} 
Let us discuss the case of adding a new firewall rule; deleting a rule is similar.
The gateway runs AddRange from \S\ref{s:rangealg}. This produces a new encrypted range [$\enc(s), \enc(e)$] 
along with a list $L$ of other encrypted ranges that get updated. The gateway sends [$\enc(s), \enc(e)$] 
 and the list $L$ to the firewall. SP converts these to changes in the firewall rules and 
reconfigures the firewall.  Due to the guarantees of our RangeMatch protocol, this list contains a 
small number (logarithmic) of intervals that changed, so the overhead for the reconfiguration should be modest.
 
The list of range changes $L$ and  [$\enc(s), \enc(e)$] 
 is relevant to other middleboxes too: since the encryption of a value $v$ depends on the tree of ranges, an IP address $v$, which used to be encrypted into $e$ before the ranges were changed is now encrypted into $e'$; this can break logic at other middleboxes which stored $e'$. We discuss how the NAT uses this information in \S\ref{sec:nat}.





\subsection{Middlebox: NAT}\label{sec:nat}

The fast path of the NAT in \sys{} is unchanged from the regular NAT. The NAT maps IP addresses/ports that are encrypted to external IP addresses/ports 
as if they were not encrypted, as in Fig.~\ref{fig:packetflow}. 

The difference is  when the firewall sends a list of range changes $L$ and maybe a new interval  [$\enc(s), \enc(e)$] .
The NAT identifies all  encrypted IP addresses/ports  it has mapped that are affected: it computes an interval that covers all intervals in $L$ and $\enc(s), \enc(e)$ and checks which values fit in it. It then sends these values to the gateway, who reencrypts them using EncryptValue (\S\ref{s:rangealg}) so the NAT updates them. 

Such adjustment of encryptions are necessary for an encryption scheme like RangeMatch, as discussed in \S\ref{sec:range}. Nevertheless, this operation happens rarely and is not expensive: it happens only when firewall rules change and the number of ranges changed is  amortized logarithmic.

\subsection{Middlebox: Proxy/cache}\label{s:proxy}
\todo{Cut out any of this, it's leftover text from something I tried that didn't work but didn't want to delete in cas it winds up being helpful}
They also {\it terminate} connections rather than allowing packets to ``pass through''.
When a client opens a new HTTP connection, a typical proxy will capture the client's SYN packet and open a new connection to the client, as if it were the web server the client wished to connect to. 
The proxy then opens a second connection in the background to the original web server, as if it were the client. :w

Multiple clients may attempt to access the same web page through the proxy, in which case, the proxy maintains multiple client-facing connections, but one or few persistent server-facing connections.
When a client sends a request for new content, the proxy can either forward the request to the web server, or, the proxy may serve the content from its {\it cache} -- images and content that other clients have already requested which the proxy then stored locally. 
Caching improves client-perceived performance because content is served from the proxy, which is closer to the client than the web server.

In order to understand what content the client is requesting (index.html from google.com, photo.jpg from flickr.com, \etc{}), the proxy must parse the HTTP header, which, unlike IP/TCP/UDP headers is variable-length (\eg{}, URLs may be any number of characters long, while port numbers are always 16 bytes long) and variable-offset (\eg{}, the IP source always appears 12 bytes from the beginning of the packet, where the HTTP method may appear anywhere within an HTTP payload).

\sys implements web proxying using the keyword match encryption algorithm. 
However, rather than encrypting fixed values at fixed locations, the \sys gateway parses the HTTP header to determine what data to encrypt.
Nevertheless, as we show in \S\ref{sec:eval}, this parsing has a negligible overhead on gateway throughput -- less than 1\% when added in addition to the existing encryption required for Firewalling and NAT.
We implement two versions of gateway encryption: one which uses the stateless gateway and can encode HTTP requests which are not pipelined, and one which uses the stateful gateway, which supports pipelined HTTP requests as well (we discuss the difference as follows).

First case: unpipelined. HTTP header is **always** the first few bytes in the first packet sent by the client -- so easy to know where to start parsing.
\begin{itemize}
\item architecture overview
\item HTTP request -- absolute url and Host + relative url
\item tokenization special case \\
    \begin{itemize}
    \item in order to fetch the url from the header
    \item tokenize HTTP methods: GET, POST, ...
    \item tokenize attribute names: "Host:"
    \item tokenize versions: "HTTP/1.1", ...
    \end{itemize}
\item walk through the whole process
\item do not support pipelining yet
\end{itemize}

- may want to say that it has an index for seaching fast the url which it can thanks to the det scheme


we are focusing on the transparent proxy
- discuss the kind of proxies we are focusing on

proxies have two benefits: latency savings which aplomb gives you 
and bandwidth savings, which aplomb does nogive you
 
L7 Proxy / Cache

explain how the proxy finds if there is  a match --looks at header bytes
extra field attached to the packet -- gateway understands http and points out file id, 

discuss both cases: cache miss and hit

- two indep tcp connections 

how to populate the cache: content providers pushing the content to CDNs, or the gateway understands
http and tags what is content and what is ID
oops justine forgot to mention this above

the web proxy needs to send http response 

proxy only looks at the top N bits corresponding to a large header and matches the file id with the entire path
and matches GET and a few others -- avoid parsing http this way


discuss how the proxy can reconstruct the response?
can proxy reconstruct the ip header and the http header  and the tcp header without being able to encrypt
--- check the http header details

With pipelined requests, we don't have the header is always the first part of the packet. Instead, we need to {\it reconstruct the payload} to tell where to parse. This is where we need to use the other gatway. Otherwise, the functionality is exactly the same above -- only now we operate on the reconstructed payload rather than the first few bytes of the first packet.



%!TEX root = mb.tex



\section{Middlebox: IDS}\label{sec:ids}
Unlike the previously presented middleboxes, devices which perform Intrusion Detection/Prevention (IDS/IPS) operate over the TCP bytestream, not just over packets and headers independently.
Hence, deploying an IDS at the cloud requires a stateful gateway at the enterprise.

\sys's IDS is based on BlindBox~\cite{blindbox}, an IDS which uses searchable encryption to detect malicious signatures within encrypted HTTPS connections. We highlight the functionality of BlindBox as follows, but exclude many important details all of which are presented in~\cite{blindbox}

We use BlindBox as follows.
BlindBox leaves HTTPS unmodified, but augments it with a secondary channel which transmits encrypted {\it exact match tokens}.
In it's simplest implementation, for every byte transmitted over HTTPS, the sender transmits an encrypted token using using the same approach as our keyword match algorithm.
The encrypted token encodes that byte, and the next 8 bytes; \eg{}, a sender transmitting the word `maliciousl' would transmit the encrypted tokens for [$\enc$(`maliciou'), $\enc$(`alicious')].
An IDS looking for the word `malicious' would declare a match if it observed both these tokens consecutively.
In practice, the sender does not send encrypted tokens for {\it every} byte, but optimizes out some which are redundant or unnecessary (we defer to~\cite{blindbox} for details about these optimizations).
For IDS signatures which consist {\it only} of exact match rules, keyword match is sufficient to detect attacks.
The security guarantee it gives is as follows: the IDS will learn what a byte in the flow is if it is {\it suspicious}, \ie{}, it matches a substring of a known attack.

  However, for IDS signatures which include regular expressions, keyword match is not sufficient to completely perform signature detection; further, there exists no \todo{RALUCA::: FAST ENOUGH SCHEME} to detect over encrypted data).
  Hence, to support regular expressions, BlindBox -- and \sys -- provide a secondary protocol with a different security guarantee. 
  All IDS rules include exact match strings -- for performance reasons, an IDS like Snort~\cite{Snort} only performs regular expression detection after all exact match strings have been detected first.
  Inspired by this, BlindBox allows the middlebox to decrypt the session {\it only} if suspicious exact match strings have already been detected.
  The client encodes the key with the tokens such that if a match occurs, the middlebox can automatically decrypt the key (once again, details are in~\cite{blindbox}).
  This approach can detect regular expressions, but provides a different security guarantee. 
  The first model allows the middlebox to {\it detect a substring, only if that substring is suspicious.} 
  The second model allows the middlebox to {\it decrypt the entire stream} of a substring is suspicious.
  This model is called `probable cause decryption'.

  Everything we have described to this point is identical between BlindBox and \sys. We now describe how they are different, and how this improves performance and security.

The key difference between BlindBox and how \sys implements IDS comes in how rules are given the IDS.
With \sys, the gateway can simply encrypt rules with its key $k$ and transmit them to the IDS, who then applices them over the enterprises traffic.
With BlindBox, this is not appropriate.
BlindBox aims to allow a client (\eg{} laptop) connecting to {\em any} network to receive IDS processing over their HTTP traffic, even if the IDS is unknown or untrusted to them.
In this scenario, the IDS wants to {\it enforce} that all traffic be scanned for malicious behavior, but the client wants to maintain {\it privacy} from the untrusted IDS.
The middlebox will not give its rules to the client, and the client will not give its key to the middlebox.
 To generate the encrypted rules that the IDS can use for detection requires BlindBox to perform
a sophisticated computation at the start of each connection, which makes the connection
startup time be slow. 
Further, this computation scales with the number of exact-match rules the IDS needs to learn: the more rules, the longer the startup time.
\sys has no such overhead.

The first implication of this is performance: this startup time is eliminated, as the rules are generated once, at the gateway, and transmitted to the IDS. After this there is no more setup cost.

The second implication is improved security security. We showed how BlindBox has a stronger security guarantee for exact matches than it does for regular expressions. However, many regular expressions can be converted to exact matches. 
For example, the regular expression 'alice[1-3]' is equivalent to any of the exact matches [`alice1', `alice2', `alice3'].
To do this expansion with BlindBox would be prohibitively expensive, as each additional exact match rules increases the already lengthy setup time.
However, as \sys has no such cost, we can expand many regular expressions and thus detect them using the stronger security model.
Not all rules are amenable to this -- \eg{} `bob[a-z]+' would result in a prohibitively large (and also far too general) number of expansions, even for \sys. 
In \S\ref{sec:eval}, we find for some rulsets regular expression expansion almost doubles the number of rules that can be detected using the stronger security model.


\section{Middlebox: Proxy/cache}\label{s:proxy}
\todo{Cut out any of this, it's leftover text from something I tried that didn't work but didn't want to delete in cas it winds up being helpful}
They also {\it terminate} connections rather than allowing packets to ``pass through''.
When a client opens a new HTTP connection, a typical proxy will capture the client's SYN packet and open a new connection to the client, as if it were the web server the client wished to connect to. 
The proxy then opens a second connection in the background to the original web server, as if it were the client. :w

Multiple clients may attempt to access the same web page through the proxy, in which case, the proxy maintains multiple client-facing connections, but one or few persistent server-facing connections.
When a client sends a request for new content, the proxy can either forward the request to the web server, or, the proxy may serve the content from its {\it cache} -- images and content that other clients have already requested which the proxy then stored locally. 
Caching improves client-perceived performance because content is served from the proxy, which is closer to the client than the web server.

In order to understand what content the client is requesting (index.html from google.com, photo.jpg from flickr.com, \etc{}), the proxy must parse the HTTP header, which, unlike IP/TCP/UDP headers is variable-length (\eg{}, URLs may be any number of characters long, while port numbers are always 16 bytes long) and variable-offset (\eg{}, the IP source always appears 12 bytes from the beginning of the packet, where the HTTP method may appear anywhere within an HTTP payload).

\sys implements web proxying using the keyword match encryption algorithm. 
However, rather than encrypting fixed values at fixed locations, the \sys gateway parses the HTTP header to determine what data to encrypt.
Nevertheless, as we show in \S\ref{sec:eval}, this parsing has a negligible overhead on gateway throughput -- less than 1\% when added in addition to the existing encryption required for Firewalling and NAT.
We implement two versions of gateway encryption: one which uses the stateless gateway and can encode HTTP requests which are not pipelined, and one which uses the stateful gateway, which supports pipelined HTTP requests as well (we discuss the difference as follows).

First case: unpipelined. HTTP header is **always** the first few bytes in the first packet sent by the client -- so easy to know where to start parsing.
\begin{itemize}
\item architecture overview
\item HTTP request -- absolute url and Host + relative url
\item tokenization special case \\
    \begin{itemize}
    \item in order to fetch the url from the header
    \item tokenize HTTP methods: GET, POST, ...
    \item tokenize attribute names: "Host:"
    \item tokenize versions: "HTTP/1.1", ...
    \end{itemize}
\item walk through the whole process
\item do not support pipelining yet
\end{itemize}

- may want to say that it has an index for seaching fast the url which it can thanks to the det scheme


we are focusing on the transparent proxy
- discuss the kind of proxies we are focusing on

proxies have two benefits: latency savings which aplomb gives you 
and bandwidth savings, which aplomb does nogive you
 
L7 Proxy / Cache

explain how the proxy finds if there is  a match --looks at header bytes
extra field attached to the packet -- gateway understands http and points out file id, 

discuss both cases: cache miss and hit

- two indep tcp connections 

how to populate the cache: content providers pushing the content to CDNs, or the gateway understands
http and tags what is content and what is ID
oops justine forgot to mention this above

the web proxy needs to send http response 

proxy only looks at the top N bits corresponding to a large header and matches the file id with the entire path
and matches GET and a few others -- avoid parsing http this way


discuss how the proxy can reconstruct the response?
can proxy reconstruct the ip header and the http header  and the tcp header without being able to encrypt
--- check the http header details

With pipelined requests, we don't have the header is always the first part of the packet. Instead, we need to {\it reconstruct the payload} to tell where to parse. This is where we need to use the other gatway. Otherwise, the functionality is exactly the same above -- only now we operate on the reconstructed payload rather than the first few bytes of the first packet.


\section{Middlebox: Intrusion Detection}
Like proxies, Intrusion Detection/Prevention systems operate over packet payloads, not just the IP/TCP/UDP headers.
IDS/IPS are canonical examples of DPI devices.




%!TEX root = mb.tex

\section{Other middleboxes}\label{sec:vpn} \label{sec:other_apps} \label{sec:not_supp}\label{sec:loadb}

In this section, we discuss briefly other middleboxes \sys supports, which are straightforward extensions of the ones presented so far. 

\bpara{L4 and L7 Load balancer} 
The L4 load balancer spreads the load from packets destined to the same IP address across different servers. \sys supports this middlebox in a 
very similar way to the NAT presented in \S\ref{sec:nat}. The L7 load balancer spreads the load destined for the same URL or path across 
different servers. \sys supports this middlebox in a very similar way to the web proxy presented in \S\ref{s:proxy}. 


\bpara{IP forwarding} This middlebox is implemented similarly to a NAT and firewall. 


\bpara{VPN} The APLOMB~\cite{aplomb} model supports the VPN by installing an APLOMB client at the user endpoint, as part of the VPN client.
\sys also installs a \sys client as part of the VPN client which does the encryption and decryption work of the gateway. The VPN middlebox remains
largely the same because it does not compute on the traffic. 


\bpara{Application firewall} This middlebox is a simple version of an IDS combined with a firewall. 


\bpara{WAN optimizers} WAN optimizers compress network traffic. APLOMB~\cite{aplomb} requires the compression to happen
at the gateway, because otherwise, it loses the bandwidth benefits of the compression. Hence, this middlebox does not benefit much 
 from the cloud outsourcing. 
Nevertheless, we can still support it. 
If the other middleboxes used are only header middleboxes, WAN optimizers can be  supported without modification.
If one uses an IDS middlebox too, \sys  requires  the IDS tokenization  to run before the compression so it can tokenize
the traffic; in this case, unfortunately, \sys  reduces the compression benefit of the WAN optimizer because the encrypted tokens 
are not compressible. 




% some potentially useful things
%\section{Discussion}
%
%No one will put a firewall behind a NAT
%
%since we are giving primitives, one might be able to add later applications that can also be supported with the same primitives, although one has to add encryption at the gateway for the specific fields 
%
%What do we do about these issues?
%
%Checksum Many middleboxes change packet contents, and thus packet check-
%sums. However, it is not clear that if we can compute correct checksums based
%
%on encrypted packet contents.
%
%IP Fragmentation Middleboxes working above L3 may need to reassemble
%
%IP fragments, our encryption scheme need to be carefully designed to avoid any
%
%issues.





%!TEX root = mb.tex

\section{Implementation} \label{sec:impl}

\todo{this is important to write properly because it is a systems paper and there were some nontrivial decisions}.

Things to cover:

- gateway and middlebox implementation

- second flow udp/tcp

- new packet structure and headers -- graph? (there is some text and a figure in a google doc about this)

- some things are specific to web proxy from what I remember



%!TEX root = mb.tex

\section{Evaluation} \label{sec:eval}

As we showed in \S\ref{sec:impl}, \sys supports all middlebox applications in typical outsourcing environments~\cite{aplomb,nfv} -- including header-only middleboxes (which \sys supports without modification) as well as DPI middleboxes (which \sys supports with modifications to their codebase). 
Hence, from a functionality perspective, \sys answers our original question, ``Is it possible to enable a third party to perform traffic processing for an enterprise, {\em without seeing the enterprise's traffic}?''  strongly in the affirmative.

To evaluate \sys more deeply, we now investigate whether \sys is practical from a performance perspective, looking at the overheads due to encryption and redirection. 
Overall, we find that \sys provides client performance comparable to APLOMB~\cite{aplomb} -- e.g., page load times increase by \todo{foo}\% relative to APLOMB when caching is disabled, and by \todo{bar\%} when caching is enabled.
\sys does impose some overhead at the gateway and middleboxes:
\sys with no DPI reduces gatway throughput by \todo{foo\%} and has zero overhead at the outsourced middleboxes. \sys with DPI enabled imposes a higher overhead, with the gateway throughput reduced by \todo{bar\%} and typical middlebox throughput reduced to \todo{baz\%} of baseline. 

\todo{methodology, datasets...}

\subsection{End-to-End Performance}
We first inspect end-to-end client performance when traffic is redirected through \sys.

{\it Does \sys provide web performance comparable to other outsourcing approaches?}
Page load times relative to APLOMB.

{\it How much does bandwidth increase between the gateway and the cloud from using \sys? How much would this bandwidth increase an enterprises networking costs?}



\subsection{Gateway Performance}
\begin{figure}[t]
  \includegraphics[width=3in]{fig/gatewayxput}
  \caption[]{\label{fig:gwxput} Throughput at the gateway without \sys, with header-only \sys, with HTTP-aware \sys, and with DPI-enabled \sys.}
\end{figure}


\noindent{\it How does encryption impact throughput at the outsourcing gateway relative to an outsourcing gateway without \sys?}
Figure~\ref{fig:gwxput}...

\begin{figure}[t]
  \includegraphics[width=3in]{fig/xputrange}
  \caption[]{\label{fig:xputrange} Throughput as number of rules for range encrypt increases.}
\end{figure}


\begin{figure}[t]
  \includegraphics[width=3in]{fig/latencyrange}
  \caption[]{\label{fig:latencyrange} Per-packet latency as number of rules for range encrypt increases.}
\end{figure}


\noindent{\it How do throughput and latency at the gateway scale with the number of rules for range encryption?} 
Figs~\ref{fig:latencyrange} and~\ref{fig:xputrange}

\noindent{\it What is the memory overhead of the stateful range map encryption scheme?}


\subsection{Middlebox Throughput}

\begin{table}[t!]
\begin{tabular}{p{2.5cm}|p{1.4cm}|p{2cm}|p{1cm}}
Application & Header / HTTP / DPI & Baseline xput & \sys xput \\
\hline \hline
IP Firewall &   &  &  \\
Application Firewall & & & \\
NAT & Yes  & &  \\
IP Forwarding & & & \\
VPN Gateway &  &  &  \\ 
Load Balancer L4 & & & \\
Load Balancer L7 & & & \\
WAN optimizer  & & & \\
Web proxy/cache forward/ reverse & & &\\
IDS & & & \\
Exfiltration/parental filtering & & &  \\
\todo{split this} 
\end{tabular}
\caption{Middlebox applications supported by \sys and their throughput with an emprical traffic workload. \label{tbl:appsxput}}
\end{table}

\noindent{\it Is throughput reduced at the middleboxes due to \sys?}
(Hopefully we can say, for most apps, not at all)
Table~\ref{tbl:appsxput}...


\subsection{Header-Only Middleboxes}


\noindent{\bf Firewalls.} {\it Does \sys support all rules in a typical firewall configuration? How much does the ruleset ``expand'' due to encryption?}

\noindent{\it How often do updates to the firewall require a rule refresh? How long does it take to refresh rules at the firewall?}

\noindent{\bf NAT.}
\noindent{\bf LB...}

\subsection{DPI Middleboxes}

\noindent{\bf Proxy/Caching.}
{\it How many active connections per second can the Proxy accept? How does this compare to an unencrypted Proxy implementation, like Squid?}

{\it What improvement in page load times does a client experience due to proxying, relative to no proxy at all? Relative to an unencrypted proxy implementation?}

\noindent{\bf Intrusion Detection}
{\it How much does MBArk reduce flow completion times relative to BlindBox~\cite{blindbox}?}

\begin{table}[h]
\centering
\begin{tabular}{l|l|l|l|l}
&MBArk&APLOMB&BlindBox&SSL\\
\hline
\hline
Handshake&&&&\\
\hline
Total FCT&&&&\\
\end{tabular}

\end{table}

\noindent{\it What fraction of IDS rules can be supported without requiring decryption?}
Table, Zhi's stuff.


%!TEX root = mb.tex

\section{Related work}\label{sec:related}

compare to BlindBox

get some related work from BlindBox

any other work trying to protect the traffic? 

Vern has a few papers on packet trace anonymization for offline analysis. The objectives, techniques, and contexts are different, but more or less related.

http://www.icir.org/enterprise-tracing/devil-ccr-jan06.pdf
http://www.icir.org/vern/papers/bro-anonymizer-sigcomm03.pdf



\subsection{Computing on encrypted data}

\subsection{Theory}


\subsection{Systems}

Compare to CryptDB here carefully. 
- need to make clear that these enc schemes are not like in CryptDB, that it is not just another CryptDB; only the vision is the same; % pioneered the vision of extracting core operations and then supporting them. Of course, the core operations are different here
none of the technical part is the same, the enc schemes are different and in fact they are all more secure. 

\subsection{Work related to our building blocks}

-- what we build on here and the relationship of OPE with mOPE and others, range queries and others


 mOPE unfortunately requires that the gateway and the service provider interact for a number of roundtrips (e.g., xxx in our experiments) which is too slow and requires additional setup for this interaction, and violates requirement~\ref{req:sec} or~\ref{req:injective}, and BCLO has weak security (leaking always the top half bits of the values encrypted and the order of IP addresses across different packets, thus violating requirement~\ref{req:sec}), is too slow, and not format-preserving. 

we do not readjust for encryption - this is expensive, we do not leak data between two encryptions 
% some points on comparison to mOPE from a technical standpoint
%The tree is stored at the gateway. The tree contains as nodes the ends of the intervals as opposed to all values encoded-- thus, the tree is much smaller. firewalls have on the order of thousands such rules, so the tree is not large. also store only ranges and not everything encoded, making it smaller and fit into the gateway, etc., they need adjustments when they encrypt too, etc. -- better point to related work for this
%Difference:
% we encode different values in the tree, have a different encryption algorithm, and create a much smaller tree that can be stored at the gateway. no roundtrips any more; they don't have the deterministic property
%store the tree at the gateway.
% This tree is stored at the gateway. The tree stores edges of the interval 
% one important point is that there are ciphertext updates only for rule changes and not for regular encryption


%!TEX root = mb.tex

\section{Conclusions} \label{sec:concl}



In this paper, we presented \sys, the first system that enables running a wide range of middleboxes at a service provider 
while maintaining the confidentiality of traffic.
    \sys delivers on the promise of APLOMB and NFV to reduce costs, decrease the burden of managing and configuring these devices, and provide redundant resources for elasticity and fault tolerance.
    However, \sys provides strong privacy for enterprises by  by enabling the service provider to compute on the encrypted traffic without decrypting it. 
We showed that \sys supports a wide-range of middleboxes, and despite the encryption, \sys has modest overheads. In our tests on EC2, middlebox throughput with \sys was nearly unchanged relative to normal middlebox processing. In our local experiments, we saw that a single server is capable of generating 8Gbps of encrypted traffic: enough for most enterprises to replace all of their middleboxes with only a single server.




%
% The following two commands are all you need in the
% initial runs of your .tex file to
% produce the bibliography for the citations in your paper.
\bibliographystyle{acm}
    \bibliography{related_work,cryptobib,rp,rp-str,rp-conf}
  % sigproc.bib is the name of the Bibliography in this case
% You must have a proper ".bib" file
%  and remember to run:
% latex bibtex latex latex
% to resolve all references




%
% ACM needs 'a single self-contained file'!
%
%APPENDICES are optional
\balancecolumns

% That's all folks!
\end{document}
