%!TEX root = mb.tex

\section{Related work}\label{sec:related}

Confidentiality of data in the cloud has been widely recognized as an important problem and researchers proposed solutions for generic applications~\cite{Baumann:Haven}, web applications \cite{giffin:hails, Mylar},  filesystems~\cite{blaze:cfs, kallahalla:plutus, goh:sirius},  databases~\cite{popa:cryptdb},  and virtual machines~\cite{Zhang:CloudVisor}. However, there has been almost no work on data confidentiality for network processing in the cloud. 
%\sys is the first system that enables running a wide range of middleboxes at the cloud while protecting the confidentiality of the traffic from the cloud.  

A few systems~\cite{Vern:Anonymize03, Vern:Anonymize06} attempt to anonymize the packet stream in the hope of hiding the identity of the hosts.
Unlike \sys, this approach breaks certain middlebox functionality and does not hide the data content.
Yamada et al.~\cite{Yamada_IDS} show how one can perform some very limited processing on an 
SSL-encrypted packet.
     by using only the size of the data and the timing of packets. % While this provides privacy, it does not 
   %  support middlebox processing. 


As discussed, computing on encrypted data promises to provide both confidentiality and functionality. Nevertheless, generic cryptography schemes~\cite{BSW11,gentry:fhe} remain impractical by orders of magnitude~\cite{gentry:fhe-aes-eprint}.
CryptDB~\cite{popa:cryptdb} introduced a new vision for building practical systems that compute on encrypted data, but none of its encryption schemes or the systems techniques  apply to our setting.  The order-preserving encryption~\cite{boldyreva:ope} used in CryptDB is 
 $>3000$ times slower than RangeMatch (\S\ref{sec:eval}) and cannot sustain realistic network throughputs; moreover, it leaks the order of the IP addresses encrypted, while RangeMatch protects this information. %Similarly, we use KeywordMatch for equality checks which is more secure than CryptDB's deterministic encryption.% because it is randomized. 
 Finally, the database techniques in CryptDB do not apply to our networking setting. 

The BlindBox~\cite{blindbox} system enables running an IDS on encrypted traffic. Our IDS is built on top of BlindBox, but it is much more efficient and more secure than BlindBox as discussed in \S\ref{sec:ids} and \S\ref{sec:eval}. Moreover, BlindBox focuses only on DPI and IDS and does not provide solutions for other middleboxes. %such as firewall, NAT, load balancer, web proxy and others;  \sys supports these with new techniques. %In addition, \sys takes an overall systems view and integrates all the different middleboxes into a system architecture that can process encrypted traffic in the cloud in a multitude of ways.


\eat{
Let us now present related work to our RangeMatch scheme. 
As  mentioned, order-preserving encryption  (OPE) schemes \cite{boldyreva:ope}, \cite{popa:mope} is less secure than RangeMatch because they leak order, and are $>3000$ slower than RangeMatch, which is not feasible for networks.
The  encryption scheme of Boneh et al.~\cite{BonehRange} enables detecting if an encrypted value matches a range and provides a similar security to ours. 
This scheme is less efficient, orders of magnitude more than OPE itself, its ciphertext size is exponential in the size of the input, and it does not preserve the ordering between inputs, which is the basis of packet classification algorithms used by firewalls and other applications. By taking advantage of the networks and systems setting, we constructed a much more efficient scheme. 
}
%  mOPE unfortunately requires that the gateway and the service provider interact for a number of roundtrips (e.g., xxx in our experiments) which is too slow and requires additional setup for this interaction, and violates requirement~\ref{req:sec} or~\ref{req:injective}, and BCLO has weak security (leaking always the top half bits of the values encrypted and the order of IP addresses across different packets, thus violating requirement~\ref{req:sec}), is too slow, and not format-preserving. 

% HERE ARE A FEW USEFUL NOTES ABOUT HOW OUR DESIGN IS DIFFERENT FROM mOPE -- THERE IS SOME SIMILARITY DUE TO TREE AND ADJUST
% we do not readjust for encryption 
% - this is expensive, we do not leak data between two encryptions 
% The tree is stored at the gateway. The tree contains as nodes the ends of the intervals as opposed to all values encoded-- thus, the tree is much smaller. firewalls have on the order of thousands such rules, so the tree is not large. also store only ranges and not everything encoded, making it smaller and fit into the gateway, etc., they need adjustments when they encrypt too, etc. -- better point to related work for this
% Difference:
% we encode different values in the tree, have a different encryption algorithm, and create a much smaller tree that can be stored at the gateway. no roundtrips any more; they don't have the deterministic property
%store the tree at the gateway.
% This tree is stored at the gateway. The tree stores edges of the interval 
% one important point is that there are ciphertext updates only for rule changes and not for regular encryption







