%!TEX root = mb.tex

\section{Evaluation} \label{sec:eval}

As we showed in \S\ref{sec:impl}, \sys supports all middlebox applications in typical outsourcing environments~\cite{aplomb,nfv} -- including header-only middleboxes as well as bytestream-aware middleboxes . 
Hence, from a functionality perspective, \sys answers our original question, ``Is it possible to enable a third party to perform traffic processing for an enterprise, {\em without seeing the enterprise's traffic}?''  strongly in the affirmative.

To evaluate \sys more deeply, we now investigate whether \sys is practical from a performance perspective, looking at the overheads due to encryption (over the header or the payload) and redirection. 
\sys has very low hardware overheads at the enterprise: our eight-core server could easily saturate a 10Gbps link with encrypted traffic to and from the cloud; we evaluate the gateway throughput in \S\ref{sec:gateway}.
Because most middleboxes are unchanged in the dataplane, throughput overheads at the cloud due to \sys are negligible, typically under 1\%, as we show in \S\ref{sec:homiddleboxes}.  

We ran our experiments using the implementation described in \ref{sec:impl}. 
Our prototype gateway runs with four 2.6GHz Xeon E5-2650 cores and 128 GB RAM; the network hardware is a single 10GbE Intel 82599 compatible network card.
We deployed our prototype in our research lab and redirected traffic from a 3-server testbed through the gateway; these three client servers had the same hardware specifications as the server we used as our gateway.
We deployed our middleboxes on EC2 large instances.
For most experiments, we use a synthetic workload generated by the DPDK PacketGen~\cite{pktgen}; for experiments where an empirical trace is specified we use the m57 patents trace~\cite{m57} and the ICTF 2010 trace~\cite{ictf}.

In what follows, we evaluate our performance at the enterprise, including gateway throughput, end-to-end page load times, and bandwidth costs (\S\ref{sec:enterprise}). We then evaluate performance at the cloud, evaluating each middlebox we implemented one by one (\S\ref{sec:evalcloud}).

\subsection{Enterprise Performance}
\label{sec:enterprise}
We first evaluate \sys's overheads at the enterprise, including the gateway, end-to-end performance, and bandwidth costs from deploying \sys.

\subsubsection{Gateway}

\begin{figure}[t]
  \centering
  \begin{tabular}{cc}
  \includegraphics[height=1.25in]{fig/gatewayxput}&
  \includegraphics[height=1.25in]{fig/gateway_pps}\\
  \end{tabular}
  \includegraphics[width=2.75in]{fig/key}
  \caption[]{\label{fig:gwxput} Throughput/Packets per Second on a single core at the stateless gateway.\todo{Fix labels, add back in DPI}}
\end{figure}

\noindent{\it How many servers does a typical enterprise require to outsource traffic to the cloud?}
Figure~\ref{fig:gwxput} shows the gateway throughput when encrypting traffic to send to the cloud, first with normal redirection (as used in APLOMB~\cite{aplomb}), then with \sys's IP/TCP-header encryption, and finally with IP/TCP-header encryption as well as HTTP/proxy encryption. 
For empirical traffic traces with payload encryption (DPI) disabled, \sys averages 1.5Gbps per core; our 8-core server could easily saturate a 10Gbps link.
With DPI enabled (not shown), throughput dropped to 240Mbps, suggesting that an enterprise would need a small cluster to reach higher bandwidth.

While DPI introduces a reduction to about one-sixth relative to other encryption schemes, there is little difference between the HTTP overhead and the IP/TCP overhead, as the HTTP encryption only occurs on HTTP requests -- a small fraction of packets. Overall, \sys encryption for the stateless gateway reduces by about 60\% relative to baseline APLOMB encryption in the worst case (the min-sized workload; the reduction for the empirical (m57) workload is 38\%.  

\begin{figure}[t]
  \includegraphics[width=3.25in]{fig/xputrange}
  \caption[]{\label{fig:xputrange} Throughput as number of rules for range encrypt increases.}
\end{figure}
\todo{call them rules or ranges? depending on the flow}

\noindent{\it How do throughput and latency at the gateway scale with the number of rules for range encryption?} 
In \S\ref{sec:range}, we discussed how our range encryption scheme stores encrypted values in a tree; every packet encryption requires a traversal of this tree.
Hence, as the size of the tree goes larger, we can expect to require more time to process each packet and throughput to decrease.
We measure this effect in Figure~\ref{fig:xputrange}. 
On the $y_1$ axis, we show the aggregate per packet throughput at the gateway as the number of rules from 0 to 100k. The penalty here is logarithmic, which is the expected performance of tree data structures. From 0-10k rules, throughput drops from 670Kpps to 480Kpps; after this point the performance penalty of additional rules tapers off. Adding an additional 90k rules drops throughput to 400Kpps.
On the $y_2$ axis, we measure the processing time per packet, \ie{}, the amount of time from when the gateway begins encrypting the packet to when the gateway completes encrypting the packet; the processing time follows the same logarithmic trend.
\todo{new latency numbers}

\noindent{\it Is range encryption faster than existing order preserving algorithms?}
Our range encryption is the only encryption scheme that has low enough latency for packet processing while preserving the ordering information needed for firewall rules. 
Existing order-preserving approaches require multiple round trip times -- and hence many milliseconds -- for each encryption operation.
Our range encoding approach encrypts each value in microseconds.
We compare against BCLO~\cite{boldyreva:ope} and mOPE~\cite{popa:mope} below:

\begin{table}[h]
\centering
\begin{tabular}{c|c|c|c}
Operation&BCLO~\cite{boldyreva:ope}&mOPE~\cite{popa:mope}&\sys\\
\hline
\hline
Encrypt, 10K rules&9333$\mu$s&6640$\mu$s&1.95$\mu$s\\
\hline
Encrypt, 100K rules&9333$\mu$s&8300$\mu$s&3$\mu$s\\
\hline
Decrypt&169$\mu$s&0.128$\mu$s&0.128$\mu$s\\
\hline
\end{tabular}
\end{table}

\noindent{\it What is the memory overhead of the stateful range map encryption scheme?}
Storing 10k rules in memory requires 1.6MB, and storing 100k rules in memory requires 28.5MB -- using unoptimized C++ objects.
This state overhead is negligible on any modern server.




\subsubsection{Client Performance}

\begin{figure}
  \hspace{-15pt}
  \begin{tabular}{cccc}
  \includegraphics[height=1in]{fig/cdflabel}
  &\hspace{-10pt}\includegraphics[height=1in]{fig/e2e_loadtimes}
  &\hspace{-10pt}\includegraphics[height=1in]{fig/e2e_delta_absolute}
  &\hspace{-10pt}\includegraphics[height=1in]{fig/e2e_delta_relative}
  \\
  &(a)&(b)&(c)\\
  \end{tabular}
  \caption[]{\label{fig:e2eloads} Page load times through \sys compared to direct download.}
\end{figure}

We use web performance to understand end-to-end client experience using \sys.
Figure~\ref{fig:e2eloads} shows a CDF for the Alexa top-100 sites loaded through our testbed, we plot (a) the page load times in seconds, (b) the absolute increase in page load times in seconds, and (c) the relative increase in page load time as a multiple of the baseline. 
Because of the `bounce' redirection \sys uses, all page load times increase; in the median case this increase is less than a second. At the 90th percentile, page loads increase by 2 seconds.
Half of all pages see less than a 50\% overhead in load times.
\todo{These numbers are so-so.???}

\subsubsection{Bandwidth Overheads}
We evaluate two costs: the increase in bandwidth due to our encryption and metadata, and the increase in bandwidth cost due to `bounce' redirection.

\noindent{\it How much does \sys encryption increase the amount of data sent to the cloud?}
The gateway inflates the size of traffic due to three encryption costs:
\begin{itemize}
  \item If the enterprise uses IPv4, there is a 20-byte per-packet cost to convert from IPv4 to IPv6. If the enterprise uses IPv6 by default, there is no such cost.
  \item If HTTP proxying is enabled, there are on average 132 bytes per request in additional encrypted data.
  \item If HTTP IDS is enabled, there is at worst a 5$\times$ overhead on all HTTP payloads.
\end{itemize}
We used the m57 trace to understand how these overheads would play out in aggregate for an enterprise.
On the uplink, from the gateway to the middlebox service provider, traffic would increase by 2.5\% due to encryption costs for a Header-Only Gateway. Traffic would increase by 4.3$\times$ on the uplink for a bytestream-aware gateway. 


\noindent{\it How much does bandwidth increase between the gateway and the cloud from using \sys? How much would this bandwidth increase an enterprises networking costs?}
\sys sends all network traffic to and from the middlebox service provider for processing, before sending that traffic out to the Internet at large. In NFV contexts, the clients' middlebox service provider and network connectivity provider are one and the same and one might expect costs for relaying the traffic to and from the middleboxes to be rolled in to one `package.' 
However, in the APLOMB setting, the middlebox service provider is a cloud, meaning that the client must pay a third party ISP to transfer the data to and from the cloud, before paying that ISP a third time actually transfer the data over the network.

Using current bandwidth pricing~\cite{comcast-costs, megapath-costs, verizon-costs}, we can estimate how much outsourcing would increase overall bandwidth costs.
Multi-site enterprises typically provision two kinds of networking costs: Internet access, and intra-domain connectivity. 
Internet access typically has high bandwidth but a lower SLA -- 99 or 99.5\% uptime; traffic may also be sent over shared Ethernet~\cite{comcast-costs, verizon-costs}.
Intra-domain connectivity usually has a private, virtual Ethernet link between sites of the company with a high SLA (over 99\%) and lower bandwidth.
This latter connectivity tends to cost about two orders of magnitude more than the former, because of the high SLA, extra configuration, and support required.
Because bounce redirection is over the `cheaper' link, the overall impact to bandwidth cost with header-only encryption given public sales numbers is between 15-50\%; with bytestream aware encryption this cost increases to between 30-150\%. 

\eat{Getting frustrated with these numbers so leaving them here and will come back to them. Megapath offers a dedicated link at 5x5Mbps for \$250/mo; 20x20Mbps for \$1300/mo. Comcast offers enterprise cable at 150/20Mbps for \$250/mo. The three way bounce should result in a cost increase of 15-50\%, depending on how much the internal link is providioned for. The DPI should be 2-3 times that, so, between 30-150\%... 
}

\subsection{Middleboxes}
\label{sec:evalcloud}

\begin{table}[t!]
\begin{tabular}{p{3cm}|p{2cm}|p{2cm}}
Application &  Baseline Throughput & \sys Throughput \\
\hline \hline
IP Firewall &  9.8Gbps &  9.8Gbps \\
%Application Firewall  & & \\
NAT & 3.6Gbps   &   3.5 Gbps \\
%IP Forwarding  & & \\
%VPN Gateway &  &  &  \\ 
Load Balancer L4  &9.8 Gbps & 9.8Gbps \\
%Load Balancer L7 & & & \\
%WAN optimizer  & & & \\
Web Proxy &1.1Gbps &1.1Gbps\\
%IDS & & & \\
IDS & 85Mbps & 166Mbps~\cite{blindbox}   \\
\end{tabular}
\caption{Middlebox applications supported by Header-Only \sys; throughput measured with an empirical traffic workload. \label{tbl:appsxput}}
\
\end{table}

\noindent{\it Is throughput reduced at the middleboxes due to Header-Only \sys?}
Table~\ref{tbl:appsxput}...\todo{FILL IN}



\noindent{\bf Firewalls.} 
{\it Does \sys support all rules in a typical firewall configuration? How much does the ruleset ``expand'' due to encryption?}

\noindent{\it How often do updates to the firewall require a rule refresh? How long does it take to refresh rules at the firewall?}

\noindent{\bf NAT.}
\noindent{\bf LB...}

\noindent{\bf Proxy/Caching.}
{\it How many active connections per second can the Proxy accept? How does this compare to an unencrypted Proxy implementation, like Squid?}

{\it What improvement in page load times does a client experience due to proxying, relative to no proxy at all? Relative to an unencrypted proxy implementation?}


\noindent{\bf Intrusion Detection}
Our IDS is based on BlindBox~\cite{blindbox}. However, \sys affords better performance and stronger privacy than BlindBox. BlindBox incurs a substantial `setup cost' every time a client initiates a new connection. With \sys, however, the gateway and the cloud maintain one, long-term persistent connection. 
Hence, this setup cost is paid once when the gateway is initially configured. This results in two benefits:

\noindent{\it (1) End-to-end performance improves.} Where BlindBox incurs an initial handshake of 414s~\cite{blindbox} to open a new connection, clients under \sys never pay this cost; instead they perform a normal TCP or SSL handshake of only 3-5 RTTs. In our testbed, this amounts to between 30 and 100 ms, depending on the site and protocol -- an improvement of 4 orders of magnitude.

\noindent{\it (2) Security improves.} As we presented in \S\ref{sec:ids}, BlindBox operates at one of two security levels: a stronger security level for exact-match strings, and a weaker security level for regular expressions. The stronger security level requires a round of handshake for each rule, but the weaker security level does not require more handshakes for each regular expression. 
With \sys, the initial handshake/setup is performed exactly once, between the gateway and the middlebox. 
Hence, we can convert regular expressions to exact match strings, even if they result in many hundreds of times more exact match rules. 
Consequently, more rules can be detected using the stronger security guarantee, without incurring any handshake overhead.
Using IDS rulesets from Snort, we converted regular expressions to exact match strings as discussed in \S\ref{sec:ids}; we were able to convert about half of all regular expressions to a finite number of exact match strings. 
Consequently, \sys can detect more then 80-88\% of attacks using the higher security level, rather than 42-67\% as with BlindBox:

\begin{table}[h]
  \centering
  \begin{tabular}{l|c|c}
    {\bf Dataset}&{\bf Exact Match}&{\bf Prob. Cause}\\
    \hline
    \hline
    BlindBox: Community&67\%&100\%\\
    \hline
    \sys: Community&88.7\%&100\%\\

    \hline
    \hline
    BlindBox: Emerging Threats&42\%&100\%\\
    \hline
    \sys: Emerging Threats&79.8\%&100\%\\
    \hline
  \end{tabular}
\end{table}


