%!TEX root = mb.tex

\section{Other middleboxes}\label{sec:vpn} \label{sec:other_apps} \label{sec:not_supp}\label{sec:loadb}

In this section, we discuss briefly other middleboxes \sys supports, which are straightforward extensions of the ones presented so far. 

\bpara{L4 and L7 Load balancer} 
The L4 load balancer spreads the load from packets destined to the same IP address across different servers. \sys supports this middlebox in a 
very similar way to the NAT presented in \S\ref{sec:nat}. The L7 load balancer spreads the load destined for the same URL or path across 
different servers. \sys supports this middlebox in a very similar way to the web proxy presented in \S\ref{s:proxy}. 


\bpara{IP forwarding} This middlebox is implemented similarly to a NAT and firewall. 


\bpara{VPN} The APLOMB~\cite{aplomb} model supports the VPN by installing an APLOMB client at the user endpoint, as part of the VPN client.
\sys also installs a \sys client as part of the VPN client which does the encryption and decryption work of the gateway. The VPN middlebox remains
largely the same because it does not compute on the traffic. 


\bpara{Application firewall} This middlebox is a simple version of an IDS combined with a firewall. 


\bpara{WAN optimizers} WAN optimizers compress network traffic. APLOMB~\cite{aplomb} requires the compression to happen
at the gateway, because otherwise, it loses the bandwidth benefits of the compression. Hence, this middlebox does not benefit much 
 from the cloud outsourcing. 
Nevertheless, we can still support it. 
If the other middleboxes used are only header middleboxes, WAN optimizers can be  supported without modification.
If one uses an IDS middlebox too, \sys  requires  the IDS tokenization  to run before the compression so it can tokenize
the traffic; in this case, unfortunately, \sys  reduces the compression benefit of the WAN optimizer because the encrypted tokens 
are not compressible. 




% some potentially useful things
%\section{Discussion}
%
%No one will put a firewall behind a NAT
%
%since we are giving primitives, one might be able to add later applications that can also be supported with the same primitives, although one has to add encryption at the gateway for the specific fields 
%
%What do we do about these issues?
%
%Checksum Many middleboxes change packet contents, and thus packet check-
%sums. However, it is not clear that if we can compute correct checksums based
%
%on encrypted packet contents.
%
%IP Fragmentation Middleboxes working above L3 may need to reassemble
%
%IP fragments, our encryption scheme need to be carefully designed to avoid any
%
%issues.


