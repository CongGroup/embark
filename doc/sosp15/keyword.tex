%!TEX root = mb.tex



\section{Building blocks}


\subsection{Keyword match}

\warning{this section is a mess. We need to decide what we do with cross-boundary match before I can fix it}
%%!TEX root = mb.tex

\paragraph{Some attacks to be aware of.}
Before the boundary scheme can be secure, it has to withstand the following attacks:

\begin{enumerate}
\item given rule (a1, b1), (a1, b2), an attacker may try to compute rule (b1, b2)
\item given rule (a2, b2), (a1, b1), an attacker creates rule (a1, b2)
\item attacker detects if two boundaries are teh same when neither match a rule string fully
\item \label{step:rule} given rules (a2, b1), (a1, b1), (a2, b2), create a rule for (a1, b2) or detect if this shows up in package
\end{enumerate}

\paragraph{In search for a scheme.}
Intuitively, a secure scheme prevents the rule from being separated in components of $a$ and of $b$. Hence, it cannot be a simple multiplication or xor. 

Here are some directions where I looked for ideas until I found this protocol: polynomial interpolation, garbled circuit inspired, encryption with certain combined key like in garbled circuits, exponentiation, recognize $s | f(s)$, ElGamal type. 

There is an obvious and simple solution with bilinear maps but that would be too slow.  

Another protocol I had relied on composite order groups (hopefully on elliptic curves so exponentiation is cheaper): packets: $R$, $R^{a_1}$, $R^{b_1}$, the encrypted rule is $(a_1b_1)^{-1}) \in \phi_n$. Note that an attacker cannot compute the inverse of this encrypted rule because it does not know $\phi_n$ as in RSA. 


\paragraph{Proof sketch for our scheme.}
Ways to prove secure our boundary scheme. First create a hybrid that replaces the random oracle $\salt, H(\salt, a)$ with an entity that gets the question ``do you have $a$''. The entity answers yes or no. 

The second hybrid tries to get rid of the random oracles completely for rules that do not match.
Basically, we need to show that one cannot compute $g^{a_1 r_1}$ without getting both of $g^{a_1 b_1}$ and $b_1^{-1} r_1$ for some $b_1$.
We show this in two cases:

Case 1:   one cannot compute $g^{a_1 r_1}$  when given only
$g^{a_1 b_1}$, $c_1^{-1} r_1$, $g^{a_1b_2}$, $b_2^{-1} r_2$, and so forth.

Case 2:   one cannot compute $g^{a_1 r_1}$  when given only
$g^{a_2 b_1}$, $b_1^{-1} r_1$, $g^{a_1b_2}$, $b_2^{-1} r_2$, and so forth.

Otherwise CDH is broken (note that if these were not exponents of $g$, the attacker could mount the attack~\ref{step:rule} above which should be prevented by raising to the power of $g$).

An useful fact to use is that each $r_i$ shows up in only two terms.

 \subsection{Crypto scheme}
 
 \todo{sequence number}
 \todo{say somewhere that we do not have full formalism, see some paper}
 All operations are in a group $\bbG$ of prime order $p$, even if we don't say so.  
 
 The scheme is as follows:
 \begin{itemize}
 \item $\keygen$: generate a random key $k$ for AES and a random value $g$. The final key is $(k, g)$. 
 \item $\kwenc((k,g),  r_1, r_2)$: receives as input a split of a rule into a part $r_1$, at the end of a packet, and $r_2$ at the beginning of the next packet. Outputs the encryption of the rule \[ \encr = g^{\aes_k(r_1, \text{``left"}) \cdot \aes_k(r_2, \text{``right"})}.\] 
 \item $\encleft((k, g), s, \sno)$: receives as input the beginning $s$ of a packet with sequence number $\sno$. It chooses a random $\salt$ and outputs \[ \salt, \ct_L = H(\salt, g^{\aes_k(s) \cdot \aes_k(\sno)^{-1}}).\]
 \item $\encright((k, g), s, \sno):$ receives as input the end of a packet $s$ with sequence number $\sno$ and outputs $\ct_R = \aes_k(s) \cdot \aes_k(\sno+1)$. 
 \item $\match(\encr, (\salt, \ct_L), \ct_R)$: check whether \[H(\salt, \encr^{\ct_R}) \stackrel{?}{=} \ct_L.\]
 \end{itemize}
 
 It is easy to check that if a rule consists of strings $(r_1, r_2)$ and we have $r_1$ at the end (right) of a packet and $r_2$ at the beginning (left) of the next packet, a match will be flagged. 
 The reason is that the right encryption is done with a serial number of $\sno +1$ to match the serial number of the left encryption of the subsequent packet. 
 
The security of the scheme relies on the standard Diffie-Hellman assumption. We prove formally that this scheme achieves the same security as the regular matching scheme (\S\ref{simple_match}) in XXX, as if the rule was not split over two packets. The scheme is implemented over elliptic curves allowing for faster exponentiation and shorter ciphertext sizes. 

Keyword match allows detecting if an encrypted keyword matches an encrypted data item by equality.
For example, given a certain type of encryption of the keyword ``malicious'', $\kwenc(``malicious'')$ someone can determine that it matches the second string in the list of encrypted data items [$\Enc$(``alice''), $\Enc$(``malicious''), $\Enc$(``alice'')] but it should learn nothing else about the other strings.

A few systems such as CryptDB~\cite{cryptdb, someother} used deterministic encryption for this purpose, which has the property that if $x = y$ then $\enc(x) = \enc(y)$. This encryption scheme is weaker than ours because it leaks equality relations even when no keyword is provided for matching. In contrast, we use strong encryption scheme, a randomized encryption scheme that leaks the minimum information needed: which string in the list matches the keyword and nothing else about the strings in the list. 

Since we are in a network setting where the matching is run over packets, sometimes a keyword to be matched spans two packets. \todo{give example here and explain why it is challenging}. Hence we had to come up with a new such scheme. \todo{it is a new contribution}

We employ two types of keyword matches: simple match (as above) and one for cross-boundary match. 


we will call the keywords rules because 

 and Enc(``malicious''), they should be able 

The API of this encryption scheme is in 


\begin{tabular}{c|c|c|c}
Scheme    & Function        & Input(s) & Output \\
\hline
 both                & $\keygen$     & security parameter	      & a secret key $k$ \\ 
                 & $\kwenc$	& $k$, a rule string $r$   &  encryption of $r$, $\encr$ \\
\hline
  simple               & $\enc$           &  $k$, a string $s$                        &  encryption of $s$, $\ct$          \\
                 & $\match$  & $\encr$, $\ct$ & yes/no if there is match \\
\hline
  cross-               & $\encleft$  & $k$, a string $s$ &encryption of $s$, $\ct_L$   \\
   boundary              & $\encright$ & $k$, a string $s$ & encryption of $s$, $\ct_R$  \\
                 & $\match$ & $\encr$, $\ct_L$, $\ct_R$ & yes/no if there is a match \\
\end{tabular}

-- make sure the API so far is clear and a reader can skip the crypto to the systems becasue they don't have
to understand it 
-- might be better without table because it is more clear
\todo{cross bondary has two rules in the encryption}

\subsubsection{Simple match}


searchable enc

more secure than DET

for full enc explain why it does not leak security, it was already proven and we dont need to 
also for DET point to that paper 

2 types:

- full and boundary
DET special case of full if we do not have repetitions

all randomized and strong

\begin{itemize}
\item deterministic enc
\item strawman: enc every byte -- not secure
\item solution: det enc + tokenization
\end{itemize}

\subsubsection{Cross-boundary match}

We present this encryption scheme when we talk about IDS in \S\ref{sec:IDS} because its API and design is
motivated by the network setting.
\todo{or should I extract it here?}


