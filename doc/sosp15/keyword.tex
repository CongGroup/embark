%!TEX root = mb.tex



\section{Building blocks}

In this section we present the building blocks \sys relies on: keyword match and range match. The first scheme already existed and the second is a new encryption scheme we provide. 

When describing these schemes, for concreteness, we refer to the encryptor as the gateway whose secret key is $k$ and to the entity computing on the encrypted data as the service provider (SP) because this is how these schemes will be used by \sys.


\subsection{Keyword match}\label{s:kwmatch}


KeywordMatch allows detecting if an encrypted keyword matches an encrypted data item by equality.
For example, given an encryption of the keyword ``malicious'', and a list of encrypted strings  [$\Enc$(``alice''), $\Enc$(``malicious''), $\Enc$(``alice'')], SP can  detect that the keyword matches the second string. 
For  this, we use a searchable encryption scheme~\cite{song:search, blindbox}.
Using this scheme, the gateway can encrypt a value $v$ into $\enc(v)$ and a rule $r$ into $\encr$ and SP can detect if there is a match between $v$ an $r$. 
%In such a scheme, 
%to encrypt a string $x$, the gateway computes 
%$\enc(v) = (\salt, H(\salt, \aes_k(v))),$ where $\salt$ is a random value, $H$ is a hash function and $\aes$ is the AES algorithm.
%To encrypt a keyword $r$ to search for, the gateway computes $\encr = \aes_k(r)$. 
%Given an encrypted string $\enc(v) = (\salt, A)$ and an encrypted keyword $\encr$, SP knows that there is a match if  $H(\salt, \encr) = A$.  
 The security of searchable encryption is well studied~\cite{song:search, blindbox}: at a high level,  given a list of encrypted strings, and an encrypted keyword, SP does not learn anything about the encrypted strings, other than which strings match the keyword. % In \sys, we will not be concerned with hiding the keyword, although $\encr$ does hide it. 
 The encryption of the strings is {\em randomized}, so it does not leak whether two encrypted strings are equal to each other, unless, of course, they both match the encrypted keyword. 
 %For instance, in the example above, SP cannot tell that the first and third encrypted strings 
%in the list [$\Enc$(``alice''), $\Enc$(``malicious''), $\Enc$(``alice'')] are equal. The reason is that they each have a different salt.  
%
As an optimization, in some applications, we can use a deterministic encryption scheme instead of the scheme above {\em without sacrificing security}: this is the case when we are guaranteed that the strings encrypted are distinct. %In this case, the salt is not needed, and we can simply encrypt a value $v$ with $\Enc(v) = \aes_k(v)$. This enables fast equality matching and building indexes for fast search (which will be useful for the HTTP proxy middlebox in \S\ref{s:proxy} ).



