%!TEX root = mb.tex

\section{Related Work}
\label{sec:related}

\noindent{\bf Middlebox Outsourcing:}
APLOMB~\cite{aplomb} is a practical service for outsourcing enterprise's middleboxes to the cloud, which we discussed in more detail in \S\ref{sec:overview}.

\noindent{\bf Data Confidentiality:}
Confidentiality of data in the cloud has been widely recognized as an important problem and researchers proposed solutions for software~\cite{Baumann:Haven}, web applications \cite{giffin:hails, Mylar},  filesystems~\cite{blaze:cfs, kallahalla:plutus, goh:sirius},  databases~\cite{popa:cryptdb, blindseer},  and virtual machines~\cite{Zhang:CloudVisor}. 
CryptDB~\cite{popa:cryptdb} was one of the first practical systems to compute on encrypted data, but its encryption schemes and database system design do not apply to our network setting. 

Focusing on traffic processing, the most closely related work to \sys is BlindBox~\cite{blindbox}, discussed in \S\ref{sec:bbarch}.  mcTLS~\cite{mctls} proposed a protocol in which client and server can jointly authorize a middlebox to process certain portions of the encrypted traffic. Unlike \sys, the middlebox  gains access to {\em unencrypted data}. A recent paper~\cite{secmb} proposed a system architecture for outsourced middleboxes to specifically perform deep packet inspection over encrypted traffic.

\noindent{\bf Trace Anonymization and Inference:}
Some systems which focus on {\it offline} processing allow some analysis over anonymized data \cite{Vern:Anonymize06, Vern:Anonymize03}; they are not suitable for online processing as is \sys.
Yamada et al~\cite{Yamada_IDS} show how one can perform some very limited processing on an SSL-encrypted packet by using only the size of data and the timing of packets, however they cannot perform analysis of the contents of connection data.

\noindent{\bf Encryption Schemes:}
\sys's PrefixMatch scheme is similar to order preserving encryption schemes~\cite{agrawal:ope}, but no existing scheme provided both the performance and security properties we required.
Order-preserving encryption (OPE) schemes such as~\cite{boldyreva:ope, popa:mope}  are 
 $>10000$ times slower than PrefixMatch (\S\ref{sec:eval}) and additionally leak the order of  the IP addresses encrypted. On the other hand, OPE schemes are more generic and applicable to a wider set of scenarios. PrefixMatch, on the other hand, is designed for our  particular scenario.

The encryption scheme of Boneh et al.~\cite{BonehRange} enables detecting if an encrypted value matches a range and provides a similar security guarantee to PrefixMatch; at the same time, it is orders of magnitude slower than the OPE schemes which are already slower than PrefixMatch. 

%  mOPE unfortunately requires that the gateway and the service provider interact for a number of roundtrips (e.g., xxx in our experiments) which is too slow and requires additional setup for this interaction, and violates requirement~\ref{req:sec} or~\ref{req:injective}, and BCLO has weak security (leaking always the top half bits of the values encrypted and the order of IP addresses across different packets, thus violating requirement~\ref{req:sec}), is too slow, and not format-preserving. 

% HERE ARE A FEW USEFUL NOTES ABOUT HOW OUR DESIGN IS DIFFERENT FROM mOPE -- THERE IS SOME SIMILARITY DUE TO TREE AND ADJUST
% we do not readjust for encryption 
% - this is expensive, we do not leak data between two encryptions 
% The tree is stored at the gateway. The tree contains as nodes the ends of the intervals as opposed to all values encoded-- thus, the tree is much smaller. firewalls have on the order of thousands such rules, so the tree is not large. also store only ranges and not everything encoded, making it smaller and fit into the gateway, etc., they need adjustments when they encrypt too, etc. -- better point to related work for this
% Difference:
% we encode different values in the tree, have a different encryption algorithm, and create a much smaller tree that can be stored at the gateway. no roundtrips any more; they don't have the deterministic property
%store the tree at the gateway.
% This tree is stored at the gateway. The tree stores edges of the interval 
% one important point is that there are ciphertext updates only for rule changes and not for regular encryption







