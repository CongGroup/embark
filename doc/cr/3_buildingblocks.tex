%!TEX root = mb.tex



\section{Cryptographic Building Blocks}
\label{sec:buildingblocks}

In this section we present the building blocks \sys relies on.
Symmetric-key encryption (based on AES) is well known, and we do not discuss it here. Instead, we briefly discuss KeywordMatch (introduced by~\cite{blindbox}, to which we refer the reader for details) and more extensively discuss PrefixMatch, a new cryptographic scheme we designed for this setting.
When describing these schemes we refer to the encryptor as the gateway whose secret key is $k$ and to the entity computing on the encrypted data as the service provider (SP).


\subsection{KeywordMatch}\label{s:kwmatch}


KeywordMatch detects if an encrypted keyword matches an encrypted data item by equality.
For example, given an encryption of the keyword ``malicious'', and a list of encrypted strings  [$\Enc$(``alice''), $\Enc$(``malicious''), $\Enc$(``alice'')], SP can  detect that the keyword matches the second string. 
For  this, we use a searchable encryption scheme~\cite{song:search, blindbox}.
Using this scheme, the gateway can encrypt a value $v$ into $\enc(v)$ and a rule $r$ into $\encr$ and SP can detect if there is a match between $v$ and $r$. 
 The security of searchable encryption is well studied~\cite{song:search, blindbox}: at a high level,  given a list of encrypted strings, and an encrypted keyword, SP does not learn anything about the encrypted strings, other than which strings match the keyword. 
 The encryption of the strings is {\em randomized}, so it does not leak whether two encrypted strings are equal to each other, unless, of course, they both match the encrypted keyword. 
  We use the technique from~\cite{blindbox} and hence do not elaborate on it.

%!TEX root = mb.tex

%\section{Building block: Range match}
%An important operation over an encrypted packet is to determine if an encrypted field in the packet falls in an encrypted range.
%We will use the firewall as an example. 
%Consider the following firewall rule:
%
%Constructing an encryption scheme that allows checking if an encrypted value is in an encrypted range, has been a challenge in the applied cryptography community. The reason is that ..
%
%\begin{itemize}
%\item preserve the order between Encryptd values
%\item candidate: OPE
%\item candidate: mOPE
%\item So none of the existing schemes are satisfactory. A new scheme \RM.
%\end{itemize}
%
%\RM applies to cases when we know an upper bound on the values encrypted with OPE and this is a small number of values (say, less than 10,000).
%
%The small number of values permits us to improve in two ways over mOPE [1]
%No more interaction. We store the tree in mOPE on the gateway (client) side, which means that the gateway can compute a new encryption by itself without help from service provider. The storage at the gateway will remain small.
%Rare updates to ciphertexts. We can space out the ciphertexts of the values encrypted sufficiently. 
%
%This also enjoys a stronger security than OPE! It leaks less than order.
%The reason is that, the server does not learn the order between the values in packets, and only whether they map between two values in the rules. 
%
%this one is new
%
%discuss 
%
%would be good to explain the challenge from the 
%
%\todo{a more interesting name to the scheme}
%
%prefix gets mapped into interval, at most a certain number
%
%talk about building certain data structures that all works the same
%
%firewall need not change 

\subsection{PrefixMatch} \label{sec:range}


Many middleboxes perform detection over {\it prefixes} or {\it ranges} of  IP addresses or port numbers  (i.e. packet classification). To illustrate PrefixMatch, we use IP addresses (IPv6), but the scheme works with 
ports and other value domains too.
For example, a network administrator might wish to block access to all servers hosted by MIT, in which case the administrator would block access to the prefix 0::ffff:18.0.0.0/104, \ie{}, 0::ffff:18.0.0.0/104--0::ffff:18.255.255.255/104. 
PrefixMatch enables a middlebox to tell whether an encrypted IP address $v$ lies in an encrypted range $[s_1, e_1]$, where $s_1$ = 0::ffff:18.0.0.0/104 and $e_1$ = 0::ffff:18.255.255.255/104; however, the middlebox does not learn the values of $v$, $s_1$, or $e_1$. Unlike KeywordMatch, PrefixMatch is a new scheme we designed specifically for our setting.


One might ask whether PrefixMatch is necessary, or one can instead employ KeywordMatch using the same expansion technique we used for some (but not all) regexps in \S\ref{sec:bbarch}. 
To detect whether an IP address is in a range, one could enumerate all IP addresses in that range and perform an equality check. However, the overhead of using this technique for common network ranges such as firewall rules is prohibitive.
For our own department network, doing so would convert our IPv6 and IPv4 firewall rule set of only 97 range-based rules to $2^{238}$ exact-match  rules; looking only at IPv4 rules would still lead to 38M exact-match rules.
Hence, for efficiency, we need a scheme for matching ranges.

\mypara{Requirements}
%
Supporting the middleboxes from Table~\ref{tbl:mbreqs} and meeting our system security and performance requirements entail the following requirements in designing PrefixMatch.
%
First, PrefixMatch must allow for direct order comparison (i.e., using $\leq$/$\geq$) between an encrypted value $\Enc(v)$ and the encrypted endpoints $\ov{s_1}$ and $\ov{e_1}$ of a range, $[s_1, e_1]$. This allows existing packet classification algorithms, such as tries, area-based quadtrees, FIS-trees, or hardware-based algorithms~\cite{packet_classif}, 
 to run unchanged. 

Second, to support the functionality of NAT in Table~\ref{tbl:mbreqs}, $\Enc(v)$ should be {\em deterministic} within a flow. Recall that a flow is a 5-tuple of source IP and port, destination IP and port, and protocol. Moreover, the encryption corresponding to two pairs (IP$_1$, port$_1$) and  (IP$_2$, port$_2$) that are different should be different.

Third, for security, we require that nothing leaks about the values $v$ encrypted other than what is needed by the functionality above. Note that \sys does not need order comparison between $\Enc(v_1)$ and $\Enc(v_2)$; hence, PrefixMatch does not leak such order information.  PrefixMatch also provides protection for the endpoints of ranges: SP should not learn their values, and SP should not learn the ordering of the intervals. 
 Further, note that the NAT does not require that $\Enc(v)$ be deterministic across flows; hence, PrefixMatch hides whether two IP addresses encrypted as part of different flows are equal or not. In other words, PrefixMatch is randomized across flows.
 
Finally, both encryption (performed at the gateway) and detection (performed at the middlebox) should be practical for typical middlebox line rates.
Our PrefixMatch encrypts in $< 0.5\mu$s per value (we evaluate in \S\ref{sec:eval}), and the detection is the same as regular middleboxes based on the $\leq$/$\geq$ operators.

 %To the best of our knowledge, PrefixMatch is the only encryption scheme that achieves these security guarantees and is practical for our setting.
 Order-preserving encryption such as BCLO~\cite{boldyreva:ope} or mOPE~\cite{popa:mope} can provide the same and more functionality than PrefixMatch. In comparison, PrefixMatch provides significantly stronger security by not leaking the order of values encrypted, and achieves four orders of magnitude higher performance  (as we show in \S\ref{sec:eval}).






\mypara{Functionality}
PrefixMatch encrypts a set of ranges or prefixes $P_1, \dots, P_n$ into a set of encrypted prefixes. For each $P_i$, there is one or more corresponding encrypted prefixes: $\ov{P_{i,1}} \dots, \ov{P_{i, n_i}}$.  Additionally, \pmatch{} encrypts a value $v$ into an encrypted value $\Enc(v)$. These encryptions have the  property that, for all $i$,  
%
\[  v \in P_i  \Leftrightarrow \Enc(v) \in \ov{P_{i, 1}} \cup \dots \cup \ov{P_{i, n_i}}. \]
%
In other words, the encryption preserves prefix matching.



For example, suppose that encrypting $P$ =  0::ffff:18.0.0.0/104 results in one encrypted prefix $\ov{P}$ = 1234::/16, encrypting $v_1$ = 0::ffff:18.0.0.2 results in  
 $\ov{v_1}$ = 1234:db80:85a3:0:0:8a2e:37a0:7334, and encrypting $v_2$ = 0::ffff:19.0.0.1 results in 
$\ov{v_2}$ = dc2a:108f:1e16:992e:a53b:43a3:00bb:d2c2. We can see that $\ov{v_1} \in \ov{P}$ and $\ov{v_2} \notin \ov{P}$. 




\subsubsection{Scheme} 
\label{sec:rmscheme}

\begin{figure}[t]
  \centering
  \includegraphics[width=3in]{fig/rangeopts3.pdf}
  \caption[]{Example of prefix encryption with PrefixMatch.\label{fig:rangeopts3}}
\end{figure}

We now explain how PrefixMatch works.  PrefixMatch provides two algorithms: EncryptPrefixes to encrypt prefixes/ranges and EncryptValue to encrypt a value $v$.



\mypara{Prefixes Encryption} \pmatch{} takes as input a set of prefixes or ranges $P_1 = [s_1, e_1], \dots, P_n=[s_n, e_n]$, 
whose endpoints have size $\len$ bits. \pmatch{} encrypts each prefix  into a set of encrypted prefixes: these prefixes are $\plen$ bits long. As we discuss below, the choice of $\plen$ depends on the maximum number of prefixes to be encrypted. For example,  $\plen = 16$ suffices for a typical firewall rule set.

Consider all the endpoints $s_i$ and $e_i$ laid out on an axis in increasing order as in Fig.~\ref{fig:rangeopts3}.
Add on this axis the endpoints of $P_0$, the smallest and largest possible values, $0$ and $2^{\ptlen}-1$.
Consider all the non-overlapping intervals formed by each consecutive pair of such endpoints. 
For example, in Fig.~\ref{fig:rangeopts3},  there are two prefixes to encrypt: $P_1$ and $P_2$. PrefixMatch computes the intervals $I_0, \dots, I_4$.

PrefixMatch now assigns an encrypted prefix to each interval. Note that each interval belongs to a set of prefixes. Let $\prefixes(I)$ denote the prefix of interval $I$. For example, $\prefixes(I_2) = \{P_0, P_1, P_2\}$. The encrypted prefix is simply a {\em random} number of size $\plen$. Each interval gets a different random value, except for intervals that belong to the same prefixes. For example, in Fig.~\ref{fig:rangeopts3}, intervals $I_0$ and $I_4$ receive the same random number because $\prefixes(I_0) = \prefixes(I_4)$.

When a prefix overlaps partially with another prefix, it will have more than one encrypted prefix because it is broken into intervals. For example, $I_1$ was assigned a random number of 0x123c and $I_2$ of 0xabcc. The encryption of  $P_1$ in Fig.~\ref{fig:rangeopts3} will be the pair ($123c::/16$, $abcc::/16$).

Since the encryption is a random prefix, the encryption does not reveal the original prefix. Moreover, the fact that intervals pertaining to the same set of prefixes receive the same encrypted number hides where an encrypted value matches, as we discuss below. For example, for an IP address $v$ that does not match either $P_1$ or $P_2$, the cloud provider will not learn whether it matches to the left or to the right of $P_1 \cup P_2$ because $I_0$ and $I_4$ receive the same encryption. The only information it learns about $v$ is that $v$ does not match either $P_1$ or $P_2$. 

We now present the EncryptPrefixes procedure, which works the same for prefixes or ranges.



\begin{framed}
\ALGORITHM{EncryptPrefixes}{proc:enc_prefixes}{$P_1$, $\dots$, $P_n$, $\plen$, $\ptlen$}{
\item Let $s_i$ and $e_i$ be the endpoints of $P_i$. \comment{$P_i = [s_i, e_i]$}
	\item Assign $P_0 \gets [0, 2^{\ptlen}-1]$
	\item Sort all endpoints in $\cup_i P_i$ in increasing order
	\item  Construct non-overlapping intervals $I_0, \dots, I_m$ from each pair of consecutive endpoints. Multiple equal endpoints form a unit interval. For each interval $I_i$, compute $\prefixes(I_i)$, the list of prefixes $P_{1}, \dots, P_{m_i}$ that contain   $I_i$. 
	\item Assign a distinct random value of size $\plen$ to each of $\ov{I_0}, \dots, \ov{I_m}$  
	\item For all $i, j$ with $i<j$ if  $\prefixes(I_i) =  \prefixes(I_j)$, set $\ov{I_j} \gets \ov{I_i}$
	\item The encryption of $P_i$ is $\ov{P_i} = \{\ov{I_j}/\plen, \text{for all}\ I_j \in \prefixes(P_i) \}$. The encrypted prefixes are output sorted by value (as a means of randomization).
	\item Output $\ov{P_1}, \dots,\ \ov{P_n}$ and the {\em interval map} [$I_i \rightarrow \ov{I_i}$] 
}
\end{framed}

\mypara{Value Encryption}
To encrypt a value $v$, PrefixMatch locates the one interval $I$ such that $v \in I$. It then looks up $\ov{I}$ in the interval map computed by EncryptPrefixes and sets $\ov{I}$ to be the prefix of the encryption of $v$. This ensures that the encrypted $v$, $\ov{v}$, matches $\ov{I}/\plen$. The suffix of $v$ is chosen at random. The only requirement is that it is deterministic. 
Hence, the suffix is chosen based on a pseudorandom function~\cite{goldreichvolume1}, $\prf^{\slen}$, seeded in a given seed  $\seed$, where $\slen = \len-\plen$.  As we discuss below, the seed used by the gateway depends on the 5-tuple of a connection (SIP, SP, DIP, DP, P).



For example, if $v$ is 0::ffff:127.0.0.1, and the assigned prefix for the matched interval is $abcd::/16$, a possible encryption given the ranges encrypted above is $\Enc(v) = abcd:ef01:2345:6789:abcd:ef01:2345:6789$. Note that the encryption does not retain any information about $v$ other than the interval it matches in because the suffix is chosen (pseudo)randomly. In particular, given two values $v_1$ and $v_2$ that match the same interval, the order of their encryptions is arbitrary. Thus, PrefixMatch does not reveal order.

\begin{framed}
\ALGORITHM{EncryptValue}{proc:enc_value}{$\seed$, $v$, $\slen$, interval map}{
\item Run binary search on interval map to locate the interval $I$ such that $v \in I$.
\item Lookup $\ov{I}$ in the interval map.
\item Output 
\begin{equation}
\Enc(v) = \ov{I} \Vert \prf_\seed^\slen(v)
\end{equation}
}
\end{framed}


%\begin{figure}
%  \includegraphics[width=3.25in]{fig/tree}
%  \caption{\label{fig:tree} PrefixMatch tree. The values of nodes in the tree are the %unencrypted IP addresses, and the blue values on the horizontal axis are their encryptions. }
%\end{figure}

\mypara{Comparing encrypted values against rules}
Determining if an encrypted value matches an encrypted prefix is straightforward: the encryption preserves the prefix and a middlebox can use the regular $\le$/$\ge$ operators.
Hence, a regular packet classification can be run at the firewall with no modification. Comparing different encrypted values that match the same prefix is meaningless, and returns a random value.



\subsubsection{Security Guarantees}
\label{sec:buildingblocks:guarantees}

\pmatch{} hides the  prefixes and values encrypted with EncryptPrefixes and EncryptValue. \pmatch{} reveals  matching information desired to enable functionality at the cloud provider.  Concretely, the cloud provider learns the number of intervals and which prefixes overlap in each interval, but no additional information on the size, order or endpoints of these intervals. Moreover, for every encrypted value $v$, it learns the indexes of the prefixes that contain $v$ (which is the functionality desired of the scheme), but no other information about $v$. For any two encrypted values $\Enc(v)$ and $\Enc(v')$, the cloud provider learns if they are equal only if they are encrypted as part of the same flow (which is the functionality desired for the NAT), but it does not learn any other information about their value or  order.  
Hence, PrefixMatch leaks less information than OPE, which reveals the order of encrypted prefixes/ranges.

%
Since EncryptValue is seeded in a per-connection identifier, correlation attacks between different flows are not possible. 
In particular, even though EncryptValue is deterministic within a flow, it is randomized across flows:
 for example, the encryption of the same IP address in different flows is different because the seed differs per flow. 


We formalize and prove the security guarantees of \pmatch{} in our extended paper. 


