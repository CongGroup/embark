%!TEX root = mb.tex
\newpage
\appendix
\section{Required Properties of PrefixMatch} 
\label{sec:appendix:middleboxes}
In this section, we discuss the core functionality of IP Firewall, NAT, L3/L4 Load Balancers 
in Table~\ref{tbl:mbreqs}, and how do we derive the requirements for the encryption scheme. We 
omit the discussion of other middleboxes in the table since their requirements are obvious to obtain.

\subsection{IP Firewall}
Firewalls from different vendors may have significantly different configurations and rule organizations, 
and thus we need to extract a general model of firewalls. We used the model defined in ~\cite{fireman}, 
which describes Cisco
PIX firewalls and Linux iptables. In this model, the firewall consists of several access control lists (ACLs). 
Each ACL consists of a list of rule. Rules can be interpreted in the form (\emph{predicate}, \emph{action}), 
where the \emph{predicate} describes the packets matching this rule and the \emph{action} describes
the action performed on the matched packets. The predicate is defined as a combination of ranges of source/destination
IP addresses and ports as well as the protocol. The set of possible actions includes "\emph{accept}" and "\emph{deny}".

Now we discuss the encryption. Let $\Enc$ denote a generic encryption protocol,
$(SIP[],\ DIP[],\ SP[],\ DP[],\ P)$ denote the predicate of a rule. Any packet with 5-tuple 
$(SIP,\ DIP,\ SP,\ DP,\ P) \in (SIP[],\ DIP[],\ SP[],\ DP[],\ P)$ matches that rule. 
We encrypt both tuples and rules. In order to preserve the functionality of the firewall, 
we should ensure that the following requirement holds after the encryption.

\begin{equation}
\label{eqn:firewall}
\begin{split}
& (SIP,\ DIP,\ SP,\ DP,\ P) \in (SIP[],\ DIP[],\ SP[],\ DP[],\ P) \Leftrightarrow \\
& \Enc(SIP,\ DIP,\ SP,\ DP,\ P) \in \Enc(SIP[],\ DIP[],\ SP[],\ DP[],\ P) 
\end{split}
\end{equation}

\subsection{NAT}
A typical NAT translates a pair of source IP and port into a pair of
external source IP and port (and back), where the external source IP is the external address of the gateway,
and the external source port is arbitrarily chosen. Essentially, a NAT maintains a mapping from a pair of source IP
and port to an external port. There are two requirements: 1) same pair should be mapped to the same external source
port; 2) different pairs should not be mapped to the same external source port. In order to satisfy the two
requirements, the encryption scheme should ensure that the following requirements hold.

\begin{equation}
\begin{split}
& (SIP_1, SP_1)\ =\ (SIP_2, SP_2) \\ 
 \Rightarrow & \Enc(SIP_1, SP_1) = \Enc(SIP_2, SP_2) 
\end{split}
\end{equation}

\begin{equation}
\begin{split}
& \Enc(SIP_1, SP_1) = \Enc(SIP_2, SP_2) \\
& \Rightarrow (SIP_1, SP_1) = (SIP_2, SP_2) 
\end{split}
\end{equation}

However, we may relax 1) to 1*): the source IP and port pair that belongs to the same 5-tuple 
should be mapped to the same external port. We will explain as we get to the
formal definition of PrefixMatch. After relaxing this requirement, the functionality of NAT is still preserved,
but the NAT table may get filled up more quickly since the same pair may be mapped to different ports. However,
we argue that this may not happen in practice, as an application on a host rarely connects to different hosts 
or ports using the same source port. The requirements then become:

\begin{equation}
\label{eqn:nat1}
\begin{split}
& (SIP_1, DIP_1, SP_1, DP_1, P_1)\ =\ (SIP_2, DIP_2, SP_2, DP_2, P_2) \\
\Rightarrow & \Enc(SIP_1, SP_1) = \Enc(SIP_2, SP_2) \\
\end{split}
\end{equation}

\begin{equation}
\label{eqn:nat2}
\begin{split}
& \Enc(SIP_1, SP_1) = \Enc(SIP_2, SP_2) \\
& \Rightarrow (SIP_1, SP_1) = (SIP_2, SP_2) 
\end{split}
\end{equation}

\subsection{L3 Load Balancer}
L3 Load Balancer maintains a pool of servers. It chooses a server for an incoming packet based on
L3 connection information. A common implementation of L3 Load Balancing uses the ECMP scheme in the 
switch. It guarantees that packets of the same flow will be forwarded to the same server by hashing
the 5-tuple. Therefore, the requirement is

\begin{equation}
\label{eqn:l3lb}
\begin{split}
& (SIP_1, DIP_1, SP_1, DP_1, P_1)\ =\ (SIP_2, DIP_2, SP_2, DP_2, P_2) \Rightarrow \\
& \Enc(SIP_1, DIP_1, SP_1, DP_1, P_1)\ =\ \Enc(SIP_2, DIP_2, SP_2, DP_2, P_2)
\end{split}
\end{equation}

\subsection{L4 Load Balancer}
L4 Load Balancer~\cite{haproxy}, or TCP Load Balancer also maintains a pool of servers.
It acts as a TCP endpoint that accepts the client's connection. After accepting a connection from a client,
it connects to one of the server and forwards the bytestreams between client and server. The encryption
scheme should make sure that same 5-tuples have the same encryption. In addition, two different 5-tuple
should not have the same encryption, otherwise the L4 Load Balancer cannot distinguish those two flows.
Thus, the requirement of supporting L4 Load Balancer is:

\begin{equation}
\label{eqn:l4lb}
\begin{split}
& (SIP_1, DIP_1, SP_1, DP_1, P_1)\ =\ (SIP_2, DIP_2, SP_2, DP_2, P_2) \Leftrightarrow \\
& \Enc(SIP_1, DIP_1, SP_1, DP_1, P_1)\ =\ \Enc(SIP_2, DIP_2, SP_2, DP_2, P_2)
\end{split}
\end{equation}

\section{Formal Properties of PrefixMatch}
In this section, we show how PrefixMatch supports middleboxes discussed in \S\ref{sec:appendix:middleboxes}.
First of all, we formally list the properties that PrefixMatch preserves. As discussed in \label{sec:range},
PrefixMatch preserves the functionality of firewalls by guaranteeing Property \ref{eqn:firewall}:

In addition, PrefixMatch also ensures the following properties:

\begin{equation}
\label{eqn:pm1}
\begin{split}
& (SIP_1, DIP_1, SP_1, DP_1, P_1)\ =\ (SIP_2, DIP_2, SP_2, DP_2, P_2) \Rightarrow \\
& \Enc(SIP_1, DIP_1, SP_1, DP_1, P_1)\ =\ \Enc(SIP_2, DIP_2, SP_2, DP_2, P_2)
\end{split}
\end{equation}

\begin{equation}
\label{eqn:pm2}
\Enc(SIP_1) = \Enc(SIP_2)\ \Rightarrow\ SIP_1 = SIP_2
\end{equation}

\begin{equation}
\label{eqn:pm3}
\Enc(DIP_1) = \Enc(DIP_2)\ \Rightarrow\ DIP_1 = DIP_2
\end{equation}

\begin{equation}
\label{eqn:pm4}
\Enc(SP_1) = \Enc(SP_2)\ \Rightarrow\ SP_1 = SP_2
\end{equation}

\begin{equation}
\label{eqn:pm5}
\Enc(DP_1) = \Enc(DP_2)\ \Rightarrow\ DP_1 = DP_2
\end{equation}

\begin{equation}
\label{eqn:pm6}
\Enc(P_1) = \Enc(P_2)\ \Rightarrow\ P_1 = P_2
\end{equation}

We discuss how those properties imply all the requirements in \S\ref{sec:appendix:middleboxes} as follows.

\noindent{\bf NAT} 
We will show that Eq.\eqref{eqn:pm1}-Eq.\eqref{eqn:pm6} imply Eq.\eqref{eqn:nat1}-
Eq.\eqref{eqn:nat2}. Given $(SIP_1, DIP_1, SP_1, DP_1, P_1)\ =\ (SIP_2, DIP_2, SP_2, DP_2, P_2)$, 
by Eq.~\eqref{eqn:pm1},
we have $\Enc(SIP_1, SP_1) = \Enc(SIP_2, SP_2)$. Hence, Eq.\eqref{eqn:nat1} holds. Similarly, 
given $\Enc(SIP_1, SP_1) = \Enc(SIP_2, SP_2)$, by Eq.\eqref{eqn:pm2} and Eq.\eqref{eqn:pm4}, 
we have $(SIP_1, SP_1) = (SIP_2, SP_2)$. Hence, Eq.\eqref{eqn:nat2} also holds. Note that if
we did not relax the requirement in Eq.\eqref{eqn:nat1}, we could not obtain such proof.

\noindent{\bf L3 Load Balancer} 
By Eq.\eqref{eqn:pm1}, Eq.\eqref{eqn:l3lb} holds.

\noindent{\bf L4 Load Balancer} 
By Eq.\eqref{eqn:pm1}, the left to right direction of 
Eq.\eqref{eqn:l4lb} holds. By Eq.\eqref{eqn:pm2}-Eq.\eqref{eqn:pm6}, the right to left 
direction of Eq.\eqref{eqn:l4lb} also holds.