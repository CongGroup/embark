%!TEX root = mb.tex


\section{Middleboxes: Design \& Implementation}
\label{sec:mbs}

 \sys supports the core functionality of a set of middleboxes as listed in Table~\ref{tbl:mbreqs}.
Table~\ref{tbl:mbreqs} also lists the functionality supported by \sys.   In Appendix~\ref{sec:appendix:middleboxes}, we review the core functionality of each middlebox
  and explain why the functionality in Table~\ref{tbl:mbreqs} is sufficient to support these middleboxes. 
 In this section, we focus on implementation aspects of the middleboxes.



\subsection{Header Middleboxes}
Middleboxes which operate on IP and transport headers only include firewalls, NATs, and L3/L4 load balancers.
Firewalls are read-only, but NATs and L4 load balancers may rewrite IP addresses or port values. 
For header middleboxes, per-packet operations remain unchanged for both read and write operations.

For read operations, the firewall receives a set of encrypted rules from the gateway and compares them directly against the encrypted packets just as normal traffic. Because PrefixMatch supports $\leq$ and $\geq$, the firewall may use any of the standard classification algorithms~\cite{packet_classif}.




For write operations, the middleboxes assign values from an address pool; it receives these encrypted pool values from the gateway during the rule generation phase.
These encrypted rules are marked with a special suffix reserved for rewritten values.
When the gateway receives a packet with such a rewritten value, it restores the plaintext value from the pool rather than decrypting the value from the options header. 

Middleboxes can recompute checksums as usual after they write.

\subsection{DPI Middleboxes}
We modify middleboxes which perform DPI operations as in BlindBox~\cite{blindbox}. 
The middleboxes search through the encrypted extension channel -- not the packet payloads themselves -- and block or log the connection if a blacklisted term is observed in the extension.
We also improve setup time and security for regular expression rules as we discussed in \S\ref{s:archbb}.
%Much of the BlindBox paper is devoted to designing new fast-path algorithms which provide high throughput and cryptographic compatibility; as we showed \sys sidesteps this problem for header middleboxes since it can re-use existing algorithms.

\subsection{HTTP Middleboxes}
Parental filters and HTTP proxies read the HTTP URI from the extension channel. 
If the parental filter observes a blacklisted URI, it drops packets that belong to the connection.
%The L7 load balancer uses the URI to rewrite the IP header, and operates exactly like a NAT or L4 LB, except that it uses the URI to select which destination address to assign new connections.

The web proxy required the most modification of any middlebox \sys supports; nonetheless, our proxy achieves good performance as we will discuss in \S\ref{sec:eval}.
The proxy caches HTTP static content (e.g., images) in order to improve client-side performance. 
When a client opens a new HTTP connection, a typical proxy will capture the client's SYN packet and open a new connection to the client, as if the proxy were the web server. The proxy then opens a second connection in the background to the original web server, as if it were the client. 
When a client sends a request for new content, if the content is in the proxy's cache, the proxy will serve it from there. Otherwise, the proxy will forward this request to the web server and cache the new content. 

The proxy has a map of encrypted file path to encrypted file content. When the proxy accepts a new TCP connection on port 80, the proxy extracts the encrypted URI for that connection from the extension channel and looks it up in the cache. The use of deterministic encryption enables the proxy to use a fast search data structure/index, such as a hash map, unchanged. We have two possible cases: there is a hit or a miss. If there is a cache hit, the proxy sends the encrypted file content from the cache via the existing TCP connection. Even without being able to decrypt IP addresses or ports, the proxy can still accept the connection, as the gateway encrypts/decrypts the header fields transparently.
If there is a cache miss, the proxy opens a new connection and forwards the encrypted request to the web server. Recall that the traffic bounces back to gateway before being forwarded to the web server, so that the gateway can decrypt the header fields and payloads. Conversely, the response packets from the web server are encrypted by the gateway and received by the proxy. The proxy then caches and sends the encrypted content back. The content is separated by packets. Packet payloads are encrypted on a per-packet basis. Hence, the gateway can decrypt them correctly.


\subsection{Limitations}\label{s:limitations}

 \sys supports the core functionality of a wide-range of  middleboxes, as listed in Table~\ref{tbl:mbreqs},
but not all middlebox functionality one could envision outsourcing. We now discuss some examples.
%
First, for intrusion detection, \sys does not support regular expressions that cannot be expanded 
in a certain number of keyword matches, running arbitrary scripts on the traffic~\cite{bro}, or advanced statistical techniques that correlate different flows
studied in the research literature~\cite{steppingstones}.

Second, \sys does not support application-level middleboxes, such as SMTP firewalls, application-level gateways or transcoders.
These middleboxes parse the traffic in an application-specific way -- such parsing is not supported by KeywordMatch.
%For example, application-level firewalls such as the SMTP firewall parse the content received into SMTP format. Application-level gateways are used for  certain address-dependent 
%protocols such as FTP, and carry 
%out mechanical substitution of fields based on a certain parsing of the stream.
%Similarly, \sys does not support transcoders, which perform on-the-fly conversion of application level 
%data.
%
Third, \sys does not support port scanning because the encryption of a port depends on the associated IP address. 
Supporting all these functionalities is part of our future work.


