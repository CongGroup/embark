%!TEX root = mb.tex

\section{Enterprise Gateway}

\label{sec:gateway}

\begin{figure}[t]
  \centering
  \includegraphics[width=3.25in]{fig/gateway2cloud}
  \caption[]{\label{fig:gatewaymeta} Communication between the cloud and gateway services: rule encryption, data encryption, and data decryption.} 
\end{figure}


%\eat{
%\begin{figure}[t]
%  \centering
%  \includegraphics[width=3.25in]{fig/gatewaydiag}
%  \caption[]{\label{fig:gateway} Data encryption (enterprise to cloud) click module.}
%\end{figure}
%}




The gateway serves two purposes. First, it redirects traffic to/from the cloud for middlebox processing. Second, it provides the cloud with encryptions of rulesets and updates to the PrefixMatch tree.
Every gateway is configured statically to tunnel traffic to a fixed IP address at a single service provider point of presence.
A gateway can be logically thought of as three services: the rule encryption service, the pipeline from the enterprise to the cloud (Data encryption), and the pipeline from the cloud to the enterprise (Data decryption). 
All three services share access to the PrefixMatch tree and the private key $k$.
Figure~\ref{fig:gatewaymeta} illustrates  these three services and the data they send to and from the cloud provider.

We design the gateway with two goals in mind: 

\noindent{\bf Format-compatibility}: in converting plaintext traffic to encrypted traffic, the encrypted data should be structured in such a way that the traffic {\it appears as normal IPv6 traffic} to middleboxes performing the processing. Format-compatibility allows us to leave fast-path operations unmodified not only in middlebox software, but also in hardware components like NICs and switches; this results in good performance at the cloud.

\noindent{\bf Scalability and Low Complexity}: the gateway should perform only inexpensive per-packet operations and should be parallelizable. The gateway should not require configuration other than to generate a key and establish a session with the cloud provider. If the gateway were as expensive and complex as the middleboxes, there would be no cost or management benefits from outsourcing. 

%We now discuss how the gateway performs encryption and decryption (\S\ref{sec:dataenc}) and how rules are encrypted (\S\ref{sec:rulenc}) with these design goals in mind.

\subsection{Data Encryption and Decryption}
\label{sec:dataenc}

As shown in Table~\ref{tbl:mbreqs}, we categorize middleboxes as Header middleboxes, which operate only on IP and transport headers; DPI middleboxes, which operate on arbitrary fields in a connection bytestream; and HTTP middleboxes, which operate on values in HTTP headers (these are a subclass of DPI middleboxes). We discuss how each category of data is encrypted/decrypted in order to meet middlebox requirements as follows.

\subsubsection{IP and Transport Headers}
IP and Transport Headers are encrypted field by field (\eg{}, a source address in an input packet results in an encrypted source address field in the output packet) with PrefixMatch.
We use PrefixMatch for these fields because many middleboxes perform analysis over prefixes and ranges of values -- e.g., a firewall may block all connections from a restricted IP prefix.


To encrypt a value with PrefixMatch's EncryptValue, the gateway seeds the encryption with  $\seed = \prf_k(SIP,\ SP,\ DIP,\ DP,\ P)$, a function of both the key and connection information using the notation in Table~\ref{tbl:mbreqs}. Note that in the system setup with two gateways, the gateways generate the same encryption because they share $k$.  


When encrypting IP addresses,  two different IP addresses must not map to the same encryption because this breaks the NAT. To avoid this problem, encrypted IP addresses in \sys must be IPv6 because the probability that two IP addresses get assigned to the same encryption is negligibly low. The reason is that each encrypted prefix contains a large number of possible IP addresses. Suppose we have $n$ distinct firewall rules, $m$ flows and a $\len$-bit space, the probability of a collision is approximately: 

\todo{is this up to date? n is number of rules, but intervals is what matters; there are at most 2n+1 intervals}
\begin{equation}
1 - e^\frac{-m^2 (2n+1)}{2^{\len+1}}
\end{equation}
% for n firewall rules, there are at most 2n +1 intervals


Therefore, if $\len=128$ (which is the case when we use IPv6), the probability is negligible in a normal setting. 

When encrypting ports, it is possible to get collisions since the port field is only 16-bit. However, this will not break the NAT's functionality as long as the IP address does not collide, because NATs (and other middleboxes that require injectivity) considers both IP addresses and ports. For example, if we have two flows with source IP and source ports of $(SIP, SP_1)$ and $(SIP, SP_2)$ with $SP_1 \neq SP_2$, the encryption of SIP will be different in the two flows because the encryption is seeded in the 5-tuple of a connection. As we discuss in Appendix~\ref{sec:appendix:middleboxes}, the NAT table can be larger for \sys, but the factor is small in practice.




\mypara{Decryption}
PrefixMatch is not reversible. To enable packet decryption, we store the AES-encrypted values for the header fields in the IPv6 options header. 
When the gateway receives a packet to decrypt, if the values haven't been rewritten by the middlebox (\eg, NAT),
it decrypts the values from the options header and restores them. 
% to the IP or transport header.

\mypara{Format-compatibility} 
Our modifications to the IP and transport headers place the encrypted prefix match data back in to the same fields as the unencrypted data was originally stored; because comparisons between rules and encrypted data rely on $\leq$ $\geq$, just as unencrypted data, this means that operations performing comparisons on IP and transport headers {\it remain entirely unchanged at the middlebox.}
This ensures backwards-compatibility with existing software {\it and hardware} fast-path operations.
Because per-packet operations are tightly optimized in production middleboxes, this compatibility ensures good performance at the cloud despite our changes.

An additional challenge for format compatibility is where to place the decryptable AES data; one option would be to define our own packet format, but this could potentially lead to incompatibilities with existing implementations. By placing it in the IPv6 options header, middleboxes can be configured to ignore this data.\footnote{It is a common misconception that middleboxes are incompatible with IP options. Commercial middleboxes are usually aware of IP options but many administrators {\it configure} the devices to filter or drop packets with certain kinds of options enabled.}


\subsubsection{Payload Data} 
The connection bytestream is encrypted with KeywordMatch.
This allows \sys to support DPI middleboxes, such as intrusion detection or exfiltration prevention.
These devices must detect whether or not there exists an exact match for an encrypted rule string {\it anywhere} in the connection bytestream.
Because this encrypted payload data is over the {\it bytestream}, we need to generate encrypted values which span `between' packet payloads. 
Searchable Encryption schemes, which we use for encrypted DPI, require that traffic be {\it tokenized} and that a set of fixed length substrings of traffic be encrypted along a sliding window -- e.g., the word malicious might be tokenized into {`malici', `alicio', `liciou', `icious'}.
If the term `malicious' is divided across two packets, we may not be able to tokenize it properly unless we reconstruct the TCP bytestream at the gateway. Hence, if DPI is enabled at the cloud, we do exactly this.

After reconstructing and encrypting the TCP bytestream, the gateway transmits the encrypted bytestream over 
an `extension', secondary channel that only those middleboxes which perform DPI operations inspect. 
This channel is not routed to other middleboxes. We implement this channel as a persistent TCP connection 
between the gateway and middleboxes. The bytestream in transmission is associated with its flow identifier, 
so that the DPI middleboxes can distinguish between bytestreams in different flows.
DPI middleboxes handle both the packets received from the extension channel as well as the primary channel containing the data packets. 
Hence, if an intrusion prevention system finds a signature in the extension channel, it can sever or reset connectivity for the primary channel.

\noindent{\bf Decryption.} The payload data is encrypted with AES and placed back into the packet payload -- like PrefixMatch, KeywordMatch is not reversible and we require this data for decryption at the gateway.
Because the extension channel is not necessary for decryption, it is not transmitted back to the gateway.

\noindent{\bf Format-compatibility.} To middleboxes which only inspect/modify packet headers, encrypting payloads has no impact. 
By placing the encrypted bytestreams in the extension channel, the extra traffic can be routed past and ignored by middleboxes which do not need this data. % hence it will not interfere with normal processing. 

DPI middleboxes which do inspect payloads must be modified to inspect the extension channel alongside the primary channel, as described in~\cite{blindbox}; DPI devices are typically implemented in software and and these modifications are both straightforward and introduce limited overhead (as we will see in \S\ref{sec:eval}). 

\subsubsection{HTTP Headers} 

HTTP Headers are a special case of payload data. Middleboxes such as web proxies do not read arbitrary values from packet payloads: the only values they read are the HTTP headers. They can be categorized as DPI middleboxes since they need to examine the TCP bytesteam. However, due to the limitation of full DPI support, we treat these values specially compared to other payload data: we encrypt the entire (untokenized) HTTP URI using a deterministic form of KeywordMatch.

Normal KeywordMatch permits comparison between encrypted values and rules, but not between one value and another value; deterministic KeywordMatch permits two values to be compared as well.
Although this is a weaker security guarantee relative to KeywordMatch, it is necessary to support web caching which requires comparisons between different URIs.
The cache hence learns the frequency of different URIs, but cannot immediately learn the URI values.
 This is the only field which we encrypt in the weaker setting.
We place this encrypted value in the extension channel; hence our HTTP encryption has the same format-compatibility properties as other DPI devices.

Like other DPI tasks, this requires parsing the entire TCP bytestream. However, in some circumstances we can extract and store the HTTP headers statelessly; so long as HTTP pipelining is disabled and packet MTUs are standard-sized (>1KB), the required fields will always appear contiguously within a single packet. Given that SPDY uses persistent connections and pipelined requests, this stateless approach does not apply to SPDY.
%Hence, if DPI is disabled we can avoid reconstructing the TCP bytestream at the middlebox.
%When DPI is enabled, the gateway scans through every byte of data in the connection already;  extracting these headers allows the middlebox to avoid re-doing work which has already been done at the gateway.

\noindent{\bf Decryption.} The packet is decrypted as normal using the data stored in the payload; IP options are removed.


%\eat{
%JS: I vote to cut this. Here's why -- we are going to be basically hosed when it comes to bandwidth, whether or not we have this. And adding this in adds a huge amount of complexity to the gateway! So way sacrifice our primary goal (cheap, less complex) to only partially fix another problem??
%\subsubsection{Optimizing Decryption}
%To reduce bandwidth usage, the gateway temporarily caches the packets destined to the service provider indexed by their hashes for a short time. If the middleboxes on the cloud do not change the packet content, instead of sending the whole packet back, the service provider sends a control packet containing the hash of the original packet. The gateway then retrieves the local packet by looking up the hash in the cache.
%}

\subsection{Rule Encryption}
\label{sec:rulenc}

\todo{need to say that there is a blowup in the size of the firewall}

Rules for firewalls and DPI services come from a variety of sources and can have different policies regarding who is or isn't allowed to know the rules. 
For example, exfiltration detection rules may include keywords for company products or unreleased projects which the client may wish to keep secret from the cloud provider. 
On the other hand, many DPI rules are proprietary features of DPI vendors, who may allow the provider to learn the rules, but not the client.
\sys supports three different models for KeywordMatch rules which allow clients and providers to share rules as they are comfortable: (a) the client knows the rules, and the provider does not; (b) the provider knows the rule, and the client does not; or (c) both parties know the rules.
PrefixMatch rules only supports (a) and (c) -- the gateway who maintains the rule tree {\it must} know the rules to perform encryption properly.

If the client is permitted to know the rules, they encrypt them -- either generating a KeywordMatch, AES, or PrefixMatch rule -- and send them to the cloud provider. If the cloud provider is permitted to know the rules as well, the client will send these annotated with the plaintext for each encryption; if the cloud provider is not allowed, the client sends only the encrypted rules in random order.

If the client is not permitted to know the rules, we must somehow allow the cloud provider to learn the encryption of each rule with the client's key. This is achieved using a classical combination of Yao's garbled circuits~\cite{Yao86} with oblivious transfer~\cite{Naor-Pinkas}, as originally applied by BlindBox~\cite{blindbox}.
As in BlindBox, this exchange only succeeds if the rules are signed by a trusted third party (such as McAffee, Symantec, or EmergingThreats) -- the cloud provider should not be able to generate their own rules without such a signature as it would allow the cloud provider to read arbitrary data from the clients' traffic.
Unlike BlindBox, this rule exchange occurs exactly once -- when the gateway initializes the rule. 
After this setup, all connections from the enterprise are encrypted with the same key at the gateway.
%This is important because for a typical industry ruleset, a garbled circuit + oblivious transfer exchange takes 97 seconds~\cite{blindbox}; with BlindBox this exchange must be performed for every connection and thus is prohibitive for practicality.
%For \sys, on the otherhand, this amounts to a relatively cheap one-time setup cost.

%\eat{
%\subsection{Discussion}
%\label{sec:gwimpl}
%We built our gateway as described above using BESS~\cite{bess} on an off-the-shelf 16-core server with 2.6GHz Xeon E5-2650 cores and 128GB RAM; the network hardware is a single 10GbE Intel 82599 compatible network card. 
%We discuss a few systems properties of our gateway as follows before discussing middlebox implementations in \S\ref{sec:mbimpl}.
%
%\eat{
%Our gateway datapath implementation has 3 major components: Header Rewrite, Stream Reconstruction, and Packet Cache. We primarily introduce the dataflow from the client to the gateway below:
%
%\noindent\textbf{Header Rewrite.} At this stage, the gateway rewrites the header fields in the packets using the PrefixMatch scheme using in algorithm in Section \ref{sec:dataenc}. It consists of multiple Click elements: ProtocolTranslator46 that translates an IPv4 packet to an IPv6 packet, AppendIP6Option that populates DET-encrypted header fields into the IPv6 options, HeaderEncrypt that performs the encryption, and SetTCPChecksum that sets the checksum.
%
%\noindent\textbf{Stream Reconstruction.} This component acts as a TCP transparent proxy by terminating incoming TCP connections from clients. It encrypts the reconstructed bytestreams using the stream cipher, and relays them in new connections. Note that the IP options are preserved during this phase. It also pushes bytestreams into the secondary channel, where middleboxes can perform DPI operations. Meanwhile, it extracts HTTP headers from the stream and puts the deterministically encrypted HTTP header fields into the IP options.
%
%\noindent\textbf{Packet Cache.} The dataflow from the service provider back to the gateway is relatively simple. The Packet Cache restores the packets, the Stream Reconstruction component decrypts the streams into plaintext, then the Header Rewrite component decrypts the header fields and performs 6to4 translation (if the client uses IPv4).
%
%Having discussed the design and implementation of the gateway, we now revisit a few of its system properties before moving on to the design and implementation of the middleboxes in \S\ref{sec:mbimpl}.
%}
%
%\noindent\textbf{Scalability.}
%Encryption and decryption are entirely parallel: they require no synchronization or communication between threads. We show in \S\ref{sec:eval} that we achieve almost linear improvements from adding additional processing cores.
%However, per-packet operations rely only on relatively inexpensive techniques; the most expensive operation performed is AES encryption with is supported in hardware through the AES-NI instructions. 
%Consequently, the gateway can support 10Gbps of traffic with only 8 cores enabled.
%
%\noindent\textbf{Complexity.}
%An administrator only supplies the gateway with the IP address of the cloud and the secret key $k$.
%The administrator places the gateway at the border of their enterprise network to the Internet.
%The administrator supplies any rules/policies to the cloud provider and the gateway is then configured by the cloud provider.
%
%\noindent\textbf{Fault Tolerance.}
%When DPI is disabled, the gateway keeps no per-connection state. Hence, a cold standby can take over correctly for the gateway so long as it has the `static' state of the PrefixMatch tree, key $k$, and IP address of SP.
%When DPI is enabled, the gateway maintains the last few packets from each connection (window-length) to perform tokenization; under these conditions the gateway can use standard techniques for backup including an active standby~\cite{colo}.
%}



\noindent \textbf{Updating ranges.}
Adding a new range or removing an existing range would affect the intervals it covers. Therefore, in the worst case $O(N)$ rules needs to be updated. 

\todo{ and it has to be clear that to encrypt firewall rules at the gateway using the Prefix match, you need to go through each part of the rule and encrypt that; 
The gateway will use an instance of PrefixMatch for every field, destination IP, etc.
}
\todo{and need to discuss the prefix part}

\todo{need to discuss the fact that they blowup here}


The gateway stores the intervals and the mapping from intervals to prefixes for each field of the rules, and it maintains no state per IP address encrypted or per connection.

The gateway can use the following functions. EncryptRules encrypts the initial rules. EncryptTuple encrypts packet header fields which are represented as a tuple; it requires identifying the interval $I$ for each field. We can compute $I$ efficiently in logarithmic time by doing a binary search. 




% We now describe the procedure for AddRange which adds an interval (deleting an interval is similar). These procedures modify the state at the gateway. To add a range [$s$, $e$], the gateway inserts these values in the tree. If the tree needs to be rebalanced, for each endpoint that changed location in the tree, the gateway records the old and new encrypted value based on the tree in a list $L$. Note that  the number of endpoints that change encryption is $O(\log n)$ amortized worst-case. The gateway then computes the encryption of $s$ and $e$ based on their location in the tree. It sends to SP $\enc(s)$ and $\enc(e)$ along with $L$. 

%Besides the interval added or deleted, a small number
%of other intervals may be moved -- at worst, $O(\log n)$. 
%For these, the algorithm returns the old and new encryption of the interval. 

%
%\begin{framed}
%\begin{algorithmic}[1]
%
%\Procedure{AddRange}{$[s, e]$}
%  \State Insert $s$ and $e$ into the scapegoat tree. If $s=e$, insert the value only once.
%  %	
%  \State Initialize $L$ to be the empty list.
%  \If{tree needs to be rebalanced}
%  	\State Record which nodes change position in the tree during rebalancing, together with 
%	their old and new encryptions. Namely, record	\[L = \{ \en_1 \leftarrow \en^*_1, \dots, \en_m \rightarrow \en^*_m\},\] where $m$ is the number of nodes who changed position in the tree, and $\en_i$ and $\en^*_i$ are the old and new encryption of the $i$-th node that changed position. 
%  \EndIf
%  \State Compute  $\enc(s)$ and $\enc(e)$, the encryptions of $s$ and $e$, as in EncryptRanges.
%   \State \Return{$[\enc(s), \enc(e)], L$}
%\EndProcedure
%
%\end{algorithmic}
%\end{framed}

%  \State Determing the smallest and the largest encryption in the values $[\enc(s), \enc(e)]$ and $L$, and call this $\dirtyrange$.



\mypara{Rule Updates at the Cloud}
\label{sec:updates}
At the cloud, PrefixMatch poses two additional challenges to updating rules, both stemming from the fact that a rule update can also change encrypted values within packets.
The first challenge is middlebox state. Consider a NAT with a translation table containing ports and IP addresses for active connections
Adding or removing a rule will modify these values in the PrefixMatch tree, and thus to continue correct processing the NAT state must be updated.
The second challenge is a race condition: when the middlebox adopts a new ruleset while packets encrypted under the old tree are still flowing, these packets will be misclassified as their encryption values are inconsistent with the ruleset being applied. 
The same problem occurs if packets encrypted under the new tree arrive before the new ruleset.

To avoid these problems the gateway and cloud provider do the following. 
When the gateway updates the PrefixMatch tree, it announces to the cloud provider of the pending update, and the middleboxes ship their current state to the gateway to receive the new encryption values.
The gateway then sends a signal to the cloud provider that it is about to `swap in' the new map. 
The cloud provider buffers traffic for a few hundred milliseconds after this signal to allow all old traffic to complete processing at the cloud; it signals to all middleboxes to `swap in' the new rules and state; and finally it releases the new traffic.
Note that all changes to middleboxes are in the {\it control plane} of the middlebox, and require no modifications to the algorithms and operations performed in per-packet processing. 

The cloud provider updating middlebox rules will have to write rules appropriately to account for the fact that a 
single prefix maps to multiple intervals. For example, if a middlebox is counting the number of connections to a 
prefix $P$. $P$ maps to a single interval $I$. After updating, a new rule is added which partially overlaps 
with $P$ and hence $I$ becomes $I_1$ and $I_2$. The cloud provider should update rules appropriately; i.e. 
\texttt{if $v$ in $I$ then counter++} becomes \texttt{if $v$ in $I_1$ or $v$ in $I_2$ then counter++}.

\todo{Reread all this section}

\todo{THIS  SUBSECTION IS WORK IN PROGRESS}




\todo{make clear that the same key is used for equality matching, somewhere in the gateway}
\todo{we do 4 to 6 translation}

