%!TEX root = mb.tex 

\section{Introduction}\label{sec:intro}

\newcommand{\myspace}{\rule{0pt}{13em}}

\begin{table*}[t]
\centering
\small
{\renewcommand{\arraystretch}{1.8}%
\begin{tabular}{|>{\raggedright}m{3cm}|>{\raggedright}m{8.7cm}|c|c|}
\hline
{\bf Middlebox} & {\bf Functionality} & {\bf Support} & {\bf Scheme} \\ \hline
IP Firewall & 
\parbox {8cm}{
\begin{flalign*}
& (SIP,\ DIP,\ SP,\ DP,\ P) \in (SIP[],\ DIP[],\ SP[],\ DP[],\ P)  \\ 
& \hspace{5mm} \Leftrightarrow  \Enc(SIP,\ DIP,\ SP,\ DP,\ P) \in \Enc(SIP[],\ DIP[],\ SP[],\ DP[],\ P) 
% & \text{To support existing packet classification algorithms {\em unchanged:}} \\
% &    E_1 \le E_2 \Leftrightarrow \Enc(E_1) \le \Enc(E_2), \text{ where } E \text{ is any of } SIP,\ DIP,\ SP,\ DP,\ P
\end{flalign*}
}
& Yes & PrefixMatch \\  \hline 

NAT (NAPT) & 
\parbox {8cm}{
\begin{flalign*}
(SIP_1, SP_1) = (SIP_2, SP_2) &\Leftrightarrow \Enc(SIP_1, SP_1) = \Enc(SIP_2, SP_2) \\
(SIP_1, SP_1) \neq (SIP_2, SP_2) &\Leftrightarrow \Enc(SIP_1, SP_1) \neq \Enc(SIP_2, SP_2) 
\end{flalign*}
}
& Yes & PrefixMatch \\ \hline

L3 Load Balancer (ECMP) &
\parbox {8cm}{
\begin{flalign*}
&(SIP_1, DIP_1, SP_1, DP_1, P_1)\ =\ (SIP_2, DIP_2, SP_2, DP_2, P_2) \\
\Leftrightarrow &\Enc(SIP_1, DIP_1, SP_1, DP_1, P_1)\ =\ \Enc(SIP_2, DIP_2, SP_2, DP_2, P_2) 
\end{flalign*}
}
& Yes & PrefixMatch \\ \hline

L4 Load Balancer &
\parbox {8cm}{
\begin{flalign*}
&(SIP_1, DIP_1, SP_1, DP_1, P_1)\ =\ (SIP_2, DIP_2, SP_2, DP_2, P_2) \\
\Leftrightarrow &\Enc(SIP_1, DIP_1, SP_1, DP_1, P_1)\ =\ \Enc(SIP_2, DIP_2, SP_2, DP_2, P_2) \\
&(SIP_1, DIP_1, SP_1, DP_1, P_1)\ \neq\ (SIP_2, DIP_2, SP_2, DP_2, P_2) \\
\Leftrightarrow &\Enc(SIP_1, DIP_1, SP_1, DP_1, P_1)\ \neq\ \Enc(SIP_2, DIP_2, SP_2, DP_2, P_2) 
\end{flalign*}
}& Yes & PrefixMatch \\ \hline

HTTP Proxy / Cache &
\parbox {8cm}{
\begin{flalign*}
&\text{Request-URI} \in \text{HTTP Header} \\
\Leftrightarrow \quad &\Enc(\text{Request-URI}) \in \Enc(\text{HTTP Header})
\end{flalign*}
} &
Yes & KeywordMatch \\ \hline

Deep Packet Inspection: & & & \\


$\circ$ Parental Filter &
\parbox {8cm}{
\begin{flalign*}
&\text{Request-URI} \in \text{HTTP Header} \\
\Leftrightarrow \quad &\Enc(\text{Request-URI}) \in \Enc(\text{HTTP Header})
\end{flalign*}
} &
Yes & KeywordMatch \\
%
%
$\circ$ Data Exfiltration /  Watermark Detection 
&
$\text{Watermark} \in \text{Stream} \quad \Leftrightarrow \quad \Enc(\text{Watermark}) \in \Enc(\text{Stream})$
 &
Yes & KeywordMatch \\ 
%
%
$\circ$ Intrusion detection
&
$ \text{Keyword} \in \text{Stream} \quad \Leftrightarrow \quad \Enc(\text{Keyword}) \in \Enc(\text{Stream})$
&
Yes & KeywordMatch \\ 
%
%
&
$\text{RegExpMatch(RegExp, Stream)} \ \Leftrightarrow \ \text{RegExpMatch(} \Enc(\text{RegExp}) \text{, } \Enc(\text{Stream}) \text{)}$
&
Partially & KeywordMatch \\

&
Run intrusion detection scripts on traffic stream
 &
No & KeywordMatch \\



 \hline

\end{tabular}
}
\caption[]{Middleboxes supported by \sys. The second column indicates the functionality desired out of the encryption to support these middleboxes correctly. ``Support'' indicates if the middlebox is supported and ``Scheme'' the encryption scheme used.  {\bf Legend:} $\Enc$ denotes encryption, $SIP$ = source IP address, $DIP$ = destination $IP$, $SP$ = source port, $DP$ = destination port, $P$ = protocol,  $E[]$ = a range of $E$. The $(SIP,\ DIP,\ SP,\ DP,\ P)$ thus denotes the tuple describing a connection. \label{tbl:mbreqs}} 
\end{table*}

%\multicolumn{3}{c}}\\

%%%%%%%%%%%%%%%%%%%





\todo{fix the text below in comments}

%A useful piece of text:
%
%Table 1 presents the ``textbook'' functionality of each middlebox supported. This functionality is the basica functionality
%of this box as presented in textbook .. and RFC ...
%Note that some of these middleboxes have become even more complex, supporting additional functionality not present in
%Table 1. Such functionality may or may be supported by \sys. \sys does not support all complex functionalities out there adn we discuss in \S what we do not support. Make clear that the table spells out precisely what we can support and what not. 
%
%In Table XX, we provide the functionality that \sys provides. This covers the functionality of xxx of above and most of the one
%below. Moreover, in eval, we show that we support at least a real middlebox from each of hte examples above. 
%
%------




Middleboxes such as firewalls, NATs, and proxies, have grown to be a vital part of modern networks, but are 
also widely recognized as bringing significant problems including high cost, inflexibility, and complex management.  
These problems have led both research and industry to explore an alternate approach: moving middlebox functionality out of dedicated boxes and into 
software applications that run multiplexed on commodity server hardware~\cite{mb-manifesto,comb,aplomb,opennf,clickos,flowtags,etsi-nfv,domain20,opnfv}.
This approach -- termed Network Function Virtualization (NFV) in industry -- promises many advantages including the cost benefits of commodity infrastructure, 
the efficiencies of statistical multiplexing, and the flexibility of software solutions. 
In a short time, NFV has gained a significant momentum with over 270 industry participants~\cite{etsi-nfv} and a number of emerging product offerings~\cite{brocade,dell,juniper}.

Leveraging the above trend, several efforts are exploring a new model for middlebox deployment in which a third-party offers middlebox processing as a  
\emph{service}.
Such a service may be hosted in a public cloud~\cite{aplomb,zscaler,aryaka} or in private clouds embedded within an ISP 
infrastructure~\cite{domain20, telefonica}.  
This service model allows customers such as enterprises to ``outsource'' middleboxes from their networks entirely, and hence promises many of the known benefits of cloud computing  such as decreased costs and ease of management.%: decreased costs, ease of management, \etc{}.

However, outsourcing middleboxes brings a new challenge: the confidentiality of the traffic. 
Today, in order to process an organization's traffic, the cloud sees the traffic {\em unencrypted}.  This means that the cloud 
now has access to potentially sensitive packet payloads and headers. This is 
worrisome considering the number of documented data breaches by cloud employees or hackers~\cite{PrivacyRecords,databreach}.
Hence, an important question is: can we enable a third party to process traffic for an enterprise, {\em without seeing the enterprise's traffic}?

To address this, we designed and implemented \sys, the first system to allow an enterprise to outsource  a comprehensive set of enterprise middleboxes  to a cloud provider, while keeping its traffic confidential. 
Middleboxes in \sys operate directly over {\it encrypted} traffic. %without decrypting the traffic. 


In recent work, we designed a system called BlindBox to operate on encrypted traffic for a {\em specific} class of middleboxes: Deep Packet Inspection~\cite{blindbox} -- middleboxes that examine only the payload of packets. 
However, BlindBox is far from sufficient for this setting because
 (1) it has restricted functionality that does not support all middleboxes required for outsourcing, and (2) it has prohibitive performance overheads.
 
 Regarding functionality, BlindBox enables equality-based operations on  encrypted payloads of packets, which supports certain DPI devices. However, this excludes middleboxes such as firewalls, proxies, load balancers, NAT,  and those DPI devices that also examine packet headers, because these need the ability to examine packet headers and/or perform range queries. 
 % I added this packet header thing due to NAT
 % even though some of these MB only do exact, the architecture that puts them all together is new
 Achieving such generality provided new challenges to \sys. 
As we discuss below, such middleboxes require a new encryption scheme and, moreover, a system design that supports all middleboxes {\it simultaneously}: for performance, adoption and extensibility, it does not suffice to craft {\it n} different designs to support {\it n} classes of middleboxes. 

 
Regarding performance, BlindBox has a prohibitive setup time of 97s per-connection. 
\sys's setting targets enterprise outsourcing where BlindBox targets end hosts connecting to arbitrary middleboxes. This difference brings \sys new opportunities: it
not only allows us to remove the setup overhead, but also to achieve better deployability and higher security, as we discuss in \S\ref{sec:bbarch}. 

\sys supports a wide range of middleboxes with practical performance. These cover almost all the middleboxes typically outsourced as surveyed in~\cite{aplomb}. 
Table~\ref{tbl:mbreqs} lists these applications. It also lists the desired functionality from encryption in order to support the functionality of this middlebox.  \sys achieves this functionality through a combination of systems and cryptographic innovations, as follows.
   
From a cryptographic perspective, \sys provides a new and fast encryption scheme called PrefixMatch  to enable the provider to perform prefix matching (\eg{}, if an IP address is in the subdomain 56.24.67.0/16) or port range detection (\eg{}, if a port is in the range 1000-2000). Prior to PrefixMatch, there was no mechanism that provided the functionality, performance, and  security needed in our setting. The closest practical encryption schemes are Order-Preserving Encryption (OPE)~\cite{boldyreva:ope,popa:mope,popa:cryptdb}. 
However, these schemes are four orders of magnitude slower than {\it PrefixMatch} making them unusable for our network setting. At the same time, PrefixMatch provides stronger security guarantees than these schemes, which leak unnecessary information to the provider: for every encrypted value, {\em PrefixMatch} reveals only whether the value lies within a range or not, whereas OPE reveals the total ordering among all values.

  From a systems design perspective, one of the key insights behind \sys is to keep packet formats unchanged: an encrypted IP packet is structured just as a normal IP packet, with each field (e.g., source address) containing an encrypted value of that field.
  This strategy ensures that encrypted packets never appear invalid, e.g., to existing network interfaces, forwarding algorithms, and error checking. 
  We choose which encryption scheme to use for each field based on the processing operations applied by typical middleboxes; this results in an encrypted packet that simultaneously supports {\it all} of the wide range of middleboxes we outsource.
  
  Another systems insight is that by {\it combining} standard packet formats with PrefixMatch, \sys can retain compatibility with existing classification algorithms and obtain good performance for middleboxes operating over header fields.
  PrefixMatch ensures that comparisons (such as $\leq$) function normally, hence, a middlebox receiving an encrypted packet can simply perform the normal set of `fast-path' packet processing operations unmodified.
  Importantly, middleboxes can continue to take advantage of highly-efficient packet classification algorithms~\cite{packet_classif} as already implemented, and which are among the more expensive tasks in software middleboxes~\cite{comb, ethan-paper}.
  Furthermore, even software-based NFV deployments make use of some hardware forwarding components, \eg{} NIC multiqueue flow hashing~\cite{nicdocument}, `whitebox' switches~\cite{whitebox}, and error detection in NICs and switches~\cite{nicdocument, ciscov6}; \sys is compatible with all of these.
  
 
 \begin{table}[t]
\centering
\small
\hspace{-2pt}
\begin{tabular}{c|c}
\hline
{\bf Middlebox} &  {\bf Scheme} \\ \hline
IP Firewall & 
 PrefixMatch \\ 

NAT (NAPT) & 
 PrefixMatch \\

L3 Load Balancer (ECMP) &
 PrefixMatch \\ 

L4 Load Balancer &
 PrefixMatch \\ 

Deep Packet Inspection &

KeywordMatch \\ 

Parental Filter &

KeywordMatch \\ 

HTTP Proxy / Cache &

KeywordMatch \\ 

\end{tabular}
\caption[]{Middleboxes supported by \sys and the encryption scheme used.\label{tbl:mbreqs}} 
\end{table}



We implemented and evaluated \sys on EC2. \sys supports all applications typically deployed by outsourcing as surveyed in~\cite{aplomb}.
Further, \sys imposes  negligible throughput overheads at the service provider: for example, a single-core firewall operating over encrypted data achieves 9.8Gbps, equal to the same firewall over unencrypted data.
Our enterprise gateway can tunnel traffic at 9.6 Gbps on a single core;  a single server can easily support 10Gbps for an small-medium enterprise.

%\eat{Relative to BlindBox, \sys is more general. But even when comparing the two directly, \sys (1) has practical performance for deployment, where BlindBox does not: connection session establishment uses standard handshakes and does not have the 97s overhead that BlindBox does; (2) is more secure: \sys can support 79.8\%-88\% of IDS rules without resorting to decryption, where BlindBox can only support 40-67\%; (3) requires no changes to endhosts, where BlindBox requires a new protocol to be supported by each client.} 


\todo{ and where do we put this? In this paper, we consider the functionality of the middlebox to be the one 
described in textbooks and references\todo{add cite}. We note that there exist versions of these middleboxes with different and quite complex functionality that \sys does not support, and we discuss the limitations of \sys in \S\ref{s:limitations}.}

